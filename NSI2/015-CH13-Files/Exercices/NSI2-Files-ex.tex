\documentclass[a4paper,12pt,french]{article}
\usepackage[margin=2cm]{geometry}
\usepackage[thinfonts]{uglix2}
\nouveaustyle

\begin{document}
\titre{Files - Exercices}{NSI2}{2020} 

\begin{exercice}[ : Implémentation objet]
	\'Ecrire une classe \pythoninline{Queue} qui implémente la structure de file vue en cours. Sauvegarder dans un module \texttt{queue.py}. Celui-ci servira pour les exercices suivants.\\
    
    Vous pouvez nommer les méthodes \pythoninline{enqueue}, \pythoninline{dequeue} et \pythoninline{is_empty}.
\end{exercice}

\begin{exercice}[: 2 piles pour une file]
\'Ecrire une classe \pythoninline{QueueWith2Stacks} qui implémente la structure de file avec 2 piles, comme sur le schéma du cours.
\end{exercice}

\begin{exercice}[ : Maximum d'une file]
    
\'Ecrire une fonction \pythoninline{max_queue} qui
\begin{enumerate}[--]
    \item 	en entrée prend une file composée d'\pythoninline{int};
    \item 	renvoie le maximum de cette file. Attention la file doit être remise dans l'état initial.
\end{enumerate}
\end{exercice}

\end{document}