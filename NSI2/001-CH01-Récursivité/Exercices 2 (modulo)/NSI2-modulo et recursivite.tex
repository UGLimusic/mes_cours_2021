\documentclass[a4paper,12pt,french]{book}
\usepackage[margin=2cm]{geometry}
\usepackage[thinfonts]{uglix2} % ou cmb % NE MARCHERA PAS EN PSTRICKS

\begin{document}

\chapter*{Récursivité et arithmétique}
\introduction{La Mathématique est la reine des sciences et l'Arithmétique est
	la reine des mathématiques.}

\begin{definition}[ : division euclidienne dans \N]

	Soient A et B deux entiers naturels, et $B\neq 0$. Il existe deux nombres uniques Q et R (vérifiant $0\leqslant R<B$) tels que l'on puisse écrire
	$$A = Q\times B + R$$

	C'est exactement la division que l'on a apprise à l'école primaire (celle où l'on s'arrête aux nombres entiers):
	\begin{center}
		\begin{tabular}{r|l}
			A & B\\
			\cline{2-2}
			R & Q
		\end{tabular}

		\begin{enumerate}[\textbullet]
			\item 	A est appelé le \textit{dividende};
			\item 	B est le \textit{diviseur};
			\item	Q est le \textit{quotient};
			\item 	R est le \textit{reste}, il est \textit{impérativement} plus petit que B.
		\end{enumerate}
	\end{center}
\end{definition}

En \textsc{Python} on obtient \pythoninline{Q} en évaluant \pythoninline{A // B} et \pythoninline{R} en évaluant \mintinline{python}{A % B}, cette dernière opération se lit \og \pythoninline{A} modulo \pythoninline{B}\fg{}.Voici un exemple

\begin{pythonshell}
>>> 22 // 7
3
>>> 22 % 7
1
\end{pythonshell}

\begin{exercice}[]
\begin{enumerate}[\bfseries 1.]
	\item 	En \textsc{Python}, écrire une fonction \pythoninline{units_digit} qui
			\begin{enumerate}[--]
				\item 	en entrée prend un \pythoninline{int} positif;
				\item 	renvoie un \pythoninline{int} qui est son chiffre des unités.
			\end{enumerate}
	\item 	De même écrire une fonction \pythoninline{hundreds_digit} pour le chiffre des centaines.
	\item 	De même pour une fonction \pythoninline{thousands_digit} qui renvoie le chiffre des milliers.
\end{enumerate}
\end{exercice}

\begin{exercice}[]
\'Ecrire une fonction récursive \pythoninline{decimal_length}  basée sur \pythoninline{//} et/ou  \mintinline{python} qui
	\begin{enumerate}[--]
		\item 	en entrée prend un \pythoninline{int} positif;
		\item 	renvoie un \pythoninline{int} qui est le nombre de chiffres de l'écriture binaire de ce nombre.
	\end{enumerate}
\end{exercice}


\begin{exercice}[]
	On considère le procédé suivant :
	\begin{enumerate}[--]
	\item 	soit $m\in \N$ un entier écrit en écriture décimale $m = (a_p\cdots a_1a_0)_{10}$,\\
	par exemple $m=\np{31976}$;\\
	\item 	on \og coupe\fg{} cette écriture en deux au niveau des unités : avec $m$ on forme $m_1=(a_p\cdots a_1)_{10}$ et $m_2=(a_0)_{10}$,\\
	pour notre exemple $m_1=\np{3197}$ et $m_2=6$;\\
	\item	On calcule  $m'=m_1-2m_2$ ,\\
	pour notre exemple cela donne $m'=\np{3197}-2\times 6 = 3185$\\
	\end{enumerate}

\begin{enumerate}[\bfseries 1.]
	\item 	à l'aide de \pythoninline{//} et/ou  \mintinline{python}{%} écrire une fonction \pythoninline{f} qui
			\begin{enumerate}[--]
				\item 	en entrée prend un \pythoninline{int} positif \pythoninline{m};
				\item 	en sortie renvoie \pythoninline{m'}.
			\end{enumerate}
	\item 	En fait, la fonction $f$ donne un critère de divisibilité par 7 pour un entier $m$ :
			\begin{enumerate}[--]
				\item 	si  $m\leqslant 70$ alors s'il appartient à {-7; 0; 7; 14; 21; 28; 35; 42; 49; 56; 63; 70}, $m$ est divisible par 7, sinon il ne l'est pas;
				\item 	sinon on regarde si $f(m)$ est divisible par 7.
			\end{enumerate}
	Pour notre exemple on obtient $$31\,976 \mapsto 3\,197 - 2\times 6 = 3\, 185 \mapsto 318- 2\times 5 =308 \mapsto 30 - 2\times 8= 14$$ et on en conclut qu'il est divisible par 7.\\

	Programmer une fonction récursive \pythoninline{is_divisible_by_7} qui
	\begin{enumerate}[--]
		\item 	en entrée prend un \pythoninline{int} positif;
		\item 	en sortie renvoie \pythoninline{True} ou \pythoninline{False} selon que l'entier est divisible par 7.
	\end{enumerate}
	Cette fonction utilisera la fonction \pythoninline{f} définie précédemment.
\end{enumerate}
\end{exercice}
\end{document}
