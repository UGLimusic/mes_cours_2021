\documentclass[a4paper,12pt,french]{article}
\usepackage[margin=2cm]{geometry}
\usepackage[thinfonts,latinmath]{uglix2}
\nouveaustyle
\begin{document}
\titre{Titre}{classe}{date}

On considère un tableau de nombres de $n$ lignes et $p$ colonnes.

Les lignes sont numérotées de $0$ à $n−1$ et les colonnes sont
numérotées de $0$ à $p−1$.

La case en haut à gauche est repérée par $(0,\,0)$ et la case en bas à droite par $(n−1,\,p−1)$.\\

On appelle \textit{chemin} une succession de cases allant de la case $(0,\,0)$ à la case $(n−1,\,p−1)$, en n’autorisant que des
déplacements case par case : soit vers la droite, soit vers le bas.\\


On appelle \textit{somme d'un chemin} la somme des entiers situés sur ce chemin. Par exemple, pour le tableau  $T$ suivant:

$$T=
\begin{tabular}{|c|c|c|c|}
\hline
\textbf{4}&\textbf{1}&\textbf{1}&3\\
\hline
2&0&\textbf{2}&1\\
\hline
3&1&\textbf{5}&\textbf{1}\\
\hline
\end{tabular}$$



\begin{enumerate}[--]
	\item 	un chemin est $(0,\,0),\,(0,\,1),\,(0,\,2),\,(1,\,2),\,(2,\,2),\,(2,\,3)$ (en gras sur le tableau);
	\item 	La somme du chemin précédent est 14;
    \item   $(0,\,0),\,(0,\,2),\,(2,\,2),\,(2,\,3)$ n’est pas un chemin.
\end{enumerate}

L’objectif de cet exercice est de déterminer la somme maximale pour tous les chemins possibles allant de la case $(0,\,0)$
à la case $(n−1,\,p−1)$.

On considère tous les chemins allant de la case $(0,\,0)$ à la case $(2,\,3)$ du tableau $T$ donné en exemple.
\begin{enumerate}[\bfseries 1.]
	\item 	\begin{enumerate}[\bfseries a.]
            	\item 	Un tel chemin comprend nécessairement 3 déplacements vers la droite. Combien de déplacements vers le bas comprend-il ?
                \begin{encadre}[Réponse]
                Un tel chemin part de la ligne 0 et doit arriver à la ligne 2, il doit nécessairement descendre 2 fois.
                \end{encadre}

            	\item 	 La longueur d'un chemin est égal au nombre de cases de ce chemin. Justifier que tous les chemins allant de $(0,\,0)$ à $(2,\,3)$ ont
                        une longueur égale à 6.
                 \begin{encadre}[Réponse]
                 Nous savons que tous les chemins comportent 5 déplacements, chacun d'entre eux les menant à une nouvelle case. En tenant compte de la case de départ, chaque chemin est de longueur 6.
                 \end{encadre}
             \end{enumerate}

    \item En listant les chemins possibles allant de $(0,\,0)$ à $(2\,,3)$ du tableau $T$, déterminer un chemin qui permet d'obtenir la somme maximale et donner la valeur de cette somme.
    \begin{encadre}[Réponse]
    Listons les chemins et donnons leur somme :
    \begin{enumerate}[--]
    	\item 	$(0,0),(0,1),(0,2),(0,3),(1,3),(2,3)$ a pour somme 11;

    	\item 	$(0,0),(0,1),(0,2),(1,2),(1,3),(2,3)$ a pour somme 12;
        \item 	$(0,0),(0,1),(0,2),(1,2),(2,2),(2,3)$ a pour somme 15;
        \item   \textit{Et cætera}.
     \end{enumerate}
     Un chemin donnant

    \end{encadre}

    \item On veut créer le tableau $T_2$ où chaque élément $T_2[i][j]$ est la somme maximale pour tous les chemins possibles allant de $(0,\,0)$ à $(i,\,j)$.
    \begin{enumerate}[\bfseries a.]
    	\item 	Compléter sur votre copie le tableau $T_2$ ci-dessous associé au tableau $T$ suivant
                $$T=
                \begin{tabular}{|c|c|c|c|}
                \hline
                4&1&1&3\\
                \hline
                2&0&2&1\\
                \hline
                3&1&5&1\\
                \hline
                \end{tabular}\qquad\qquad\qquad T_2=
                                \begin{tabular}{|c|c|c|c|}
                                \hline
                                4&5&6&?\\
                                \hline
                                6&?&8&10\\
                                \hline
                                9&10&?&16\\
                                \hline
                                \end{tabular}$$
    	   \item Justifier que si $j$ est différent de zéro alors :
           $$T_2[0][j] = T[0][j] + T_2[0][j - 1]$$
    \end{enumerate}
    \item Justifier que si $i$ et $j$ sont différents de 0 alors :\\
            $$T_2[i][j] = T[i][j]+\max(T_2[i-1][j],\, T_2[i][j-1])$$

     \item On veut créer une fonction récursive \pythoninline{somme_max} qui
     \begin{enumerate}[--]
     	\item 	en entrée prend un tableau \pythoninline{T} (qui est une liste de lignes, elles-même des listes d'\pythoninline{int}), et 2 \pythoninline{int i} et \pythoninline{j};
     	\item 	renvoie la somme maximale pour tous les chemins possibles allant de la case $(0,\,0)$ à la case $(i,\,j)$.
      \end{enumerate}
      \begin{enumerate}[\bfseries a.]
        \item Quel est le cas d'arrêt, c'est-à-dire le cas qui est traité directement, sans appel récursif ? Que renvoie-t-on dans ce cas ?
        \item À l'aide de la question précédente, écrire en Python la fonction récursive \pythoninline{somme_max}.
        \item Quel appel de fonction doit-on faire pour résoudre le problème initial ?
     \end{enumerate}
\end{enumerate}

\end{document}




\end{document}
