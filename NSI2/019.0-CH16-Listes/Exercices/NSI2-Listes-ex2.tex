\documentclass[a4paper,12pt,french]{article}
\usepackage[margin=2cm]{geometry}
\usepackage[thinfonts]{uglix2}
\nouveaustyle

\begin{document}
\titre{Listes - Exercices supplémentaires}{NSI2}{2021}


\begin{exercice}
\'Ecrire une méthode \texttt{extrait\_inf} qui
\begin{enumerate}[--]
	\item 	prend en paramètre un entier \texttt{n};
	\item 	extrait de la liste  toutes les cellules contenant des valeurs inférieures à l’entier n;
    \item les cellules enlevées ne seront pas supprimées physiquement mais elles formeront une autre liste que méthode renverra.
\end{enumerate}
Par exemple, appliquée à la liste \texttt{< 5, 7, 2, 1, 9, 8, 10, 15, 4, 20 >} et à l’entier 6, la fonction doit construire et renvoyer la nouvelle liste \texttt{< 5, 2, 1, 4 >} et elle doit transformer la liste de départ en  \texttt{< 7, 9, 8, 10, 15, 20 >}.
\end{exercice}



\begin{exercice}
Dans un nouveau fichier, implémenter la structure de liste doublement chaînée (aller voir sur Internet pour la définition).
\end{exercice}

\begin{exercice}
Faire de même avec une liste doublement  chaînée circulaire.
\end{exercice}
\end{document}