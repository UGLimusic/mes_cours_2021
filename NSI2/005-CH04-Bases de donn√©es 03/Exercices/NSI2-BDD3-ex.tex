\documentclass[a4paper,12pt,french]{article}
\usepackage[margin=2cm]{geometry}
\usepackage[thinfonts]{uglix2}
\usepackage{ulem}
\setminted{fontsize=\small}
\nouveaustyle
\begin{document}
	\titre{CH48 - BDD partie 3 - Exercices}{NSI2}{2021}
\begin{exercice}[]
Ouvrir le fichier \texttt{create\_Auteurs.sql} qui a servi à créer la BDD du cours (avec le bloc-notes ou autre).
\begin{enumerate}[\bfseries 1.]
	\item 	Trouver une écriture compacte telle que vue dans le cours.
	\item 	Trouver une écriture compacte qui n'avait pas été vue.
    \item 	En fouillant \og à vue d'\oe il\fg{} dans les données
    \begin{enumerate}[\bfseries a.]
    	\item 	Combien y a-t-il de livres ?
    	\item 	Combien y a t-il d'auteur ?
        \item 	Où voit-on clairement qu'un auteur a écrit 2 livres ?
    \end{enumerate}
\end{enumerate}
\end{exercice}
\begin{exercice}[]
\begin{enumerate}[\bfseries 1.]
	\item Voici le modèle relationnel de l'exercice \textbf{Hotel Reservation Client Chambre} du chapitre précédent :\\

    \textbf{Hotel}(\uline{nom\_hotel TEXT}, adresse TEXT)\\

    \textbf{Chambre}(\uline{num\_ch INTEGER}, prix INTEGER, \dashuline{nom\_hotel TEXT})\\

    \textbf{Client}(\uline{num\_client INTEGER}, nom\_client TEXT)\\


    \textbf{Reservation}(\uline{num\_resa INTEGER}, date\_resa DATE,\dashuline{ num\_client INTEGER},\dashuline{ num\_chambre INTEGER})\\

    Écrire le script SQL qui crée cette BDD.
    \item Insérer 3 valeurs de votre choix dans chaque table.
\end{enumerate}
\end{exercice}
\begin{exercice}[]
On considère une base de données qui ne contient aucune table.\\
Dire quelles erreurs produisent les séquences SQL suivantes (tout au moins ce qui n'est pas acceptable du point de vue du cours).
\begin{enumerate}[\bfseries 1.]
\item 	\begin{minted}{sql}
DROP TABLE Client;
CREATE TABLE Client
(
    cid             INTEGER,
    nom             TEXT,
    prenom          TEXT,
    points_fidelite INTEGER NOT NULL,
        PRIMARY KEY (cid),
    CHECK (points_fidelite >= 0)
);
    \end{minted}
\item 	\begin{minted}{sql}
CREATE TABLE Client
(
    cid             INTEGER PRIMARY KEY,
    nom             TEXT,
    prenom          TEXT,
    points_fidelite INTEGER NOT NULL,
    CHECK (points_fidelite >= 0)
);
CREATE TABLE Commande
(
    cid  INTEGER PRIMARY KEY, -- variante plus rapide et valide
    pid  INTEGER,
    date TEXT NOT NULL,
    FOREIGN KEY (cid) REFERENCES Client (cid),
    FOREIGN KEY (pid) REFERENCES Client (pid)
);
CREATE TABLE Produit
(
    pid  INTEGER PRIMARY KEY,
    nom  TEXT,
    prix REAL(10, 2) -- 10 chiffres max avant la virgule, 2 après
);
\end{minted}

\item \begin{minted}{sql}
CREATE TABLE Client
(
    cid             INTEGER PRIMARY KEY,
    nom             TEXT,
    prenom          TEXT,
    points_fidelite INTEGER NOT NULL,
    CHECK (points_fidelite >= 0)
);
CREATE TABLE Produit
(
    pid  INTEGER PRIMARY KEY,
    nom  TEXT,
    prix REAL(10, 2)
);
CREATE TABLE Commande
(
    cid  TEXT REFERENCES Client (cid),
    nomp  INTEGER REFERENCES Produit (nom),
    date TEXT NOT NULL,
    PRIMARY KEY (cid, pid)
);
\end{minted}

\item	\begin{minted}{sql}
CREATE TABLE Client
(
    cid             INTEGER PRIMARY KEY,
    nom             TEXT,
    prenom          TEXT,
    points_fidelite INTEGER NOT NULL,
    CHECK (points_fidelite >= 0)
);
CREATE TABLE Produit
(
    pid  INTEGER PRIMARY KEY,
    nom  TEXT,
    prix REAL(10, 2)
);
CREATE TABLE Commande
(
    cid  INTEGER,
    pid  INTEGER,
    date TEXT NOT NULL,
    PRIMARY KEY (cid, nomp),
    FOREIGN KEY (cid) REFERENCES Client (cid),
    FOREIGN KEY (pid) REFERENCES Produit (pid)
);

INSERT INTO Commande VALUES(0, 0, '2020-03-02')
\end{minted}
\end{enumerate}
\end{exercice}
\end{document}