\documentclass[a4paper,11pt,french]{book}
\usepackage[margin=2cm]{geometry}
\usepackage[thinfonts]{uglix2}

\pagestyle{empty}
\nouveaustyle
\begin{document}
\titre{BDD - Feuille d'exercices 4}{NSI2}{2021}\\

\exo{}

Liste les différents attributs de cette
relation. Donne le domaine de chaque
attribut. Pour chaque attribut dire s'il peut jouer ou non le rôle de clef primaire et pourquoi.
\begin{center}
\textbf{Film}\\[1em]

\begin{tabular}{|c|c|c|c|c|}
\hline
\rowcolor{UGLiOrange} \textbf{\color{white}id} &\textbf{\color{white}titre}&\textbf{\color{white}realisateur}&\textbf{\color{white}ann\_sortie}&\textbf{\color{white}note\_sur\_10}\\
\hline
1&Alien, le huitième passager &Scott &1979 &10\\
2&Dune&Lynch&1985&5\\
3&2001 : l'Odysée de l'Espace&Kubrick&1968&9\\
4&Blade Runner&Scott&1982&10\\
\hline
\end{tabular}\\[2em]
\end{center}
\exo{}

Indique les attributs qui peuvent servir de lien entre ces deux relations.

\begin{center}
\textbf{Auteur}\\[1em]

\begin{tabular}{|c|c|c|c|c|}
\hline
\rowcolor{UGLiOrange} \textbf{\color{white}id} &\textbf{\color{white}nom}&\textbf{\color{white}prenom}&\textbf{\color{white}ann\_naiss}&\textbf{\color{white}langue\_ecriture}\\
\hline
1&Orwell&George&1903&anglais\\
2&Herbert&Frank&1920&anglais\\
3&Asimov&Isaac&1920&anglais\\
4&Barjavel&René&1911&français\\
5&Verne&Jules&1828&français\\
...&...&...&...&...\\
\hline
\end{tabular}\\[2em]
\end{center}

\begin{center}
\textbf{Livre}\\[1em]

\begin{tabular}{|c|c|c|c|c|}
\hline
\rowcolor{UGLiOrange} \textbf{\color{white}id} &\textbf{\color{white}titre}&\textbf{\color{white}id\_auteur}&\textbf{\color{white}ann\_publi}&\textbf{\color{white}note}\\
\hline
...&...&...&...&...\\
34&La nuit des temps&4&1968&7\\
35&De la Terre à la Lune&5&1865&10\\
36&Les Robots&6&1950&9\\
...&...&...&...&...\\
\hline
\end{tabular}\\[2em]
\end{center}
\exo{}

\begin{enumerate}[\bfseries 1.]
	\item 	En partant de la relation \textbf{Film} ci-dessus, crée
    une relation \textbf{Realisateur} (attributs de la
    relation: \tw{id}, \tw{nom}, \tw{prenom} et
    \tw{ann\_naissance}, tu trouveras toutes les
    informations nécessaires sur le Web).
    Modifie ensuite la relation \textbf{Film} afin d'établir
    un lien entre les relations \textbf{Film} et
    \textbf{Realisateur}. Tu préciseras l'attribut qui
    jouera le rôle de clef étrangère.

	\item 	\'Ecris le code \textsc{SQL} permettant de générer les 2 tables.
    \item   \'Ecris le code \textsc{SQL} pour insérer les films et les réalisateurs correspondants.
\end{enumerate}
\end{document}