\documentclass[a4paper,12pt,french]{book}
\usepackage[margin=2cm]{geometry}
\usepackage[thinfonts]{uglix2}

\setminted{fontsize=\small}\nouveaustyle
\begin{document}
\titre{BDD partie 4 - Corrigé}{NSI2}{2020} 


\textbf{Exploration de la base}\\	

En explorant la structure de la base de données, répondez aux questions suivantes :

\begin{enumerate}[\bfseries 1.]
	\item 	Combien de tables possède la base de données ? 
            \begin{corrige}
            La BDD possèdes 2 table : \texttt{Departement} et \texttt{Ville}.
            \end{corrige}
	\item 	Pour la table \textbf{Departement}
	\begin{enumerate}[\bfseries a.]
		\item 	Identifiez le type de chaque attribut.
		\item 	Quelle est la clé primaire ?	
	\end{enumerate}	
            \begin{corrige}
            \texttt{idDepartement} est un entier, c'est la clé primaire. \texttt{numero} et \texttt{nom} sont des chaines de caractères.
            \end{corrige}
	\item Pour la table \textbf{Ville}
		\begin{enumerate}[\bfseries a.]
		\item 	Identifiez le type de chaque attribut.
		\item 	Quelle est la clé primaire ?
		\item 	Quelle est la clé étrangère et quel attribut référence-t-elle ?	
	\end{enumerate}	
    \begin{corrige}
    \texttt{idVille} est un entier et c'est la clé primaire. \texttt{nom} est une chaine de caractères, \texttt{codePostal}, \texttt{nbHabitants} et \texttt{idDepartement} sont des entiers et ce dernier attribut est une clé étrangère faisant référence à l'attribut \texttt{idDepartement} de la table précédente.
    \end{corrige}
	\item 	 Réalisez un schéma relationnel de cette base de données, sous la forme graphique, en précisant pour chaque attribut son type et s’il doit impérativement être rempli.
    \begin{corrige}
    Désolé je n'ai pas eu le courage. Demain je demanderai des schémas sous forme de lignes.
    \end{corrige}
\end{enumerate}


\textbf{Collecte des informations}\\

La base \textbf{Collectivites} est vide. Il faut la remplir. Pour cela, en vous aidant de Wikipedia, complétez les tableaux suivants.
C'est à vous de remplir comme bon vous semble les colonnes idCommune et idDepartement.
\begin{center}
\begin{tabular}{|c|c|c|c|c|}
	\hline
	\rowcolor{UGLiOrange}\color{white} idCommune &\color{white}	\textbf{Commune} & \color{white}\textbf{Code postal} & \color{white}\textbf{Département} &\color{white} \textbf{Nombre d'habitants} \\
	\hline\color{black}
	&Rouen &  &  &  \\
	\hline
	&Dieppe &  &  &  \\
	\hline
	&Envermeu &  &  &  \\
	\hline
	&Le Neubourg &  &  &  \\
	\hline
	&Igoville &  &  &  \\
	\hline
\end{tabular}\\[1em]

\begin{tabular}{|c|c|c|}
	\hline
\rowcolor{UGLiOrange}\color{white} idDepartement &\color{white}	\textbf{Département} & \color{white}\textbf{Code d'immatriculation} \\
	\hline
	&Seine-Maritime  & \\
	\hline
	&Eure&  \\
\hline
	&Calvados&  \\
\hline

\end{tabular}
\end{center}

\textbf{Exploitation des informations}\\

En utilisant l’onglet \og Exécuter le SQL\fg{}, indiquez le code SQL permettant de répondre aux consignes suivantes :\\
\begin{enumerate}[\bfseries 1.]
	\setcounter{enumi}{4}
	\item 	Insérez le département de Seine-Maritime.
            \begin{sql}
INSERT INTO Departement VALUES (1, '76', 'Seine-Maritime');
            \end{sql}
	\item 	Insérez la commune de Rouen.
\begin{sql}
INSERT INTO Ville VALUES (1, 'Rouen', 76000, 110145, 1);
\end{sql}
	\item 	Faîtes de même, en une seule requête, avec les communes de Dieppe et d’Envermeu.
	\item 	Insérez la commune d’Igoville.
	\item 	Insérez la commune du Neubourg.	\\
\end{enumerate}
\begin{corrige}
C'est pareil...
\end{corrige}
En recherchant éventuellement les informations manquantes, indiquez le code SQL permettant de répondre aux consignes suivantes :\\
\begin{enumerate}[\bfseries 1.]
		\setcounter{enumi}{9}
	\item 	Trouville, Mézidon-Canon et Crèvecoeur-en-Auge sont des villes du Calvados.
    \begin{sql}
INSERT INTO Ville VALUES 
(6,'Trouville', 14360, 4628, 3),
(7,'Mézidon-Canon', 14270, 4838, 3),
(8,'Crèvecoeur-en-Auge', 14340, 480, 3);    
\end{sql}
	\item 	Le vrai nom de Trouville est en fait Trouville-Sur-Mer.	
\begin{sql}
UPDATE Ville
SET nom = 'Trouville-Sur-Mer'
WHERE nom = 'Trouville';
\end{sql}
	\item 	Médizon-Canon et Crèvecoeur-en-Auge n’existent plus. Elles ont fusionné pour donner une nouvelle commune : Mézidon-Vallée-d’Auge.\\
\begin{sql}
DELETE FROM Ville
WHERE nom = 'Médizon-Canon' OR nom = 'Crèvecoeur-en-Auge';
INSERT INTO Ville VALUES 
(9,'Mézidon-Vallée-d’Auge', 14431, 9 712, 3),

\end{sql}
\end{enumerate}
\textbf{Vérification des données}\\

En trouvant les requêtes SQL adéquates, répondez aux questions suivantes.

\begin{enumerate}[\bfseries 1.]
	\setcounter{enumi}{12}
\item  Combien il y a-t-il de départements différents enregistrés dans la base ? (réponse : 3)
\begin{sql}
SELECT COUNT(*) from Departement
\end{sql}
\item Combien il y a-t-il de communes différentes enregistrées dans la base ? (réponse : 7)
\begin{sql}
SELECT COUNT(*) from Ville
\end{sql}
\item Combien il y a-t-il de communes dans l’Eure enregistrées dans la base ?
\begin{sql}
SELECT COUNT(*) from Ville
    JOIN Departement ON Ville.idDepartement = Departement.idDepartement
    WHERE Departement.nom = 'Eure';
\end{sql}

\item  Combien il y a-t-il de communes en Seine-Maritime enregistrées dans la base ? 
\begin{corrige}
C'est pareil...
\end{corrige}
\item Combien il y a-t-il de communes dans le Calvados enregistrées dans la base ? 
\begin{corrige}
C'est pareil...
\end{corrige}
\end{enumerate}






\end{document}