\documentclass[a4paper,12pt,french]{article}
\usepackage[margin=2cm]{geometry}
\usepackage[thinfonts,latinmath]{uglix2}
\usepackage{ulem}
\nouveaustyle
\begin{document}
\titre{Titre}{classe}{date}

\exo{}

\textbf{Partie A : Réseau}

\begin{enumerate}[\bfseries 1.]
\item 	Le terme \textit{architecture} fait référence au matériel. Lors d'une communication, les données sont découpées en \textit{paquets}.
Le terme qui désigne l'ensemble des règles de communication utilisées pour réaliser un service particulier sur le
réseau est \textit{protocole}.
\item 	L'élément A joue le rôle de \textit{passerelle} entre un réseau local et le réseau global : c'est un \textit{routeur}.
B et C sont quant à eux des \textit{switch}.
\item Le poste 3 est sur le réseau local, 192.168.11.0, donc la ligne peut être :
\begin{center}
\begin{tabular}{|c|c|c|c|}
\hline
\textbf{Matériel} &\textbf{Adresse IP}& \textbf{Masque}&\textbf{Passerelle}\\
\hline 

Poste 3  & 192.168.11.xx & 255.255.255.0 & 192.168.11.1 \\
\hline
\end{tabular}
\end{center}

où xx est n'importe quel nombre compris entre 2 et 9 inclus, et 22 et 253 inclus.

22 semble être le choix logique.\\ 

\end{enumerate}

\textbf{Partie B : Routage réseaux}

\begin{enumerate}[\bfseries 1.]
\item 	 Les réseaux \textit{directement connectés }à R1 sont ceux qui sont à distance nulle. Leurs IP sont
\begin{enumerate}[--]
\item 	10.0.0.0
\item 	 172.16.0.0
\item	192.168.0.0
\end{enumerate} 
\item 		Il suffit de repérer à quels réseaux les adresses IP appartiennent :
\begin{center}
\begin{tabular}{|c|c|}
\hline
\textbf{Adresse IP destination}&\textbf{Interface Machine ou Port}\\
\hline
192.168.1.55            & 192.168.0.1\\
\hline
172.18.10.10            & 172.15.0.1 \\               
\hline
\end{tabular}
\end{center}
\item	Il s'agit bien du protocole RIP puisqu'on a aucune information sur les débits des connexions.
\begin{center}
\begin{tabular}{|c|c|c|}
\hline
\textbf{Routeur}&\textbf{Métrique}&\textbf{Route}\\
\hline
R2 & 0 & R1 - R2 \\
\hline
R3 & 0 & R1 - R3 \\
\hline
R4 & 1 & R1 - R2 - R4 \\
\hline
R5 & 1 & R1 - R3 - R4 \\
\hline
R6 & 1 & R1 - R3 - R6 \\
\hline
R7 & 2 & R1 - R2 - R4 - R7 \\
\hline
\end{tabular}
\end{center}
Pour la ligne de R7 on aurait tout aussi bien pu indiquer R1 - R3 - R6 - R7.

\end{enumerate}


\exo{}

\begin{enumerate}[\bfseries 1.]
\item 	La liaison n'est pas valide car dans cette liste figure \pythoninline{["Luchon", "Muret"]}, qui n'est pas une \textit{liaison directe}.
\item 	\begin{minted}{python}
liaisonJoueur1 = [["Tarbes", "St Gaudens"],
                  ["Toulouse", "Castres"],
                  ["Toulouse", "Castelnaudary"],
                  ["Castres", "Mezamet"],
                  ["Castelnaudary", "Carcassonne"]] 
\end{minted}
\item \begin{enumerate}[\bfseries a.]
\item 	\pythoninline{assert listeLiaisons, "la liste est vide"} 
\item 	 \pythoninline{construireDict(listeLiaison)} renvoie
\begin{minted}{python}
{
    'Toulouse': ['Muret', 'Montauban'],
    'Gaillac': ['St Sulpice'],
    'Muret': ['Pamiers']
}
\end{minted}
Ce n'est pas la bonne valeur, car il manque des clés.

En fait, telle que la fonction est programmée, les villes figurant en deuxième position ne sont jamais mises en clé.
\item On peut procéder de plusieurs manières.

On peut rendre le code symétrique en rajoutant ce bloc en ligne 16 :

\begin{minted}{python}
if not villeB in Dict.keys():
    Dict[villeB] = [villeA]
else:
    destinationsB = Dict[villeB]
    if not villeA in destinationsB:
        destinationsB.append(villeA)
\end{minted}

Toutefois, une solution bien plus courte et élégante consiste à remplacer la ligne 7 par :

\begin{minted}[fontsize=\small]{python}
for liaison in listeLiaisons + [[y, x] for [x, y] in listeLiaisons]:
\end{minted}
Ce qui « symétrise » la liste sur laquelle on itère.
\end{enumerate}
\end{enumerate}

\exo{}

\begin{enumerate}[\bfseries 1.]
	\item 	Le langage utilisé est le SQL (\textit{Simple Query Language}).
	\item   \begin{enumerate}[\bfseries a.]
	           \item 	\texttt{ATOMES} a pour attributs
                        \begin{enumerate}[--]
                            \item 	Z : INT
                            \item nom : VARCHAR
                            \item Sym : VARCHAR
                            \item  L : INT
                            \item C : INT
                            \item  masse\_atom : DECIMAL
                        \end{enumerate}
                        \textbf{VALENCE} a pour attributs
                        \begin{enumerate}[--]
                           	\item 	Col : INTEGER
                           	\item 	Couche : TEXT
                        \end{enumerate}
            	\item 	Les attributs \texttt{Z}, \texttt{nom} et \texttt{Sym} peuvent jouer le rôle de clé primaire car chacun de ces attributs identifie de manière
                        unique un élément.

                        C peut être une clé étrangère faisant réference à \texttt{Col}, clé primaire de \texttt{VALENCE}.
                \item 	On a le schéma suivant :

                        \textbf{ATOMES}(\uline{Z INT}, nom VARCHAR , Sym VARCHAR , L INT, \dashuline{C INT}, masse\_atom DEC )\\

                        VALENCE(\uline{COL INT}, Couche TEXT)
            \end{enumerate}

    \item 	\begin{enumerate}[\bfseries a.]
               	\item 	 On obtient la table suivante :
                        \begin{center}
                            \begin{tabular}{|c|}
                                \hline
                                \textbf{nom}\\
                                \hline
                                aluminium \\
                                \hline
                                 argon\\
                                \hline     
                                 chlore    \\
                                \hline
                                  magnesium\\
                                \hline
                                sodium\\
                                \hline
                                phosphore\\
                                \hline
                                soufre\\
                                \hline
                                silice\\
                                \hline
                            \end{tabular}
                        \end{center}
           	    \item  On obtient une table avec l'attribut \texttt{C}, où tous les entiers de 1 à 18 sont présents.
            \end{enumerate}
    \item 	\begin{enumerate}[\bfseries a.]
    	   \item
\begin{minted}{sql}
SELECT  nom,masse_atom FROM ATOMES;
\end{minted}
            \item
\begin{minted}{sql}
SELECT Sym FROM ATOMES
JOIN VALENCE ON ATOMES.C = VALENCE.col
WHERE VALENCE.couche = "s";
\end{minted}
        \end{enumerate}
    \item 
\begin{minted}{sql}
UPDATE ATOMES
SET masse_atom = 39.948
WHERE Sym = "Ar";
\end{minted}
\end{enumerate}

\exo{}

\begin{enumerate}[\bfseries 1.]
	\item 	\begin{enumerate}[\bfseries a.]
            	\item 	Un fichier CSV (\textit{Comma Separated Values}) est un fichier texte contenant des informations (liste ou liste de listes) séparées par des virgules.
            	\item 	\pythoninline{prenom} et la valeur de retour sont de type \pythoninline{str}.
            \end{enumerate}
	\item 	 \begin{enumerate}[\bfseries a.]
            	\item 	\pythoninline{import csv}
            	\item 	\pythoninline{assert type(prenom) is str, "L'argument doit être de type str."}
                \item 	Voici ce que l'on peut faire :
\begin{minted}{python}
def genre(prenom):

    if not (type(prenom) is str):
        return ""

    liste_M = ['f', 'd', 'c', 'b', 'o', 'n', 'm', 'l', 'k',
                'j', 'é', 'h', 'w', 'v', 'u', 't', 's', 'r',
                'q', 'p', 'i', 'þ', 'z', 'x', 'ç', 'ö', 'ã',
                'â', 'ï', 'g']
    liste_F = ['e', 'a', 'ä', 'ü', 'y', 'ë']

    if prenom[len(prenom)-1].lower() in liste_M :
        return "M"
    elif prenom[len(prenom)-1].lower() in liste_F :
        return "F"
    else :
        return "I"                        
\end{minted}
Ainsi si l'argument n'est pas un str, la fonction renvoie \pythoninline{""} mais ne produit pas d'erreur.
\end{enumerate}
\item Il suffit de remplacer \pythoninline{prenom[len(prenom)-1]} par \pythoninline{prenom[-2:]}.
\end{enumerate}
\end{document}
