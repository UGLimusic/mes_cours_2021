\documentclass[a4paper,12pt]{article}
\usepackage[margin=2cm]{geometry}
\usepackage[thinfonts]{uglix2}
\pagestyle{empty}
\usepackage{ulem}
\nouveaustyle
\begin{document}
\titreinterro{Interrogation 02}{NSI2}{09/2021}

\exo{}\\

La SA Bulot utilise la base de données \textbf{GestSalar} pour gérer son personnel. Elle vous communique le schéma relationnel relatif à la base de données :\\

\textbf{Etablissement}(\uline{id\_etab}, ville\_etab)\\

\textbf{Employe}(\uline{id\_emp}, nom\_emp, qualif\_emp, date\_embauch\_emp,\dashuline{id\_etab}, \dashuline{id\_poste})\\

\textbf{Poste} (\uline{id\_poste}, libelle\_poste, \dashuline{id\_cat})\\

\textbf{Categorie}(\uline{id\_cat}, libelle\_cat)\\

\begin{enumerate}[\bfseries 1.]
	\item 	Un établissement se situe-t-il dans une et une seule ville ? Justifie ta réponse.\\

    \carreauxseyes{16}{4}\\
	\item 	Dans les faits, chaque employé occupe un poste de travail précis. Le schéma relationnel traduit-il bien ce fait ? Justifie ta réponse.\\

        \carreauxseyes{16}{4}\\

    \item   On voudrait ajouter une relation \textbf{Salaire}, qui permettrait d'intégrer les salaires de chaque employés pour chaque mois de la manière suivante :
            \begin{enumerate}[--]
            	\item 	le salaire de base;
            	\item 	les primes pour le mois;
            \end{enumerate}
            Écrire la relation \textbf{Salaire}.\\

                \carreauxseyes{16}{4}\\
\end{enumerate}

\exo{}\\

Son professeur a demandé à Jules Le Men, étudiant en Licence d'Informatique, de produire un script SQL pour créer une base de données permettant de
stocker des informations sur un étudiant et les diplômes qu'il prépare.

Voici ce que Jules a implémenté :

\begin{sql}
DROP TABLE IF EXISTS Etudiant;
DROP TABLE IF EXISTS Diplome;

CREATE TABLE Diplome
(
    id_dip INTEGER PRIMARY KEY,
    nom_dip TEXT
);

CREATE TABLE Etudiant(
    nom_etu TEXT PRIMARY KEY,
    date_naiss_etu TEXT,
    diplome_etu TEXT REFERENCES Diplome(id_dip)
);

INSERT INTO Diplome VALUES
    (1, "Licence d'Informatique"),
    (2, "BTS SIO");

INSERT INTO Etudiant VALUES
    ("Le Men", "2000-07-28",1),
    ("Desloges", "2002-11-20",2);
\end{sql}


\begin{enumerate}[\bfseries 1.]
	\item 	Explique pourquoi selon ce modèle, Antoine Le Men, le frère de Jules, ne peut pas préparer de diplôme.\\

                \carreauxseyes{16}{4}\\

	\item 	Explique pourquoi Killian Desloges ne peut pas s'incrire pour préparer une licence de philosophie en plus de son cursus actuel.\\

                    \carreauxseyes{16}{4}\\

    \item   Reprend le schéma relationnel décrit par le SQL et modifie-le pour que les 2 problèmes précédents soient réglés .Tu dois donc écrire un modèle relationnel (pas besoin d'indiquer les types), pas un script SQL.\\


                    \carreauxseyes{16}{4}
 \newpage
\carreauxseyes{16.8}{25.6}
\end{enumerate}
\end{document}
