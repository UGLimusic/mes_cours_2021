\documentclass[a4paper,10pt,french]{book}
\usepackage[margin=2cm]{geometry}
\usepackage[thinfonts]{uglix2}
\setminted{fontsize=\small}\nouveaustyle
\begin{document}
\titre{DS 02 (suite et fin) corrigé}{NSI2}{11/2021}

\exo{ - Analyse de code}


Voici ce qui se passe lors de l'utilisation du script :

\pythonfile{scripts/test_alarm.py}

Donc les SMS envoyés sont
\begin{enumerate}[\textbullet]
	\item 	\texttt{'971971971 Loritz : 0002 Intrusion'}
	\item 	\texttt{'971971971 Poincaré : 0004 Intrusion'}
\end{enumerate}

Et en définitive, \pythoninline{alarme1.journal} vaut :\\

\texttt{['0001 Activation ', '0002 Intrusion envoi sms au 971971971', '0003 Désactivation ']}\\

Dans le script donné en énoncé, on constate que la méthode \texttt{intrusion} est mal codée : on n'enregistre les évènements que lorsque l'alarme
est active. Pour remédier à cela, il suffit de désindenter la ligne 20 : ainsi on consigne l'événement dans le journal quoi qu'il advienne.\\

Pour en plus gérer les envois de SMS échoués on aboutit au code suivant pour la méthode \texttt{intrusion} :

\begin{pythoncode}
def intrusion(self):
    evenement = date() + " Intrusion"
    if self.active:
        sms = self.lieu + ' : ' + evenement
        if envoie_sms(self.telephone, sms):
            evenement = evenement + " envoi sms au " + self.telephone
        else:
            evenement += " envoi au " + self.telephone + "échoué"
    self.journal.append(evenement)
\end{pythoncode}

Et pour vider le journal voici la méthode la plus simple :

\begin{pythoncode}
def efface_journal(self):
    self.journal = []
\end{pythoncode}

\exo{ - Cercles}\\

La principale difficulté est se déterminer à quelle condition deux cercles se chevauchent (c'est-à-dire ont un point d'intersection) d'une part,
et d'autre part à quelle condition un cercle est contenu dans l'autre.


\begin{center}
\def\uscl{.5}
\begin{tikzpicture}[scale=\uscl]
\draw[blue] (0,0)\ball node[above]{O} circle(3);
\draw[red] (4,0)\ball node[above]{O'} circle (2);
\draw(0,0)--(4,0);
\draw[blue,<->] (0,0) --node[midway,above]{r} (3,0);

\draw[red,<->] (4,0) -- node[midway,above]{r'}(6,0);
\end{tikzpicture}
\end{center}
Quand les cercles se chevauchent on a $OO' < r + r'$ mais il ne faut pas non plus que $OO'$ soit trop petit :

\begin{center}
\def\uscl{.5}
\begin{tikzpicture}[scale=\uscl]
\draw[blue] (0,0)\ball node[above]{O} circle(3);
\draw[red] (0.5,0)\ball node[above]{O'} circle (1);
\draw (0,0) -- (0.5,0);
\draw[<->,blue] (0,0) -- node[midway,above]{r}(3,0);
\draw[red,<->] (.5,0) -- node[midway,above]{r'}(1.5,0);
\end{tikzpicture}
\end{center}
Lorsque $OO' < r - r'$ les cercles ne se chevauchent plus mais le grand contient le petit

On construit donc notre classe en faisant bien attention à ce que ces deux cas s'excluent :

\pythonfile{scripts/Cercle.py}
\end{document}
