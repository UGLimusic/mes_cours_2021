\documentclass[a4paper,10pt,french]{book}
\usepackage[margin=2cm]{geometry}
\usepackage[thinfonts]{uglix2}
\setminted{fontsize=\small}\nouveaustyle
\begin{document}
\titre{DS 02 (suite et fin)}{NSI2}{11/2021}

\exo{ - Analyse de code}

Tu viens d'être nommé développeur dans une entreprise qui gère les systèmes d'alarmes pour les maisons des particuliers.
Ton prédécesseur a commencé à écrire une classe \pythoninline{Alarme}, qui doit normalement répondre aux exigences suivantes :

\begin{enumerate}[--]
	\item 	l'alarme peut être activée et désactivée;
	\item 	chaque intrusion détectée doit être systématiquement consignée dans un journal;
    \item   en cas d'intrusion, si l'alarme est activée, un SMS doit être envoyé au centre de télésurveillance.
\end{enumerate}

Voici son code:

\begin{encadrecolore}{code Python}{python@color}
\begin{minted}[linenos]{python}
from utils import date, envoie_sms

ACTIVE = True
INACTIVE = False


class Alarme:
    def __init__(self, lieu, telephone, etat):
        self.lieu = lieu
        self.telephone = telephone
        self.active = etat
        self.journal = []

    def intrusion(self):
        evenement = date() + " Intrusion"
        if self.active:
            sms = self.lieu + ' : ' + evenement
            envoie_sms(self.telephone, sms)
            evenement = evenement + " envoi sms au " + self.telephone
            self.journal.append(evenement)

    def activer(self):
        self.active = ACTIVE
        evenement = date() + " Activation "
        self.journal.append(evenement)

    def desactiver(self):
        self.active = INACTIVE
        evenement = date() + " Désactivation "
        self.journal.append(evenement)
\end{minted}
\end{encadrecolore}

Le module \pythoninline{utils} permet d'importer les fonctions suivantes :\\

La fonction date \pythoninline{date}
\begin{enumerate}[\textbullet]
	\item 	ne prend rien en entrée;
	\item 	renvoie la date et l'heure sous la forme d'un \pythoninline{str}.
\end{enumerate}

La fonction \pythoninline{envoie_sms}
\begin{enumerate}[\textbullet]
	\item 	prend en entrée \pythoninline{numero} et \pythoninline{sms}, qui sont deux \pythoninline{str}, et envoie le sms au numéro du centre de télésurveillance.
	\item 	renvoie \pythoninline{True} si l'envoi est réussi, \pythoninline{False} sinon.\\
\end{enumerate}
\medskip
Pour ce devoir, on va considérer que la fonction \pythoninline{date} renvoie
\begin{enumerate}[\textbullet]
	\item 	\pythoninline{'0000'} au premier appel;
	\item 	\pythoninline{'0001'} au deuxième appel;
    \item   et ainsi de suite.\\
\end{enumerate}
\medskip

\textbf{1.} On exécute le script suivant :

\begin{pythoncode}
from alarme import Alarme

alarme1 = Alarme("Loritz", "971971971", False)
alarme2 = Alarme("Poincaré", "971971971", True)
alarme1.intrusion()
alarme1.activer()
alarme1.intrusion()
alarme1.desactiver()
alarme2.intrusion()
\end{pythoncode}

Donner les contenus des SMS envoyés.\\

\medskip

\textbf{2.} Que contient \pythoninline{alarme1.journal} à la fin du script ?\\

\medskip

\textbf{3.} En testant le code, on constate que lorsque l'alarme est désactivée les intrusions ne sont pas enregistrées dans le journal.\\
D'où vient l'erreur ? Proposer une correction.\\

\medskip

\textbf{4.} On constate que le journal de bord consigne parfois des envois de SMS que le centre n'a jamais reçus : le système a bien tenté de les envoyé mais l'envoi a échoué.\\
Proposer une correction du code telle que si l'envoi d'un SMS échoue, cela soit consigné dans le journal.\\

\medskip

\textbf{5.} \'Ecrire une méthode d'instance \pythoninline{efface_journal} qui efface le journal de l'instance.\\

\newpage

\exo{ - Cercles}\\ 

On considère les 3 cercles suivants :

\begin{center}
\def\uscl{.5}
\begin{tikzpicture}[scale=\uscl]
\reperevl{-4}{-3}{8}{9}
\draw[blue] (0,0)\ball node[below=1*\uscl cm,right=.8*\uscl cm]{$\mathcal{C}_1$} circle (1);
\draw[red] (2,3)\ball node[above=5*\uscl cm,right=2*\uscl cm]{$\mathcal{C}_2$} circle (5);
\draw[green] (-1,4)\ball node[above=2*\uscl cm,right=1*\uscl cm]{$\mathcal{C}_3$} circle (2);
\end{tikzpicture}
\end{center}

\'Ecrire la classe \pythoninline{Cercle} pour qu'on puisse s'en servir de la sorte :

\begin{pythoncode}
from Cercle import *

c1 = Cercle(0, 0, 1)  # crée un cercle c1 de centre (0,0) et de rayon 1
c2 = Cercle(3, 1, 5)  # idem
c3 = Cercle(-1, 4, 1)  # idem

c2.decale(-1, 2)  # décale le centre ce c2 de (-1,2)
c3.dilate(2)  # multiplie le rayon de c3 par 2

print(c1)  # affiche correctement c1
print(c2)  # idem
print(c3)  # idem

print(c2.contient(c1))  # va afficher True car c1 est « à l'intérieur de c2 »
print(c2.contient(c3))  # False
print(c1.chevauche(c3))  # False car c1 et c2 n'ont aucun point commun
print(c2.chevauche(c3))  # True
\end{pythoncode}

\textbf{Résultat :}
\begin{verbatim}
Cercle : centre(0, 0) rayon 1.
Cercle : centre(2, 3) rayon 5.
Cercle : centre(-1, 4) rayon 2.
True
False
False
True
\end{verbatim}

On suppose que tu as à ta disposition la fonction \pythoninline{distance} qui
\begin{enumerate}[\textbullet]
	\item 	prend en entrée 4 \pythoninline{float}  \pythoninline{xa}, \pythoninline{ya}, \pythoninline{xb} et \pythoninline{yb} qui sont les coordonnées de points A et B;
	\item 	renvoie la distance AB. 	
\end{enumerate}
\end{document}
