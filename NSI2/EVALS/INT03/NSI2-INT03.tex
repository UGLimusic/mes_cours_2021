\documentclass[a4paper,12pt]{article}
\usepackage[margin=2cm]{geometry}		
\usepackage[thinfonts]{uglix2}
\nouveaustyle
\begin{document}
\titreinterro{Interrogation 03}{NSI2}{10/2021}



Calculer \pythoninline{(3 & 5) << 4}\\

\carreauxseyes{16.8}{6.4}\\


Calculer \pythoninline{(11 << 3) |(128 >> 2)}\\

\carreauxseyes{16.8}{6.4}\\

\newpage

On voudrait coder une fonction \pythoninline{h(lst1 : list, lst2 : list) -> bool} qui vérifie si tous les éléments de 1\ere liste sont dans la 2\eme.\\

Proposer quelques tests en pensant aux cas particuliers.\\

\carreauxseyes{16.8}{6.4}\\



On voudrait coder une fonction \pythoninline{g(lst1 : list) -> bool} qui vérifie s'il existe un élément de la liste qui est la somme de deux autres éléments distincts de cette liste.\\
Proposer quelques tests en pensant aux cas particuliers.\\

\carreauxseyes{16.8}{6.4}\\

\newpage 
On voudrait coder une fonction \pythoninline{f(lst1 : list, lst2 : list) -> list} qui renvoie la liste de longueur \pythoninline{max(len(lst1), len(lst2))} dont l'élément d'indice i est
\begin{enumerate}[--]
	\item 	la somme des éléments d'indice i des deux listes si ceux-ci existent.
	\item 	l'élément d'indice i d'une des deux listes si l'autre liste est plus courte.
\end{enumerate}

Proposer quelques tests en pensant aux cas particuliers.\\

\carreauxseyes{16.8}{6.4}
\end{document}
