\documentclass[a4paper,11pt]{article}
\usepackage[margin=2cm]{geometry}
\usepackage{uglix2}
\pagestyle{empty}
\usepackage{numprint}

\begin{document}
\titreinterro{Interrogation 01}{NSI2}{09/2021}


\textbf{1. }Explique \textit{brièvement} (la réponse doit rentrer dans le cadre) et sans donner d'exemple ce qu'est une fonction récursive.\\

% une fonction récursive peut renvoyer un résultat dans un cas d'arrête, sinon elle s'appelle elle-même une ou plusieurs fois avec un de ses paramètres qui devient plus petit. Une pile d'appels est utilisée
\carreauxseyes{16.8}{6.4}\\

\textbf{2. }\'Ecris ici une version \textsc{Python} de la fonction récursive \pythoninline{fibo} qui
\begin{enumerate}[--]
	\item 	en entrée prend un entier positif \pythoninline{n};
	\item 	en sortie renvoie la valeur de $F_n$ définie par $$F_n=\begin{cases}
		1 & \mbox{si } n=0\mbox{ ou }n=1\\
		F_{n-1}+F_{n-2} &\mbox{sinon}
	\end{cases}$$
\end{enumerate}

\carreauxseyes{16.8}{6.4}\\
\newpage
\textbf{3. }Voici une première fonction récursive :
\pythonfile{scripts/mystere1.py}
Calcule \pythoninline{mystery1(0)}, \pythoninline{mystery1(1)} et \pythoninline{mystery1(2)}.\\

\carreauxseyes{16.8}{11.2}\\

Explique ce que fait cette fonction.\\

\carreauxseyes{16.8}{4}\\
\newpage
\textbf{4.} En voici une deuxième :
\pythonfile{scripts/mystere2.py}
Explique ce que renvoient \pythoninline{mystery2([])} et \pythoninline{mystery2([7, 3, 5])}.\\

\carreauxseyes{16.8}{11.2}\\

Explique ce que fait cette fonction.\\

\carreauxseyes{16.8}{3.2}\\




\newpage
\textbf{5.} En voici une dernière :
\pythonfile{scripts/mystere3.py}
Explique ce que renvoient \pythoninline{mystery3([8])}, \pythoninline{mystery3([1,2])}  et \pythoninline{mystery3([4,1,2])}.\\

\carreauxseyes{16.8}{9.6}\\

Explique ce que fait cette fonction.\\

\carreauxseyes{16.8}{2.4}\\



\end{document}
