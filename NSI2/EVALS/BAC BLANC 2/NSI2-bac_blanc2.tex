\documentclass[a4paper,12pt,french]{book}
\usepackage[margin=2cm]{geometry}
\usepackage[thinfonts]{uglix2}
\usepackage{ulem}
\nouveaustyle

\begin{document}
    \titre{NUM\'ERIQUE ET SCIENCES INFORMATIQUES}{}{}

    \begin{center}
        \Large
        \vspace{6cm}
        \textit{Durée de l'épreuve : 2 heures}\\[2em]

        \textit{L'usage de la calculatrice n'est pas autorisé}\\[2em]


        \textit{Le candidat traite au choix 3 exercices parmi les 4 exercices proposés}\\[2em]

        \textit{Chaque exercice devra être traité sur une copie séparée}
    \end{center}
    \newpage

    \section*{Exercice 1 \small{\hfill Bases de données}}

    Dans cet exercice, on pourra utiliser les mots clés suivants du langage \textsc{SQL} :\\

    \mintinline{SQL}{SELECT, FROM, WHERE, JOIN, ON, INSERT INTO, VALUES}

    \mintinline{SQL}{MIN, MAX, OR, AND}\\

    Les fonctions d’agrégation \mintinline{SQL}{MIN(propriete)} et \mintinline{SQL}{MAX(propriete)} renvoient respectivement
    la plus petite et la plus grande valeur de l’attribut \texttt{propriete} pour les enregistrements
    sélectionnés.\\

    Des acteurs ayant joué dans différentes pièces de théâtre sont recensés dans une base de
    données Theatre dont le schéma relationnel est donné ci-dessous :

    \begin{enumerate}[--]
        \item 	\texttt{\textbf{Piece}(\uline{idPiece}, titre, langue)}
        \item 	\texttt{\textbf{Acteur}(\uline{idActeur}, nom, prenom, anneeNaiss)}
        \item	\texttt{\textbf{Role}(\uline{\dashuline{idPiece, idActeur}, nomRole})}
    \end{enumerate}
    Dans ce schéma, les clés primaires sont soulignées en trait plein et les clés étrangères en pointillés.

    L’attribut \texttt{idPiece} de la relation \texttt{Role} est une clé étrangère faisant référence à l’attribut
    \texttt{idPiece} de la relation \texttt{Piece}.\\
    L’attribut \texttt{idActeur} de la relation \texttt{Role} est une clé étrangère faisant référence à l’attribut
    \texttt{idActeur} de la relation \texttt{Acteur}.\\
    Tous les attributs dont le nom est préfixé par \texttt{id} sont des nombres entiers ainsi que l'attribut
    \texttt{anneeNaiss}. Les autres attributs sont des chaînes de caractères.

    \begin{enumerate}[\bfseries 1.]
        \item 	Expliquer pourquoi il n'est pas possible d'insérer une entrée dans la relation \texttt{Role} si les
        relations \texttt{Piece} et \texttt{Acteur} sont vides.
        \item	Dans la pièce « Le Tartuffe », l'acteur Micha Lescot a joué le rôle de Tartuffe.\\
        L’identifiant de Micha Lescot est 389761 et celui de cette pièce est 46721.\\
        Écrire une requête SQL permettant d’ajouter ce rôle dans la table (ou relation) \texttt{Role}.
        \item Expliquer ce que fait la requête SQL suivante.
        \begin{sql}
UPDATE Piece
SET langue = "Anglais"
WHERE langue = "Américain" OR langue = "Britannique";
        \end{sql}
        \item Pour chacun des quatre items suivants, écrire une requête SQL permettant d’extraire les
        informations demandées.
        \begin{enumerate}[\bfseries a.]
            \item 	Le nom et prénom des artistes nés après 1990.
            \item 	L’année de naissance du plus jeune artiste.
            \item 	Le nom des rôles joués par l'acteur Vincent Macaigne.
            \item 	Le titre des pièces écrites en Russe dans lesquelles l’actrice Jeanne Balibar a joué.
        \end{enumerate}
    \end{enumerate}

    \section*{Exercice 2 \small{\hfill Piles et POO}}
    On crée une classe \texttt{Pile} qui modélise la structure d'une pile d'entiers.\\
    Le constructeur de la classe initialise une pile vide.\\
    La définition de cette classe sans l’implémentation de ses méthodes est donnée ci-dessous.
    \begin{pythoncode}
class Pile:
    def __init__(self):
        """Initialise la pile comme une pile vide."""

    def est_vide(self):
        """Renvoie True si la liste est vide, False sinon."""

    def empiler(self, e):
        """Ajoute l'élément e sur le sommet de la pile,
        ne renvoie rien."""

    def depiler(self):
        """Retire l’élément au sommet de la pile et le renvoie."""

    def nb_elements(self):
        """Renvoie le nombre d'éléments de la pile. """

    def afficher(self):
        """Affiche de gauche à droite les éléments de la pile, du fond
        de la pile vers son sommet. Le sommet est alors l’élément
        affiché le plus à droite. Les éléments sont séparés par une
        virgule. Si la pile est vide la méthode affiche « pile
        vide »."""
    \end{pythoncode}
    Seules les méthodes de la classe ci-dessus doivent être utilisées pour manipuler les objets de la classe
    \texttt{Pile}.

    \begin{enumerate}[\bfseries 1.]
        \item
        \begin{enumerate}[\bfseries a.]
            \item 	Écrire une suite d’instructions permettant de créer une instance de la classe \texttt{Pile}
            affectée à une variable \texttt{pile1} contenant les éléments 7, 5 et 2 insérés dans cet
            ordre.\\
            Ainsi, à l’issue de ces instructions, l’instruction \pythoninline{pile1.afficher()} produit
            l’affichage : 7, 5, 2.
            \item 	Donner l’affichage produit après l’exécution des instructions suivantes.
            \begin{minted}{python}
element1 = pile1.depiler()
pile1.empiler(5)
pile1.empiler(element1)
pile1.afficher()
            \end{minted}
        \end{enumerate}
        \item On donne la fonction \texttt{mystere} suivante :
        \begin{pythoncode}
def mystere(pile, element):
    pile2 = Pile()
    nb_elements = pile.nb_elements()
    for i in range(nb_elements):
        elem = pile.depiler()
        pile2.empiler(elem)
        if elem == element:
            return pile2
    return pile2
        \end{pythoncode}

        \begin{enumerate}[\bfseries a.]
        	\item 	Dans chacun des quatre cas suivants, quel est l’affichage obtenu dans la
            console ?
            \begin{enumerate}[\textbullet]
            	\item cas n°1
\begin{minted}{python}
>>>pile.afficher()
7, 5, 2, 3
>>>mystere(pile, 2).afficher()
\end{minted}
\ \\
                \item cas n°2
\begin{minted}{python}
>>>pile.afficher()
7, 5, 2, 3
>>>mystere(pile, 9).afficher()
\end{minted}
\ \\
            	\item 	cas n°3
\begin{minted}{python}
>>>pile.afficher()
7, 5, 2, 3
>>>mystere(pile, 3).afficher()
\end{minted}
\ \\
            	\item 	cas n°4
\begin{minted}{python}
>>>pile.est_vide()
True
>>>mystere(pile, 3).afficher()
\end{minted}
\ \\
            \end{enumerate}
        	\item  Expliquer ce que permet d’obtenir la fonction \texttt{mystere}.
        \end{enumerate}
    \item Écrire une fonction \pythoninline{etendre(pile1, pile2)} qui prend en arguments deux objets
    \texttt{Pile} appelés \pythoninline{pile1} et \pythoninline{pile2} et qui modifie \pythoninline{pile1} en lui ajoutant les éléments de
    \pythoninline{pile2} rangés dans l'ordre inverse.\\
    Cette fonction ne renvoie rien.\\
    On donne ci-dessous les résultats attendus pour certaines instructions.
\begin{minted}{python}
>>>pile1.afficher()
7, 5, 2, 3
>>>pile2.afficher()
1, 3, 4
>>>etendre(pile1, pile2)
>>>pile1.afficher()
7, 5, 2, 3, 4, 3, 1
>>>pile2.est_vide()
True
\end{minted}

\item Écrire une fonction \pythoninline{supprime_toutes_occurences(pile, element)} qui prend en
arguments un objet \pythoninline{Pile} appelé \pythoninline{pile} et un élément \pythoninline{element} et supprime tous les
éléments \pythoninline{element} de \pythoninline{pile}.\\
On donne ci-dessous les résultats attendus pour certaines instructions.
\begin{minted}{python}
>>>pile.afficher()
7, 5, 2, 3, 5
>>>supprime_toutes_occurences (pile, 5)
>>>pile.afficher()
7, 2, 3
\end{minted}
\end{enumerate}

\section*{Exercice 3 \small{\hfill Diviser pour régner}}

Dans un tableau \textsc{Python} d'entiers \pythoninline{tab}, on dit que le couple d’indices \pythoninline{(i, j)} forme une inversion
lorsque \pythoninline{i < j} et \pythoninline{tab[i] > tab[j]}. On donne ci-dessous quelques exemples.
\begin{enumerate}[--]
	\item 	Dans le tableau \pythoninline{[1, 5, 3, 7]}, le couple d’indices \pythoninline{(1, 2)} forme une inversion car \pythoninline{5 > 3}.\\
    Par contre, le couple \pythoninline{(1, 3)} ne forme pas d'inversion car \pythoninline{5 < 7}.\\
    Il n’y a qu’une inversion dans ce tableau.
	\item  Il y a trois inversions dans le tableau \pythoninline{[1, 6, 2, 7, 3]}, à savoir les couples d'indices
    \pythoninline{(1, 2)}, \pythoninline{(1, 4)} et \pythoninline{(3, 4)}.
    \item On peut compter six inversions dans le tableau \pythoninline{[7, 6, 5, 3]} : les couples d'indices
    \pythoninline{(0, 1)}, \pythoninline{(0, 2)}, \pythoninline{(0, 3)}, \pythoninline{(1, 2)}, \pythoninline{(1, 3)} et \pythoninline{(2, 3)}.
\end{enumerate}
On se propose dans cet exercice de déterminer le nombre d’inversions dans un tableau
quelconque.

\subsection*{Questions préliminaires}

\begin{enumerate}[\bfseries 1.]
	\item 	Expliquer pourquoi le couple \pythoninline{(1, 3)} est une inversion dans le tableau \pythoninline{[4, 8, 3, 7]}.
	\item 	Justifier que le couple \pythoninline{(2, 3)} n’en est pas une.
\end{enumerate}
\subsection*{Partie A : Méthode itérative}

Le but de cette partie est d’écrire une fonction itérative \pythoninline{nombre_inversions} qui renvoie le
nombre d’inversions dans un tableau.\\
Pour cela, on commence par écrire une fonction \pythoninline{fonction1} qui sera ensuite utilisée pour écrire la fonction \pythoninline{nombre_inversions}.
\begin{enumerate}[\bfseries 1.]
	\item 	On donne la fonction suivante.
    \begin{pythoncode}
def fonction1(tab, i):
    nb_elem = len(tab)
    cpt = 0
    for j in range(i+1, nb_elem):
        if tab[j] < tab[i]:
        cpt += 1
    return cpt
\end{pythoncode}
    \begin{enumerate}[\bfseries a.]
    	\item 	 Indiquer ce que renvoie \pythoninline{fonction1(tab, i)} dans les cas suivants.
        \begin{enumerate}[\textbullet]
        	\item 	  cas n°1 : \pythoninline{tab = [1, 5, 3, 7]} et \pythoninline{i = 0}.
        	\item 	 cas n°2 : \pythoninline{tab = [1, 5, 3, 7]} et \pythoninline{i = 1}.
            \item    cas n°3 : \pythoninline{tab = [1, 5, 2, 6, 4]} et \pythoninline{i = 1}.
        \end{enumerate}

    	\item Expliquer ce que permet de déterminer cette fonction.
    \end{enumerate}
	\item  En utilisant la fonction précédente, écrire une fonction \pythoninline{nombre_inversions(tab)} qui
    prend en argument un tableau et renvoie le nombre d’inversions dans ce tableau.\\
    On donne ci-dessous les résultats attendus pour certains appels.
\begin{minted}{python}
>>> nombre_inversions([1, 5, 7])
0
>>> nombre_inversions([1, 6, 2, 7, 3])
3
>>> nombre_inversions([7, 6, 5, 3])
6
\end{minted}

\item  Quel est l’ordre de grandeur de la complexité en temps de l'algorithme obtenu ?\\
    Aucune justification n'est attendue.
\end{enumerate}
\subsection*{Partie B : Méthode récursive}

Le but de cette partie est de concevoir une version récursive de la fonction de la partie précédente.
On définit pour cela des fonctions auxiliaires.

\begin{enumerate}[\bfseries 1.]
	\item 	Donner le nom d’un algorithme de tri ayant une complexité meilleure que quadratique.\\

Dans la suite de cet exercice, on suppose qu’on dispose d'une fonction \pythoninline{tri(tab)} qui prend en
argument un tableau et renvoie un tableau contenant les mêmes éléments rangés dans l'ordre
croissant.

	\item Écrire une fonction \pythoninline{moitie_gauche(tab)} qui prend en argument un tableau \pythoninline{tab} et
    renvoie un nouveau tableau contenant la moitié gauche de \pythoninline{tab}. Si le nombre d'éléments
    de \pythoninline{tab} est impair, l'élément du centre se trouve dans cette partie gauche.
    On donne ci-dessous les résultats attendus pour certains appels.
\begin{minted}{python}
>>> moitie_gauche([])
[]
>>> moitie_gauche([4, 8, 3])
[4, 8]
>>> moitie_gauche ([4, 8, 3, 7])
[4, 8]
\end{minted}
\ \\
Dans la suite, on suppose qu’on dispose de la fonction \pythoninline{moitie_droite(tab)} qui renvoie la
moitié droite sans l’élément du milieu

\item On suppose qu’une fonction \pythoninline{nb_inv_tab(tab1, tab2)} a été écrite.\\
Cette fonction renvoie le nombre d’inversions du tableau obtenu en mettant bout à bout les tableaux
\pythoninline{tab1} et \pythoninline{tab2}, à condition que \pythoninline{tab1} et \pythoninline{tab2} soient triés dans l’ordre croissant.\\
On donne ci-dessous deux exemples d’appel de cette fonction :
\begin{minted}{python}
>>> nb_inv_tab([3, 7, 9], [2, 10])
3
>>> nb_inv_tab([7, 9, 13], [7, 10, 14])
3
\end{minted}

En utilisant la fonction \pythoninline{nb_inv_tab} et les questions précédentes, écrire une fonction
récursive \pythoninline{nb_inversions_rec(tab)} qui permet de calculer le nombre d'inversions
dans un tableau.\\
Cette fonction renverra le même nombre que \pythoninline{nombre_inversions(tab)} de la partie A.\\
On procédera de la façon suivante :
\begin{enumerate}[--]
	\item 	Séparer le tableau en deux tableaux de tailles égales (à une unité près).
	\item 	Appeler récursivement la fonction \pythoninline{nb_inversions_rec} pour compter le nombre
    d’inversions dans chacun des deux tableaux.
    \item   Trier les deux tableaux (on rappelle qu'une fonction de tri est déjà définie).
    \item   Ajouter au nombre d'inversions précédemment comptées le nombre renvoyé par la
    fonction \pythoninline{nb_inv_tab} avec pour arguments les deux tableaux triés.
\end{enumerate}
\end{enumerate}

\section*{Exercice 4 \small{\hfill Gestion des processus}}

\begin{enumerate}[\bfseries 1.]
	\item 	Les états possibles d’un processus sont : \textit{prêt}, \textit{élu}, \textit{terminé} et \textit{bloqué}.
    \begin{enumerate}[\bfseries a.]
    	\item 	Expliquer à quoi correspond l’état élu.
    	\item 	Proposer un schéma illustrant les passages entre les différents états.
    \end{enumerate}
	\item 	 	On suppose que quatre processus C₁, C₂, C₃ et C₄ sont créés sur un ordinateur,
    et qu’aucun autre processus n’est lancé sur celui-ci, ni préalablement ni pendant
    l’exécution des quatre processus.\\

    L’ordonnanceur, pour exécuter les différents processus prêts, les place dans une
    structure de données de type file. Un processus prêt est enfilé et un processus
    élu est défilé.

    \begin{enumerate}[\bfseries a.]
    	\item 	Parmi les propositions suivantes, recopier celle qui décrit le fonctionnement
        des entrées/sorties dans une file :
        \begin{enumerate}[--]
        	\item   Premier entré, dernier sorti
        	\item 	Premier entré, premier sorti
            \item   Dernier entré, premier sorti
        \end{enumerate}
    \item  On suppose que les quatre processus arrivent dans la file et y sont placés
    dans l’ordre C₁, C₂, C₃ et C₄.
    \begin{enumerate}[--]
    	\item 	Les temps d’exécution totaux de C₁, C₂, C₃ et C₄ sont respectivement
            100 ms, 150 ms, 80 ms et 60 ms.
    	\item 	Après 40 ms d’exécution, le processus C₁ demande une opération d’écriture
            disque, opération qui dure 200 ms. Pendant cette opération d’écriture, le
            processus C₁ passe à l’état bloqué.
        \item   Après 20 ms d’exécution, le processus C₃ demande une opération d’écriture
            disque, opération qui dure 10 ms. Pendant cette opération d’écriture, le
            processus C₃ passe à l’état bloqué.
    \end{enumerate}
    Sur la frise chronologique donnée en annexe (à rendre avec la copie), les
    états du processus C₂ sont donnés. Compléter la frise avec les états des
    processus C1, C3 et C4.
    \end{enumerate}
    \item On trouvera ci- dessous deux programmes rédigés en pseudo-code.\\

    Verrouiller un fichier signifie que le programme demande un accès exclusif au
    fichier et l’obtient si le fichier est disponible.

    \begin{multicols}{2}
    \textbf{Programme 1}\\

    Verrouiller fichier1\\
    Calculs sur fichier1\\
    Verrouiller fichier2\\
    Calculs sur fichier1\\
    Calculs sur fichier2\\
    Calculs sur fichier1\\
    Déverrouiller fichier2\\
    Déverrouiller fichier1\\
    \columnbreak

    \textbf{Programme 2}\\

    Verrouiller fichier2\\
    Verrouiller fichier1\\
    Calculs sur fichier1\\
    Calculs sur fichier2\\
    Déverrouiller fichier1\\
    Déverrouiller fichier2
    \end{multicols}
    \begin{enumerate}[\bfseries a.]
    	\item 	En supposant que les processus correspondant à ces programmes s’exécutent
        simultanément (exécution concurrente), expliquer le problème qui peut être
        rencontré.
    	\item 	Proposer une modification du programme 2 permettant d’éviter ce problème.
    \end{enumerate}
\end{enumerate}
\end{document}