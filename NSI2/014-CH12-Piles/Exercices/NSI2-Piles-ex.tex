\documentclass[a4paper,12pt,french]{article}
\usepackage[margin=2cm]{geometry}
\usepackage[thinfonts]{uglix2}
\nouveaustyle
  
\begin{document}
\titre{Piles - Exercices}{NSI2}{2021} 

\begin{exercice}[ : Implémentation objet]
	\'Ecrire une classe \pythoninline{Stack} qui implémente la structure de pile vue en cours. Sauvegarder dans un module \texttt{Stack.py}. Celui-ci servira pour les exercices suivants.
\end{exercice}

\begin{exercice}[ : Les piles c'est renversant]
\'Ecrire une fonction \pythoninline{reverse_with_stack} qui utilise une pile et
\begin{enumerate}[--]
	\item 	en entrée prend un \pythoninline{str};
	\item 	en sortie renvoie un \pythoninline{str} qui est la chaîne passée en paramètre, mais dans l'autre sens.	
\end{enumerate}
\end{exercice}

\begin{exercice}[ : Expressions bien parenthésées]
L'expression $(a+b\times(c+(a+b)^2))^3$ est bien parenthésée. $(a-b)\times c)$ ne l'est pas.
\begin{enumerate}[\bfseries 1.]
	\item 	\'Ecrire une fonction \pythoninline{is_valid} qui utilise une pile et
	\begin{enumerate}[--]
		\item 	en entrée prend in \pythoninline{str} qui est l'expression à tester.
		\item 	en sortie renvoie un \pythoninline{bool} : \pythoninline{True} si elle est bien parenthésée, \pythoninline{False} sinon.
	\end{enumerate}
	\item 	Proposer un ensemble de tests unitaires.	
\end{enumerate}
\end{exercice}

\begin{exercice}[ : \'Egalité de deux piles]
	\begin{enumerate}[\bfseries 1.]
		\item 	\'Ecrire une fonction \pythoninline{are_equal} qui
		\begin{enumerate}[--]
			\item 	en entrée prend 2 piles.
			\item 	en sortie renvoie un \pythoninline{bool} : \pythoninline{True} si les deux piles ont les mêmes éléments (même nombre et même ordre), \pythoninline{False} sinon.
		\end{enumerate}
		La fonction doit remettre les piles telles quelles.
	\item 	Proposer un ensemble de tests unitaires.	
	\end{enumerate}

\end{exercice}

\begin{exercice}[ : Notation polonaise inversée]
La Notation Polonaise Inverse (NPI), ou notation post-fixée, est une marnière d’écrire les expressions mathématiques en se passant des parenthèses. Elle a été introduite par le mathématicien polonais Jan Lucasievicz dans les années 1920.\\
Le principe de cette méthode est de placer chaque opérateur juste après ses deux opérandes.\\
Par souci de simplicité nous ne considèrerons que des expressions mettant en jeu des entiers naturels.\\

L’expression $2 + 3$ devient en NPI  2 3 +.
\begin{itemize}
	\item $2 + 6 − 1 $ s’écrit  2\ 6 + 1 −
	\item $5 \times 3 + 4$ s’écrit  5\ 3 * 4 +
	\item $((1 + 2) \times 4) + 3$ s’écrit 1 2 + 4 * 3 +
\end{itemize}

\'Evaluer une expression post-fixée est facile. Pour cela il suffit de lire l’expression de gauche à
droite et d’appliquer chaque opérateur aux deux opérandes qui le précèdent. Si l’opérateur n’est
pas le dernier symbole on replace le résultat intermédiaire dans l’expression et on recommence avec
l’opérateur suivant.\\

On peut écrire un programme en Python qui utilise une pile s pour évaluer les expressions en NPI en suivant cet exemple
\begin{enumerate}[--]
	\item 	on considère une chaine de caractères : \pythoninline{c = '23 6 + 1 -'};
	\item 	on la transforme en une liste : \pythoninline{lst = c.split()};
	\item 	la valeur de \pythoninline{lst} est alors \pythoninline{['23', '6', '+', '1', '-']};
	\item 	on constate que \pythoninline{'23'} et \pythoninline{'6'} ne sont pas des symboles d'opérateurs donc sont convertibles en \pythoninline{int} et on empile ces deux entiers dans s;
 	\item 	le prochaine élément de \pythoninline{lst} est \pythoninline{'+'} donc on dépile les deux entiers 6 et 23, on les ajoute, et on empile le résultat 29;
 	\item	on continue : on empile 1;
 	\item	on arrive à \pythoninline{'-'}, on dépile les 2 entiers  1 et 29 et on les soustrait (dans le bon ordre) et on empile le résultat final 28.
\end{enumerate}

\'Ecrire une fonction \pythoninline{mpi_compute} qui
\begin{enumerate}[--]
	\item 	en entrée prend un \pythoninline{str} qui est une expression NPI;
	\item 	en sortie renvoie un \pythoninline{float} qui est l'évaluation de cette expression (\pythoninline{float} car l'expression peut comporter des divisions).	
\end{enumerate}
\end{exercice}
\end{document}