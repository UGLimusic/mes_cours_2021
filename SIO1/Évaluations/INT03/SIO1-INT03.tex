\documentclass[a4paper,10pt]{article}
\usepackage[margin=2cm,top=1cm,bottom=1cm]{geometry}
\usepackage[thinfonts]{uglix2}
\setlength{\columnseprule}{0.5pt}
\pagestyle{empty}
\begin{document}
\titreinterro{Interrogation 03}{SIO1}{10/2021}

Qu'affiche chacun des scripts suivants ?\\
\begin{multicols}{2}
\begin{pythoncode}
login = 'treizelettres'
print(login[:2])
\end{pythoncode}
\begin{enumerate}[\case\ \ a.]
\item \pythoninline{'es'}
\item \pythoninline{'eizelettres'}
\item \pythoninline{'tr'}
\end{enumerate}

\begin{pythoncode}
a = 5
b = 7
print(a > 7 or b < 2)
\end{pythoncode}
\begin{enumerate}[\case\ \ a.]
\item 	\pythoninline{True}
\item 	\pythoninline{False}
\item 	Il génère une erreur
\end{enumerate}
\begin{pythoncode}
a = 2
print(a == 2 and a != 2)
\end{pythoncode}
\begin{enumerate}[\case\ \ a.]
\item 	\pythoninline{True}
\item 	\pythoninline{False}
\item 	Il génère une erreur
\end{enumerate}
\columnbreak
\begin{pythoncode}
a = 2
b = 3
if a > 3:
    b = b - 7
if b > 0:
    b = b + 1
print(b)
\end{pythoncode}
\begin{enumerate}[\case\ \ a.]
\item 	\pythoninline{3}
\item 	\pythoninline{4}
\item 	\pythoninline{-3}
\end{enumerate}

\begin{pythoncode}
a = 5
b = 8
if a == 5 or b == 6:
    b = b - 2
b = b - 1
print(b)
\end{pythoncode}

\begin{enumerate}[\case\ \ a.]
\item 	\pythoninline{8}
\item 	5
\item 	rien
\end{enumerate}
\begin{pythoncode}
a = 2
b = 12
if a == 2 and b == 10:
    b = b + 2
    a = a + 1
    print(a)
\end{pythoncode}
\begin{enumerate}[\case\ \ a.]
\item 	\pythoninline{2}
\item 	\pythoninline{3}
\item 	Rien
\end{enumerate}
\newpage
\begin{pythoncode}
prenom = 'Marguerite'
if prenom[0] == ' K':
    corr = 0
elif prenom[1] == 'a':
    corr = 1
else:
    corr = -1
print(corr)
\end{pythoncode}
\begin{enumerate}[\case\ \ a.]
\item 	\pythoninline{0}
\item 	\pythoninline{1}
\item 	\pythoninline{-1}
\end{enumerate}
\begin{pythoncode}
a = 30
b = 10
if a > 40 and b > 5:
    b = b + 10
elif a > 20 and b < 5:
    b = b + 20
elif a > 10 and b > 2:
    b = b + 30
print(b)
\end{pythoncode}
\begin{enumerate}[\case\ \ a.]
\item 	\pythoninline{40}
\item 	\pythoninline{10}
\item 	\pythoninline{30}
\end{enumerate}
\end{multicols}
\ \\

Le mercure gèle à -39 °C et bout à 357 °C.\\
Écris un script qui demande à l'utilisateur de rentrer une température puis
lui indique si le mercure est solide, liquide ou gazeux à cette température.\\

\textbf{Exemple}\\

\texttt{Entrez la température :\\
25\\
À cette température le mercure est liquide\\
}

\carreaux{16.8}{12}

\end{document}

