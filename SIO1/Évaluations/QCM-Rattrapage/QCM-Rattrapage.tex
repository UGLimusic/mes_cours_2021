\documentclass[a4paper,10pt]{article}
\usepackage[margin=2cm,top=1cm]{geometry}		
\usepackage[thinfonts]{uglix2}
\pagestyle{empty}
\nouveaustyle
\begin{document}
	\titreinterro{QCM (rattrapage)}{SIO1}{10/2021}
\begin{center}
\textit{Une seule bonne réponse par question.\\
bonne réponse : +3 points, mauvaise réponse : -1 point, pas de réponse : 0 point.
}\end{center}

\ligne\\

Sur combien d'octets le nombre $(AF7B)_{16}$ est-il codé ?
	\begin{enumerate}[\case\ ]
	\item 2
	\item 4
	\item 8
	\item 64
\end{enumerate}	
\ligne\\

En base 16, le nombre $(0010\,1101\,0101)_2$ s'écrit

	\begin{enumerate}[\case\ ]
	\item $(0010\,1101\,0101)_{16}$
	\item $(5C2)_{16}$
	\item $(2C5)_{16}$
	\item $(2D5)_{16}$
\end{enumerate}	
 \ligne\\
 
\begin{pythoncode}
n = 3
while n < 12:
    n += 2
print(n)
\end{pythoncode}
Ce script affiche
\begin{enumerate}[\case\ ]
	\item 11
	\item 12
	\item 13
	\item rien du tout
\end{enumerate}	
\ligne\\

Laquelle de ces boucles s'arrête ?

\begin{enumerate}[\case\ ]
    \item \begin{minted}{python}
i = 1000
while i > 10:
    i += 1
    print("salut")
\end{minted}
    \item \begin{minted}{python}
i = 0
while i < 10:
    print("salut")  
\end{minted}
    \item \begin{minted}{python}
i = 1000
while i < 10:
    i += 1
    print("salut")
\end{minted}
    \item \begin{minted}{python}
i = -1000
while i < 10:
    i -= 1
    print("salut")
\end{minted}
\end{enumerate}
\ligne\\

\begin{pythoncode}
a = 1
for i in range(8):
    a = a * 2
\end{pythoncode}
Ce script affiche

\begin{enumerate}[\case\ ]
    \item 16
    \item 8
    \item 256
    \item 127
\end{enumerate}	
\ligne\\

\begin{pythoncode}
chaine = "Alors comment ça va ?"
c = 0
for lettre in chaine:
    if lettre == "a":
        c += 1
print(c)
\end{pythoncode}
Que fait ce script ?
\begin{enumerate}[\case\ ]
	\item 	Il affiche 3
	\item 	Il affiche \tw{Aaa}
    \item   Il affiche 2
    \item   Il produit une erreur
\end{enumerate}
\ligne\\

\begin{pythoncode}
chaine = "On m'appelle choupinou"
c = 0
n = len(chaine)
for i in range(len(chaine)):
    if chaine[i] == 'o':
        c = i
print(c)
\end{pythoncode}
Que fait ce script ?
\begin{enumerate}[\case\ ]
	\item 	Il produit une erreur
	\item 	Il affiche \tw{o}
    \item   Il affiche 15
    \item   Il affiche 20
\end{enumerate}
\ligne\\

On a
\begin{minted}{python}
age = 10
prenom = "Titouan"
\end{minted}
Lequel de ces scripts provoque une erreur ?

\begin{enumerate}[\case\ ]
\item \begin{minted}{python}
print("Je m'appelle", prenom, "et j'ai", age, "ans.")
\end{minted}
\item \begin{minted}{python}
print("Je m'appelle " + prenom + " et j'ai ", age, "ans.")
\end{minted}
\item \begin{minted}{python}
print("Je m'appelle " + prenom + " et j'ai " + age + " ans.")
\end{minted}
\item \begin{minted}{python}
print("Je m'appelle", prenom,"et j'ai" + str(age) + " ans.")
\end{minted}
\end{enumerate}
\ligne\\
 

En complément à deux sur 8 bits, l'entier -2 est représenté par
\begin{enumerate}[\case\ ]
	\item 	$(0000\,0010)_2$
	\item 	$(-0000\,0010)_2$
    \item 	$(1000\,0010)_2$
    \item 	$(1111\,1110)_2$
\end{enumerate}
\ligne\\
\newpage 
Qu'affiche ce script ?

\begin{pythoncode}
a = 1
b = 2
a = a + b
b = b - a
a = b - a
print(a, b)
\end{pythoncode}

\begin{enumerate}[\case \ ]
	\item 	2 et 1
	\item 	1 et 2
    \item   0 et 0
    \item   -4 et -1
\end{enumerate}
\ligne\\

\begin{pythoncode}
a = 1
b = a + 1
if a < b and b < 1:
    print("ok")
elif a > b or b < 1:
    print("d'accord")
elif a > b and b < 1:
    print("je vois")
elif a < b and b > 1:
    print("mais oui")
\end{pythoncode}

Qu'affiche ce script ?
\begin{enumerate}[\case \ ]
	\item 	\texttt{ok}
	\item 	\texttt{d'accord}
    \item   \texttt{je vois}
    \item   \texttt{mais oui} 	
\end{enumerate}
\ligne
\end{document}