\documentclass[a4paper,12pt]{article}
\usepackage[margin=2cm]{geometry}
\usepackage[thinfonts,latinmath]{uglix2}

\nouveaustyle
\pagestyle{empty}
\begin{document}
    \titre{Préparation au contrôle}{SIO1}{2022}

\section*{Métropole mai 2015}
Une association sportive souhaite recruter une personne pour animer son site internet et dynamiser
son image. Le candidat recruté devra remplir l'une au moins des quatre conditions suivantes :


\begin{enumerate}[--]
    \item avoir des connaissances en informatique et être sous contrat avec la mairie ;
    \item ne pas avoir de connaissances particulières en informatique, mais être membre de l'association
    et être sous contrat avec la mairie ;
    \item ne pas être membre de l'association mais être sous contrat avec la mairie ;
    \item ne pas être sous contrat avec la mairie, mais être membre de l'association.\\
\end{enumerate}

On définit les trois variables booléennes $a,\: b,$ et $c$ de la manière suivante :


\begin{enumerate}[--]
    \item $a = 1$ si la personne est membre de l'association et $a =  0$ sinon ;
    \item $b = 1$ si la personne a des connaissances en informatique, et $b = 0$ sinon ;
    \item $c = 1$ si la personne est en contrat avec la mairie, et $c = 0$ sinon.\\
\end{enumerate}

\begin{enumerate}[\bfseries 1.]
    \item Écrire une expression booléenne $E$ traduisant globalement les  conditions de recrutement.
    \item À l'aide d'un calcul booléen ou d'un tableau de Karnaugh, simplifier l'expression $E$ sous la
    forme d'une somme de deux termes, puis interpréter cela à l'aide d'une phrase.
    \item Un candidat ayant des connaissances en informatique se présente, mais il est écarté car il ne
    correspond pas eux critères de recrutement.
    Que peut-on en déduire sur le profil de ce candidat ?
\end{enumerate}

\section*{Exercice}

La connexion à un site Internet nécessite la saisie d'un mot de passe comportant de
8 à 12 caractères.\\
 Ces caractères peuvent être des lettres majuscules de l'alphabet
français, ou des chiffres, ou des caractères spéciaux (tels que \&,*,/,§ etc).\\
Un mot de passe est valide si l'une au moins des trois conditions suivantes est
réalisée:
\begin{enumerate}[--]
    \item 	Il comporte au moins trois chiffres et trois caractères spéciaux.
    \item 	Il comporte au moins cinq lettres.
    \item 	Il comporte moins de trois chiffres mais au moins cinq lettres et trois caractères
    spéciaux.
\end{enumerate}
\subsection*{Partie A - ReconnaÎtre si un mot de passe est valide}

\begin{enumerate}[\bfseries 1.]
    \item 	 Parmi les mots de passe suivants, quels sont ceux qui sont valides ?\\

    \texttt{H32EXZ\&K5=}\hspace{6em}  \texttt{LUC230598**}\hspace{6em} \texttt{123(M*K<4}
    \item 	 	Alice veut créer un mot de passe avec quatre lettres, quatre chiffres et quatre
    caractères spéciaux. Ce mot de passe sera-t-il accepté ? Et un mot de passe de huit
    lettres ?
\end{enumerate}

\subsection*{Partie B - Écriture d'une expression booléenne}

On définit trois variables booléennes $a$, $b$ et $c$ de la façon suivante :
\begin{enumerate}[--]
    \item 	$a = 1$ si le mot de passe contient au moins trois chiffres, sinon $a = O$.
    \item 	$b = 1$ si le mot de passe contient au moins cinq lettres, sinon $b = O$.
    \item 	$c = 1$ si le mot de passe contient au moins trois caractères spéciaux, sinon $c = O$.
\end{enumerate}

ainsi que la variable $A$ telle que $A = 1$ si le mot de passe est valide, $A = 0$ sinon.

\begin{enumerate}[\bfseries 1.]
    \item 	Traduire chacune des trois conditions de validité d'un mot de passe à l'aide des
    variables $a$, $b$ et $c$. En déduire l'expression de $A$.
   \item 	Représenter $A$ avec un tableau de Karnaugh. En déduire une expression simplifiée
   de $A$.
   \item Par le calcul, retrouver la forme simplifiée de $A$.
   \item Exprimer par une phrase la plus simple possible la condition pour qu'un mot de passe soit valide.
\end{enumerate}
\subsection*{Partie C - Les mots de passe non valides}

\begin{enumerate}[\bfseries 1.]
    \item 	En utilisant le tableau de Karnaugh, déterminer l'expression de $\barmaj{A}$ .
    \item 	Retrouver le résultat par le calcul.
    \item 	Exprimer par une phrase la condition pour qu'un mot de passe soit refusé.
\end{enumerate}
\end{document}