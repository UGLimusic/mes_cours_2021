\documentclass[a4paper,12pt]{article}
\usepackage[margin=2cm]{geometry}
\usepackage[thinfonts,latinmath]{uglix2}

\nouveaustyle
\pagestyle{empty}
\begin{document}
    \titre{Interrogation 06}{SIO1}{2022}

\section*{D'après métropole mai 2018}

Sur une plateforme de vidéos en ligne, les vidéos sont notées de 0 à 5 par les utilisateurs.

Après une période d'observation, les administrateurs de la plateforme décident de mettre une vidéo
sur la page d'accueil lorsqu'elle satisfait à l'un au moins des critères suivants:
\begin{enumerate}[--]
    \item la vidéo a obtenu la note 5 et comptabilise un nombre de vues supérieur ou égal à 200 ;
    \item la vidéo a obtenu la note 5 et elle est récente;
    \item la vidéo comptabilise un nombre de vues strictement inférieur à 200 et elle est récente;
    \item la vidéo n'a pas obtenu la note 5 et comptabilise un nombre de vues supérieur ou égal à 200.
\end{enumerate}

\smallskip

On définit les trois variables booléennes $a$, $b$, $c$ de la façon suivante:

\setlength\parindent{1cm}
\begin{enumerate}[--]
    \item $a = 1$ si la vidéo a obtenu la note 5, $a = 0$ sinon;
    \item $b = 1$ si la vidéo comptabilise un nombre de vues supérieur ou égal à $200$, $b = 0$ sinon;
    \item $c = 1$ si la vidéo est récente, $c = 0$ sinon.
\end{enumerate}
\setlength\parindent{0cm}

\medskip

\begin{enumerate}[\bfseries 1.]
    \item L'administrateur de la plateforme a traduit les conditions pour qu'une vidéo soit mise sur la page
    d'accueil par l'expression booléenne $E = ab + a c + \overline{b} c + \overline{a} b$.

    Justifier chacun des termes de cette somme.
    \item
    \begin{enumerate}[\bfseries a.]
        \item Représenter l'expression $E$ dans un diagramme de Karnaugh.
        \item En déduire une expression simplifiée de $E$ sous la forme d'une somme de deux termes.
        \item Retrouver cette expression par le calcul.
        \item Interpréter cette expression simplifiée de $E$ dans le contexte de l'exercice.
    \end{enumerate}
    \item  Une vidéo qui n'est pas récente, qui n'a pas obtenu la note 5 et qui comptabilise un nombre de
    vues strictement inférieur à 200 sera-t-elle mise sur la page d'accueil ?
    \item  \begin{enumerate}[\bfseries a.]
        \item 	Donner une expression de $\overline{E}$ à l'aide des variables booléennes précédemment définies en utilisant un diagramme de Karnaugh.
        \item 	Retrouver ce résultat par le calcul.
        \item 	Interpréter cette expression de $\overline{E}$ dans le contexte de l'exercice.
    \end{enumerate}
\end{enumerate}

\section*{D'après métropole mai 2017}

Le but de cet exercice est d'étudier une méthode de cryptage inventée par Gilbert Vernam en 1917, et appelée \og masque jetable \fg.

Dans tout l'exercice, on note respectivement $M$ le mot initial, $K$ la clé de cryptage et $Y$ le mot crypté.

Les trois nombres $M$, $K$, $Y$ sont des entiers naturels.

\medskip

\begin{center}\textbf{Partie 1 : Masque jetable}\end{center}

La méthode décrite dans cette partie utilise le connecteur logique \og \emph{xor} \fg, appelé \og ou exclusif \fg, qui est défini par la table de vérité suivante :

\begin{center}
    \begin{tabularx}{0.5\linewidth}{|*{3}{>{\centering \arraybackslash}X|}}\hline
        $P$ &$Q$ &$P\: xor\: Q$\\ \hline
        0 &0 &0\\ \hline
        0 &1 &1\\ \hline
        1 &0 &1\\ \hline
        1 &1 &0\\ \hline
    \end{tabularx}
\end{center}

Par exemple les deux premières lignes signifient que $0\: xor\: 0 = 0$ et que $0\: xor\: 1 = 1$.

\medskip

\begin{enumerate}[\bfseries 1.]
    \item Recopier intégralement la table de vérité ci-après et compléter la dernière colonne.
    \begin{center}
        \begin{tabularx}{0.7\linewidth}{|*{3}{>{\centering \arraybackslash}X|}c|}\hline
            $P$ &$Q$ &$P\: xor\: Q$&$(P\: xor\: Q) \:xor\: Q$\\ \hline
            0 &0 &0&\\ \hline
            0 &1 &1&\\ \hline
            1 &0 &1&\\ \hline
            1 &1 &0&\\ \hline
        \end{tabularx}
    \end{center}

    \item Parmi les quatre propositions $P$, $Q$, $(P\: xor\: Q)$ et $((P\: xor\: Q)\: xor\: Q)$, deux sont équivalentes.

    À l'aide de la table 2 complétée, déterminer lesquelles, en expliquant la réponse.
\end{enumerate}

Dans la suite de l'exercice, on note $a_b$ l'écriture du nombre entier $a$ en base $b$.

\begin{enumerate}[\bfseries 1.]
    \setcounter{enumi}{2}
    \item Donner la représentation binaire de l'entier qui s'écrit $26_{10}$ en décimal.
    \item Soit $M$ et $K$ deux entiers naturels écrits en binaire, tels que la longueur de l'écriture de $K$ est supérieure ou égale à celle de $M$.

    Pour crypter le mot $M$ avec la clé $K$, on procède comme suit : pour chaque chiffre $m$ du mot
    initial $M$, on considère le chiffre $k$ de la clé $K$ qui a la même position que $m$ dans l'écriture.

    On obtient alors le chiffre $y$ du mot crypté $Y$ qui a la même position que $m$ dans l'écriture du mot initial $M$, par la relation : $y = m \:xor\: k$.

    L'écriture binaire du mot crypté $Y$ est la juxtaposition dans le même ordre des chiffres $y$ calculés pour chaque chiffre $m$ du mot $M$.\\

        \emph{Exemple }:  avec $M = 01_2$ et $K = 10_2$\\
        \begin{enumerate}[--]
            \item Avec le chiffre de rang 1 en partant de la droite : $m = 1$ et $k = 0$
            \item avec le chiffre de rang 2 :  $m = 0$ et $k = 1$ ; donc $y = 0 \:xor\: 1 = 1$.
        \end{enumerate}
        Donc le mot crypté est $Y = 11_2$ \\

     \begin{enumerate}[\bfseries a.]
         \item 	Avec le mot initial $M =  011_2$ et la clé $K = 101_2$, déterminer le mot crypté~$Y$.
         \item 	Comment, étant donné un mot crypté $Y$ et une clé $K$, retrouver le mot initial $M$ ?
     \end{enumerate}

\end{enumerate}

\end{document}