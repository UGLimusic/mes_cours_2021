\documentclass[a4paper,12pt]{article}
\usepackage[margin=2cm]{geometry}
\usepackage[thinfonts,latinmath]{uglix2}
\pagestyle{empty}
\nouveaustyle
\def\arraystretch{1.5}
\begin{document}
\titreinterro{Interrogation 05}{SIO1}{01/2022}

\exo{}

En utilisant la table de vérité ci-dessous, montrer que, quelles que soient les valeurs de vérité de P et Q, on a $$\barmaj{P} \vee \barmaj{Q}\Leftrightarrow \barmaj{P\wedge Q}$$

\begin{center}
	\begin{tabular}{|c|c|c|c|c|c|c|}
		\hline
		$P$ & $Q$ & $\barmaj{P}$ & $\barmaj{Q}$ & $\barmaj{P} \vee \barmaj{Q}$  & $P\wedge Q$ & $\barmaj{P\wedge Q}$ \\
		\hline
		&  &  &  &  &  &\\
		\hline
&		&  &  &  &  &  \\
		\hline
&		&  &  &  &  &  \\
		\hline
&		&  &  &  &  &  \\
		\hline
	\end{tabular}
\end{center}
Indiquer les colonnes identiques qui permettent de conclure.\\[1.5em]

\exo{}

En complétant, donner la table de vérité de $$\left(P\wedge Q\right)\vee \left(P\wedge\barmaj{Q}\right)$$

\begin{center}

	\begin{tabular}{|c|c|c|c|c|c|}
		\hline
		P & Q & \hspace{1cm} &\hspace{1cm} & \hspace{1cm}  &\hspace{1cm} \\
		\hline
		&  &  &  &  &  \\
		\hline
		&  &  &  &  &  \\
		\hline
		&  &  &  &  &  \\
		\hline
		&  &  &  & 	&    \\
		\hline
	\end{tabular}
\end{center}
Finalement, à quoi est égal $\left(P\wedge Q\right)\vee \left(P\wedge\barmaj{Q}\right)$ ?\\

\carreauxseyes{16.8}{3.2}

\newpage
\exo{ - On peut retrouver tous les opérateurs à partir du nand}

Pour toutes propositions $A$ et $B$ on définit l'opération \og nand\fg{}, notée $\uparrow$ par : $$A\uparrow B \Longleftrightarrow\barmaj{A\wedge B}$$

Cette opération est dite \textit{universelle} car elle permet de retrouver toutes les autres opérations.\\

\textbf{1.} 	Donner la table de vérité de l'opération nand.\\

\carreauxseyes{16.8}{5.6}\\
	
\textbf{2.}    Montrer que $A\uparrow A \Longleftrightarrow \barmaj{A}$ (on peut donc retrouver l'opération « non »).\\

\carreauxseyes{16.8}{5.6}\\


\textbf{3.}	En déduire que l'on peut retrouver l'opération « et » ainsi : $$(A\uparrow B)\uparrow(A\uparrow B) \Longleftrightarrow A\wedge B$$ \\

\carreauxseyes{16.8}{3.2}\\





\end{document}
