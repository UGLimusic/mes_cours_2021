\documentclass[a4paper,12pt]{article}
\usepackage[margin=2cm]{geometry}
\usepackage[thinfonts,latinmath]{uglix2}

\nouveaustyle
\pagestyle{empty}
\begin{document}
\titre{Devoir maison}{SIO1}{2022}

\exo{}\\

Soient $a$, $b$ et $c$ trois éléments d'une algèbre booléenne $\mathscr{B}$.

\begin{enumerate}[\bfseries 1.]
	\item 	Soient $A=ab+\barmin{c}$ et $B=\barmin{a}+bc$, montrer par le calcul que $$\barmaj{A}=\barmin{a}c+\barmin{b}c$$ et que $$\barmaj{B}=a\barmin{c}+a\barmin{b}$$
	\item 	Soit $C=\barmin{a}\,\barmin{b}\,\barmin{c}+a\barmin{b}c+\barmin{a}\,\barmin{b}+a\barmin{b}\,\barmin{c}$.\\
			Utiliser un diagramme de Karnaugh pour simplifier C.\\
\end{enumerate}

\exo{ - Métropole mai 2017}\\

Une salle dédiée à l'informatique va être aménagée au lycée.\\
Le réseau qui équipera cette salle doit satisfaire au moins l'une des conditions suivantes :

\setlength\parindent{9mm}
\begin{itemize}
	\item[$\bullet~~$] le réseau compte 5 postes ou plus et il existe un poste qui ne reçoit pas de données en entrée
	\item[$\bullet~~$] il existe un poste qui ne reçoit pas de données en entrée, et le réseau compte strictement moins  de 5 postes, et il comporte strictement plus de 12 connexions ;
	\item[$\bullet~~$] le réseau comporte 12 connexions ou moins.
\end{itemize}
\setlength\parindent{0mm}

\smallskip

On définit les variables booléennes suivantes:

\setlength\parindent{9mm}
\begin{enumerate}
	\item[$\bullet~~$] $a = 1$ si le réseau compte 5 postes ou plus, $a = 0$ sinon;
	\item[$\bullet~~$] $b = 1$ s'il existe un poste qui ne reçoit pas de données en entrée, $b = 0$ sinon;
	\item[$\bullet~~$] $c = 1$ si le réseau comporte 12 connexions ou moins, $c = 0$ sinon.
\end{enumerate}
\setlength\parindent{0mm}

\medskip

\begin{enumerate}[\bfseries 1.]
	\item Cette question est une question à choix multiple. Une seule réponse est correcte. Recopier sur la copie seulement la réponse correcte. On ne demande pas de justification.

	Parmi les quatre phrases suivantes, donner celle qui traduit la variable $\overline{b}$ :

	\setlength\parindent{9mm}
	\begin{enumerate}[\textbullet]
		\item réponse A : \og il existe un poste qui reçoit des données en entrée \fg{} ;
		\item réponse B : \og tout poste reçoit des données en entrée \fg{} ;
		\item réponse C : \og il existe un poste qui envoie des données en sortie \fg{} ;
		\item réponse D : \og aucun poste ne reçoit des données en entrée \fg.
	\end{enumerate}
	\setlength\parindent{0mm}

	\item Donner l'expression booléenne E traduisant les critères voulus pour un réseau informatique.
	\item À l'aide d'un tableau de Karnaugh ou par des calculs, exprimer $E$ comme somme de deux
	variables booléennes.
	\item Traduire les critères de sélection simplifiés, à partir de l'expression obtenue à la question 3.
	\item Un réseau dans lequel 2 postes ne reçoivent pas de données en entrée et qui comporte 15
	connexions répond-il aux critères voulus ? Justifier la réponse.\\
\end{enumerate}

\end{document}
