\documentclass[a4paper,12pt]{article}
\usepackage[margin=2cm,top=1cm]{geometry}		
\usepackage[thinfonts, latinmath]{uglix2}
\nouveaustyle

\begin{document}
\titre{Calcul des prédicats}{BTSSIO}{}

\exo{}

Sans chercher à démontrer quoi que ce soit, donner les négations des propositions suivantes

\begin{enumerate}[\bfseries 1.]
	\item 	$\forall x\in R,\, \forall y\in\R,\, \exists z\in \R,\, x<z<y$
	\item 	$\exists x\in R,\, \exists y\in\R,\, x+y>3$
	\item 	$\forall n\in N^*,\, \exists p\in\N^*$ n divise p ou p divise n\\
\end{enumerate}

\exo{}
\begin{methode}
	\begin{enumerate}[\textbullet]
		\item 	Pour prouver qu'une proposition quantifiée par $\forall$ est fausse, il suffit de donner un \textit{contre exemple}.
		\item 	Pour prouver qu'une proposition quantifiée par $\exists x...$ est vraie, on peut déterminer la valeur de $x$ qui convient.
		\item 	Pour prouver qu'une proposition quantifiée par $\forall$ est vraie on a souvent recours à un raisonnement ou au calcul littéral.
		\item 	De même pour prouver qu'une proposition quantifiée par $\exists x...$ est fausse.
	\end{enumerate}
\end{methode}

\begin{enumerate}[\bfseries 1.]
	\item 	A : $\forall n \in \N$ 3 divise n ou 2 divise n\\
			Montrer que A est fausse
	\item 	B : $\exists n \in \N$, 3 divise n et 4 divise n\\
			Montrer que B est vraie
	\item 	C : \og Quand on prend trois nombres entiers qui se suivent, leur somme est toujours un multiple de 3 \fg.\\
			Montrer que C est vraie.
	
	\item 	D : \og Quand on prend quatre nombres entiers qui se suivent, leur somme est toujours un multiple de 4 \fg.\\
				Montrer que D est fausse.
	\item 	E : \og Il existe deux entiers k et n plus grands que 1 tels que k divise à la fois n et n+1.\\
				Montrer que E est fausse. 
\end{enumerate}
\end{document}
