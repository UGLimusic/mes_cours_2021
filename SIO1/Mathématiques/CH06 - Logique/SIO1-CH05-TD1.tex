\documentclass[a4paper,12pt]{article}
\usepackage[margin=2cm]{geometry}		
\usepackage[thinfonts]{uglix2}
\usepackage{numprint}
\nouveaustyle
\pagestyle{empty}
\begin{document}
\titre{Tables de vérité}{SIO1}{}

\exo{}

En utilisant les tables de vérités, montrer que, quelles que soient les valeurs de vérité de P et Q, on a $$\neg P \wedge \neg Q \Leftrightarrow \neg(P\vee Q)$$

Compléter

\begin{center}
	
	\begin{tabular}{|c|c|c|c|c|c|c|}
		\hline 
		P & Q & $\neg P$ & $\neg Q$ & $\neg P \wedge \neg Q$  & $P\vee Q$ & $\neg(P\vee Q)$ \\ 
		\hline 
		&  &  &  &  &  &\\ 
		\hline 
&		&  &  &  &  &  \\ 
		\hline 
&		&  &  &  &  &  \\ 
		\hline 
&		&  &  &  &  &  \\ 
		\hline 
	\end{tabular} 
\end{center}
Indiquer les colonnes identiques qui permettent de conclure.\\

\exo{}

En utilisant les tables de vérités, montrer que, quelles que soient les valeurs de vérité de P et Q, on a $$(P\vee Q)\wedge (P\vee \neg Q)\Leftrightarrow P$$

Compléter

\begin{center}
	
	\begin{tabular}{|c|c|c|c|c|c|c|}
		\hline 
		P & Q & $\neg P$ & $\neg Q$ & $P \vee Q$  & $P\vee \neg Q$ & $(P\vee Q)\wedge (P\vee \neg Q)$\\ 
		\hline 
		&  &  &  &  &  &\\ 
		\hline 
		&  &  &  &  &  &\\ 
		\hline 
		&  &  &  &  &  &\\ 
		\hline 
		&  &  &  & 	&  &  \\ 
		\hline 
	\end{tabular} 
\end{center}
Indiquer les colonnes identiques qui permettent de conclure.
\end{document}
