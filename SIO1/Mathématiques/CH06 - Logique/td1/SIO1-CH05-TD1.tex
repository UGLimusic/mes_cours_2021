\documentclass[a4paper,12pt]{article}
\usepackage[margin=2cm]{geometry}
\usepackage[thinfonts,latinmath]{uglix2}
\pagestyle{empty}
\nouveaustyle
\begin{document}
\titre{Logique TD1}{SIO1}{}

\exo{}

En utilisant les tables de vérités, montrer que, quelles que soient les valeurs de vérité de P et Q, on a $$\neg P \wedge \neg Q \Leftrightarrow \neg(P\vee Q)$$

Compléter

\begin{center}

	\begin{tabular}{|c|c|c|c|c|c|c|}
		\hline
		P & Q & $\neg P$ & $\neg Q$ & $\neg P \wedge \neg Q$  & $P\vee Q$ & $\neg(P\vee Q)$ \\
		\hline
		&  &  &  &  &  &\\
		\hline
&		&  &  &  &  &  \\
		\hline
&		&  &  &  &  &  \\
		\hline
&		&  &  &  &  &  \\
		\hline
	\end{tabular}
\end{center}
Indiquer les colonnes identiques qui permettent de conclure.\\[2em]

\exo{}

En utilisant les tables de vérités, montrer que, quelles que soient les valeurs de vérité de P et Q, on a $$(P\vee Q)\wedge (P\vee \neg Q)\Leftrightarrow P$$

Compléter

\begin{center}

	\begin{tabular}{|c|c|c|c|c|c|c|}
		\hline
		P & Q & $\neg P$ & $\neg Q$ & $P \vee Q$  & $P\vee \neg Q$ & $(P\vee Q)\wedge (P\vee \neg Q)$\\
		\hline
		&  &  &  &  &  &\\
		\hline
		&  &  &  &  &  &\\
		\hline
		&  &  &  &  &  &\\
		\hline
		&  &  &  & 	&  &  \\
		\hline
	\end{tabular}
\end{center}
Indiquer les colonnes identiques qui permettent de conclure.\\[2em]


\exo{ - Lois de De Morgan}

En utilisant des tables de vérité, montrer que, quelles que soient les valeurs de vérité de P et Q, on a $$\barmaj{P\vee Q}=\barmaj{P}\wedge\barmaj{Q}$$
De même  montrer que $$\barmaj{P\wedge Q}=\barmaj{P}\vee\barmaj{Q}$$
\newpage
\exo{ - On peut retrouver tous les opérateurs à partir du nor}

Pour toutes propositions $A$ et $B$ on définit l'opération \og nor\fg{}, notée $\downarrow$ par : $$A\downarrow B \Longleftrightarrow\barmaj{A\vee B}$$

Cette opération est dite \textit{universelle} car elle permet de retrouver toutes les autres opérations.

\begin{enumerate}[\bfseries 1.]
	\item 	Montrer que $A\downarrow A \Longleftrightarrow \barmaj{A}$ (on peut donc retrouver l'opération « non »).
	\item 	En déduire que l'on peut retrouver l'opération « et » ainsi : $$(A\downarrow B)\downarrow(A\downarrow B) = A\vee B$$ 
	\item 	Comment à partir de $A$, $B$ et $\downarrow$ obtenir $A\wedge B$ (penser aux lois de De Morgan) ?
\end{enumerate}



\end{document}
