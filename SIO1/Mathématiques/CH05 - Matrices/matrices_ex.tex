\documentclass[a4paper,10pt]{article}
\usepackage[margin=2cm]{geometry}		
\usepackage[thinfonts]{uglix2}
\usepackage{graphicx}
\usepackage{array}
\newcommand{\tabstrut}{\vrule height 1.25em depth 0.5em width 0pt}
\nouveaustyle
\begin{document}
\titre{Matrices - Exercices}{\textsc{BTSSIO}}{}
	
	
\exo{}

Dans un parc d’une ville, deux marchands ambulants vendent des beignets,
des crêpes et des gaufres. On a noté les ventes de chacun pour samedi et dimanche derniers.

\begin{multicols}{2}
\begin{center}
Marchand 1\\[1em]

\begin{tabular}{|c|c|c|c|}
\hline
 & beignets & crêpes & gaufres \\
\hline
samedi & 20 & 36 & 12 \\
\hline
dimanche & 26 & 40 & 18 \\
\hline
\end{tabular}
\columnbreak


Marchand 2\\[1em]

\begin{tabular}{|c|c|c|c|}
\hline
 & beignets & crêpes & gaufres \\
\hline
samedi & 30 & 40 & 22 \\
\hline
dimanche & 30 & 48 & 38 \\
\hline
\end{tabular}
\end{center}
\end{multicols}

On peut retenir l’information donnée par un tableau en conservant uniquement les
nombres disposés de la même façon. On représente le 1er tableau par la matrice A :
$$A=\begin{pmatrix}
20&36&12\\
26&40&18
\end{pmatrix}$$

\begin{enumerate}[\bfseries 1.]
	\item 	Donner la matrice B représentant le deuxième tableau.
	\item 	Que valent $a_{12}$, $a_{11}$, $a_{23}$ et $b_{11}$ ?
	\item 	Calculer $A+B$ et donner la signification de la matrice.
	\item 	Calculer $A-B$ et donner la signification de la matrice.
	\item  	Samedi et dimanche prochains, weekend de fête, on prévoit que les ventes vont augmenter de 50\%.\\ 
			Par quel nombre k faut-il multiplier chacune des ventes du 1\er marchand ? \'Ecrire la matrice  kA.\\
			Donner la matrice kB correspondant aux ventes du 2\eme marchand.
	\item 	Un beignet est vendu 2 euros, une crêpe 1 euro et une gaufre 1,50 euro.\\
			On note V la matrice des prix de vente
			$$V=\begin{pmatrix}
			2\\
			1\\
			1,5
			\end{pmatrix}$$
			Quelle opération matricielle donne le montant des ventes par jour pour le 1\er marchand ? Pour le 2\eme ?
	\item 	 Les deux marchands travaillent pour le compte du même patron, qui leur demande	de calculer les coûts d’achats et les revenus pour chaque jour. Le coût d’achat d’un
			beignet est 0,40 euro, d’une crêpe 0,25 euro, d’une gaufre 0,30 euro. On note T la matrice donnant prix d’achat et prix de vente par catégorie
			$$T=\begin{pmatrix}
						0,4 & 2\\
						0,25 & 1\\
						0,3 & 1,5
						\end{pmatrix}$$
			Quelle opération matricielle permet le calcul des coûts d’achat et revenus par jour
			pour le 1\er marchand ?\\
			Calculer, de même, les coûts d’achats et les revenus par jour pour le 2\eme marchand
			puis, globalement, pour le patron.			
\end{enumerate}

\exo{- Calculs à la main}

On considère les matrice $A=\begin{pmatrix}
-5&2\\
-3&1
\end{pmatrix}$ et $B=\begin{pmatrix}
1&-2\\
3&-5
\end{pmatrix}$.

\begin{enumerate}[\bfseries 1.]
	\item 	Montrer \textit{à la main} que A et B sont inverses.
	\item 	On considère le système (S) suivant :
			$$\begin{cases}
				-5x+2y	&=7	\\
				-3x+y	&=8
			\end{cases}$$
			
			Montrer que ce système peut se réécrire matriciellement $$AX=Y$$ et préciser $X$ et $Y$
	\item 	En déduire \textit{à la main} les solutions du système (S).\\
\end{enumerate}


\exo{- Avec calculatrice}

On considère les matrice $A=\begin{pmatrix}
8&11&3\\
4&7&2\\
1&3&1\\
\end{pmatrix}$ et $B=\begin{pmatrix}
1&-2&1\\
-2&5&-4\\
5&-13&12
\end{pmatrix}$.

\begin{enumerate}[\bfseries 1.]
	\item 	Comment avec la calculatrice vérifie-t-on que A et B sont inverses ?
	\item 	On considère le système (S) suivant :
			$$\begin{cases}
				8x+11y+3z	&=1	\\
				4x+7y+2z	&=2 \\
				x+3y+z&=3
			\end{cases}$$
			
			Montrer que ce système peut se réécrire matriciellement $$AX=Y$$ et préciser $X$ et $Y$
	\item 	En déduire \textit{à la main} les solutions du système (S).\\
\end{enumerate}

\exo{}

À la papeterie:
\begin{enumerate}[--]
	\item 	3 stylos, 2 cahiers et 4 gommes coûtent 6,30€;
	\item 	5 stylos, 7 cahiers et 1 gomme coûtent 15€;
	\item 	10 stylos, 1 cahier et 6 gommes coûtent 6€.
\end{enumerate}
À l'aide de la calculatrice et en expliquant la démarche, déterminer le prix de chaque article.
\end{document}
