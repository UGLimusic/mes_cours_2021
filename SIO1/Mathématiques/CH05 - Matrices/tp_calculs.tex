\documentclass[a4paper,12pt,french]{article}
\usepackage[margin=2cm]{geometry}
\usepackage[thinfonts]{uglix2}
\nouveaustyle
\begin{document}
\titre{TP Matrices}{SIO1}{12/2021}

\begin{exercice}[]
$A=\begin{pmatrix}
1&0&-2\\
2&3&1
\end{pmatrix}$,
$B=\begin{pmatrix}
1&0&1\\
2&1&3\\
3&-1&-2
\end{pmatrix}$ et
$C=\begin{pmatrix}
3&0&-2&0\\
-2&1&1&1\\
1&-1&0&3
\end{pmatrix}$.\\

\begin{enumerate}[\bfseries 1.]
	\item 	Calculer $A\times B$, puis $(A\times B)\times C$.
    \item 	Calculer $B\times C$, puis $A\times(B\times C)$.
    \item   Pouvait-on prévoir ce résultat ?\\
\end{enumerate}
\end{exercice}

\begin{exercice}
 $A=\begin{pmatrix}
 1&3\\
 2&6
 \end{pmatrix}$,
 $B=\begin{pmatrix}
 -3&-6\\
 1&2
 \end{pmatrix}$,
 $C=\begin{pmatrix}
 1&1&2\\
 2&2&4\\
 3&3&6
 \end{pmatrix}$ et
 $D=\begin{pmatrix}
 1&2&-6\\
 1&2&-6\\
 1&-2&6
 \end{pmatrix}$.

 \begin{enumerate}[\bfseries 1.]
 	\item 	Calculer le produit $A\times B$.
 	\item 	Calculer le produit $C\times D$.
     \item  Que peut-on en conclure ?
 \end{enumerate}

\end{exercice}


\begin{exercice}[ ]

$A=\begin{pmatrix}
1&1\\
0&1
\end{pmatrix}$.\\

Calculer $A^5$.
\end{exercice}




\end{document}
