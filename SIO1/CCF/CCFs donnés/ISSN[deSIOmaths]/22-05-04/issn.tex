% Options for packages loaded elsewhere
\PassOptionsToPackage{unicode}{hyperref}
\PassOptionsToPackage{hyphens}{url}
%
\documentclass[
  12pt,
  a4paper,
]{article}
\usepackage{amsmath,amssymb}
\usepackage{lmodern}
\usepackage{setspace}
\usepackage{iftex}
\ifPDFTeX
  \usepackage[T1]{fontenc}
  \usepackage[utf8]{inputenc}
  \usepackage{textcomp} % provide euro and other symbols
\else % if luatex or xetex
  \usepackage{unicode-math}
  \defaultfontfeatures{Scale=MatchLowercase}
  \defaultfontfeatures[\rmfamily]{Ligatures=TeX,Scale=1}
  \setmainfont[Numbers=Lowercase,Numbers=Proportional]{Fira Sans Light}
  \setmonofont[]{Fira Mono}
  \setmathfont[]{Fira Math Light}
\fi
% Use upquote if available, for straight quotes in verbatim environments
\IfFileExists{upquote.sty}{\usepackage{upquote}}{}
\IfFileExists{microtype.sty}{% use microtype if available
  \usepackage[]{microtype}
  \UseMicrotypeSet[protrusion]{basicmath} % disable protrusion for tt fonts
}{}
\makeatletter
\@ifundefined{KOMAClassName}{% if non-KOMA class
  \IfFileExists{parskip.sty}{%
    \usepackage{parskip}
  }{% else
    \setlength{\parindent}{0pt}
    \setlength{\parskip}{6pt plus 2pt minus 1pt}}
}{% if KOMA class
  \KOMAoptions{parskip=half}}
\makeatother
\usepackage{xcolor}
\usepackage[margin = 2cm,landscape]{geometry}
\usepackage{color}
\usepackage{fancyvrb}
\newcommand{\VerbBar}{|}
\newcommand{\VERB}{\Verb[commandchars=\\\{\}]}
\DefineVerbatimEnvironment{Highlighting}{Verbatim}{commandchars=\\\{\}}
% Add ',fontsize=\small' for more characters per line
\newenvironment{Shaded}{}{}
\newcommand{\AlertTok}[1]{\textcolor[rgb]{1.00,0.00,0.00}{\textbf{#1}}}
\newcommand{\AnnotationTok}[1]{\textcolor[rgb]{0.38,0.63,0.69}{\textbf{\textit{#1}}}}
\newcommand{\AttributeTok}[1]{\textcolor[rgb]{0.49,0.56,0.16}{#1}}
\newcommand{\BaseNTok}[1]{\textcolor[rgb]{0.25,0.63,0.44}{#1}}
\newcommand{\BuiltInTok}[1]{#1}
\newcommand{\CharTok}[1]{\textcolor[rgb]{0.25,0.44,0.63}{#1}}
\newcommand{\CommentTok}[1]{\textcolor[rgb]{0.38,0.63,0.69}{\textit{#1}}}
\newcommand{\CommentVarTok}[1]{\textcolor[rgb]{0.38,0.63,0.69}{\textbf{\textit{#1}}}}
\newcommand{\ConstantTok}[1]{\textcolor[rgb]{0.53,0.00,0.00}{#1}}
\newcommand{\ControlFlowTok}[1]{\textcolor[rgb]{0.00,0.44,0.13}{\textbf{#1}}}
\newcommand{\DataTypeTok}[1]{\textcolor[rgb]{0.56,0.13,0.00}{#1}}
\newcommand{\DecValTok}[1]{\textcolor[rgb]{0.25,0.63,0.44}{#1}}
\newcommand{\DocumentationTok}[1]{\textcolor[rgb]{0.73,0.13,0.13}{\textit{#1}}}
\newcommand{\ErrorTok}[1]{\textcolor[rgb]{1.00,0.00,0.00}{\textbf{#1}}}
\newcommand{\ExtensionTok}[1]{#1}
\newcommand{\FloatTok}[1]{\textcolor[rgb]{0.25,0.63,0.44}{#1}}
\newcommand{\FunctionTok}[1]{\textcolor[rgb]{0.02,0.16,0.49}{#1}}
\newcommand{\ImportTok}[1]{#1}
\newcommand{\InformationTok}[1]{\textcolor[rgb]{0.38,0.63,0.69}{\textbf{\textit{#1}}}}
\newcommand{\KeywordTok}[1]{\textcolor[rgb]{0.00,0.44,0.13}{\textbf{#1}}}
\newcommand{\NormalTok}[1]{#1}
\newcommand{\OperatorTok}[1]{\textcolor[rgb]{0.40,0.40,0.40}{#1}}
\newcommand{\OtherTok}[1]{\textcolor[rgb]{0.00,0.44,0.13}{#1}}
\newcommand{\PreprocessorTok}[1]{\textcolor[rgb]{0.74,0.48,0.00}{#1}}
\newcommand{\RegionMarkerTok}[1]{#1}
\newcommand{\SpecialCharTok}[1]{\textcolor[rgb]{0.25,0.44,0.63}{#1}}
\newcommand{\SpecialStringTok}[1]{\textcolor[rgb]{0.73,0.40,0.53}{#1}}
\newcommand{\StringTok}[1]{\textcolor[rgb]{0.25,0.44,0.63}{#1}}
\newcommand{\VariableTok}[1]{\textcolor[rgb]{0.10,0.09,0.49}{#1}}
\newcommand{\VerbatimStringTok}[1]{\textcolor[rgb]{0.25,0.44,0.63}{#1}}
\newcommand{\WarningTok}[1]{\textcolor[rgb]{0.38,0.63,0.69}{\textbf{\textit{#1}}}}
\setlength{\emergencystretch}{3em} % prevent overfull lines
\providecommand{\tightlist}{%
  \setlength{\itemsep}{0pt}\setlength{\parskip}{0pt}}
\setcounter{secnumdepth}{-\maxdimen} % remove section numbering
\ifLuaTeX
  \usepackage{selnolig}  % disable illegal ligatures
\fi
\IfFileExists{bookmark.sty}{\usepackage{bookmark}}{\usepackage{hyperref}}
\IfFileExists{xurl.sty}{\usepackage{xurl}}{} % add URL line breaks if available
\urlstyle{same} % disable monospaced font for URLs
\hypersetup{
  pdftitle={Corrigé du CCF sur l'ISSN},
  pdfauthor={NSI2},
  hidelinks,
  pdfcreator={LaTeX via pandoc}}

\title{Corrigé du CCF sur l'ISSN}
\author{NSI2}
\date{}

\begin{document}
\maketitle

\setstretch{1.2}
\hypertarget{question-a-1}{%
\subsubsection{Question A-1}\label{question-a-1}}

On effectue le calcul :

\(8\times 0+7\times 3+6\times 9+5\times 8+4\times 1+3\times 1+2\times 6 = 134\).

Donc \(N=134\).

Alors \(r=134\mod 11 = 2\) et la clé est \(11-2=9\).

Donc ce numéro est valide.

Pour compléter l'ISSN 0373-800 on fait pareil :

\(8\times 0+7\times 3+6\times 7+5\times 3+4\times 8+3\times 0+2\times 0 = 110\).

Donc \(N=110\).

Alors \(r=110\mod 11 = 0\) et la clé est \(0\).

Donc ce numéro ISSN complété est 0373-8000.

\hypertarget{question-a-2}{%
\subsubsection{Question A-2}\label{question-a-2}}

On n'a réécrit que la partie utile de l'algorithme :

\begin{verbatim}
Début
    N ← 0
    m ← 8
    Pour i allant de 0 à 6
        N ← N + m * int(code7[i])
        m ← m - 1
    FinPour
    Renvoyer N
Fin
\end{verbatim}

Voici la traduction en Python :

\begin{Shaded}
\begin{Highlighting}[]
\KeywordTok{def}\NormalTok{ calcN(code7: }\BuiltInTok{str}\NormalTok{) }\OperatorTok{{-}\textgreater{}} \BuiltInTok{int}\NormalTok{:}
\NormalTok{    N }\OperatorTok{=} \DecValTok{0} \CommentTok{\# on initialise N à 0}
\NormalTok{    m }\OperatorTok{=} \DecValTok{8} \CommentTok{\# on commence par multiplier par 8}
    \ControlFlowTok{for}\NormalTok{ i }\KeywordTok{in} \BuiltInTok{range}\NormalTok{(}\DecValTok{7}\NormalTok{):  }\CommentTok{\# pour parcourir code7}
\NormalTok{        N }\OperatorTok{+=}\NormalTok{ m }\OperatorTok{*} \BuiltInTok{int}\NormalTok{(code7[i])  }\CommentTok{\# on rajoute la bonne valeur}
\NormalTok{        m }\OperatorTok{{-}=}\DecValTok{1} \CommentTok{\# m "descend"}
    \ControlFlowTok{return}\NormalTok{ N }\CommentTok{\# on renvoie le résultat}
\end{Highlighting}
\end{Shaded}

\hypertarget{question-a-3} \DecValTok{11} \CommentTok{\# on calcule le reste}
    \ControlFlowTok{if} \KeywordTok{not}\NormalTok{ r:}
        \ControlFlowTok{return} \StringTok{\textquotesingle{}0\textquotesingle{}}  \CommentTok{\# comme dit, s\textquotesingle{}il vaut 0 on renvoie \textquotesingle{}0\textquotesingle{}}
    \ControlFlowTok{elif}\NormalTok{ r }\OperatorTok{==} \DecValTok{1}\NormalTok{:}
        \ControlFlowTok{return} \StringTok{\textquotesingle{}X\textquotesingle{}} \CommentTok{\# Sinon s\textquotesingle{}il vaut 1 on renvoie \textquotesingle{}X\textquotesingle{}}
    \ControlFlowTok{else}\NormalTok{:}
        \ControlFlowTok{return} \BuiltInTok{str}\NormalTok{(}\DecValTok{11} \OperatorTok{{-}}\NormalTok{ r) }\CommentTok{\# Sinon cela}
\end{Highlighting}
\end{Shaded}

\hypertarget{question-a-4}{%
\subsubsection{Question A-4}\label{question-a-4}}

De même

\begin{Shaded}
\begin{Highlighting}[]
\KeywordTok{def}\NormalTok{ ISSNok(code8: }\BuiltInTok{str}\NormalTok{) }\OperatorTok{{-}\textgreater{}} \BuiltInTok{bool}\NormalTok{:}
\NormalTok{    code7 }\OperatorTok{=}\NormalTok{ code8[:}\DecValTok{7}\NormalTok{] }\CommentTok{\# On ne prend que les chiffres sans la clé}
\NormalTok{    cle }\OperatorTok{=}\NormalTok{ code8[}\DecValTok{7}\NormalTok{] }\CommentTok{\# on stocke la clé}
\NormalTok{    N }\OperatorTok{=}\NormalTok{ calcN(code7) }\CommentTok{\# on calcule N pour les 7 premiers chiffres}
    \ControlFlowTok{if}\NormalTok{ cle }\OperatorTok{==}\NormalTok{ calcCle(N):   }\CommentTok{\# on vérifie si la clé calculée correspond}
        \ControlFlowTok{return} \VariableTok{True}  \CommentTok{\# si oui c\textquotesingle{}est bien}
    \ControlFlowTok{else}\NormalTok{:}
        \ControlFlowTok{return} \VariableTok{False} \CommentTok{\# sinon ce n\textquotesingle{}est pas bien}
\end{Highlighting}
\end{Shaded}

\hypertarget{question-a-5}{%
\subsubsection{Question A-5}\label{question-a-5}}

De même

\begin{Shaded}
\begin{Highlighting}[]
\KeywordTok{def}\NormalTok{ bonus(code8:}\BuiltInTok{str}\NormalTok{) }\OperatorTok{{-}\textgreater{}} \VariableTok{None}\NormalTok{:}
\NormalTok{    code6}\OperatorTok{=}\NormalTok{ code8[}\DecValTok{1}\NormalTok{:}\DecValTok{7}\NormalTok{] }\CommentTok{\# On ne garde que les 6 chiffres, sans les extrémités }
    \ControlFlowTok{for}\NormalTok{ i }\KeywordTok{in} \BuiltInTok{range}\NormalTok{(}\DecValTok{10}\NormalTok{):  }\CommentTok{\# i va parcourir toutes les valeurs de 0 à 9}
\NormalTok{        code\_variant }\OperatorTok{=} \BuiltInTok{str}\NormalTok{(i)}\OperatorTok{+}\NormalTok{code6 }\CommentTok{\# on forme un code de 7 chiffres commençant par i}
        \BuiltInTok{print}\NormalTok{(}\SpecialStringTok{f"Pour }\SpecialCharTok{\{}\NormalTok{code\_variant}\SpecialCharTok{\}}\SpecialStringTok{ on trouve une  }\CharTok{\textbackslash{}}
\SpecialStringTok{        clé de }\SpecialCharTok{\{}\NormalTok{calcCle(calcN(code\_variant))}\SpecialCharTok{\}}\SpecialStringTok{."}\NormalTok{)  }\CommentTok{\# on affiche la clé du code obtenu}
\end{Highlighting}
\end{Shaded}


\end{document}
