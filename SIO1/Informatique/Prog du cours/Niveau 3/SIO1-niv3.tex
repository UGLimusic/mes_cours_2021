\documentclass[a4paper,12pt,french]{book}
\usepackage[margin=2cm]{geometry}
\usepackage[thinfonts]{uglix2}
\nouveaustyle

\begin{document}
\titre{Du cours au programme Python}{SIO1-3}{11-2021}

\begin{methode}[ 1 : passer de la base 2 à la base 10]
Que vaut $(11101)_2$ ?
\begin{center}
	\begin{tabular}{|c|c|c|c|c|c|}
		\hline 
		Chiffre binaire & 1 & 1 & 1 & 0 & 1 \\ 
		\hline 
		Valeur & $2^4$ & $2^3$ & $2^2$ & $2^1$ & $2^0$ \\ 
		\hline 
	\end{tabular}
\end{center}
\begin{tabbing}
		$(11101)_2$	\= 	$=1\times 2^4+1\times 2^3+1\times 2^2+0\times 2^1+1\times 2^0$	\\
			\>	$=16+8+4+1$	\\	
			\>	$=29$	
	\end{tabbing}\nopagebreak
\end{methode}
\begin{exercice}[]
Écrire une fonction \tw{methode1} qui
\begin{enumerate}[--]
	\item 	en entrée prend un \tw{str} composé de 0 et de 1 qui est l'écriture binaire d'un entier;
	\item 	renvoie un \texttt{int} qui est cet entier (en décimal, donc).
\end{enumerate}
Tester la fonction.
\end{exercice}

\begin{methode}[ 2 : passer de la base 10 à la base 2]
\begin{tabbing}
	203	\= 	$=128+64+8+2+1$	\\
	
		\>	$=2^7+2^6+2^3+2^1+2^0$	\\
		
		\>	$=1\times 2^7+1\times 2^6+0\times 2^5 + 0\times 2^4 +1\times 2^3+0\times 2^2 + 1\times 
		2^1+1\times 2^0$	\\
		
		\> $=(11001011)_2$
\end{tabbing}
\end{methode}

\begin{exercice}[]
\'Ecrire une fonction \texttt{methode2} qui :
\begin{enumerate}[--]
	\item 	en entrée prend  un entier positif ;
	\item 	renvoie l'écriture en binaire de cet entier dans un \texttt{str} en utilisant la méthode 2.\\
\end{enumerate}
Tester la fonction.
\end{exercice}

\begin{methode}[ 3 : les divisions successives]
Voici comment on trouve les chiffres de l'écriture \textit{binaire} de 203 :
$$\division[2]{203}$$
En définitive, $203=(11001011)_2$.
\end{methode}


\begin{exercice}[]
\'Ecrire une fonction \texttt{methode3} qui :
\begin{enumerate}[--]
	\item 	en entrée prend  un entier positif ;
	\item 	renvoie l'écriture en binaire de cet entier dans un \texttt{str} en utilisant la méthode 3.\\
\end{enumerate}
Tester la fonction.
\end{exercice}
\end{document}
