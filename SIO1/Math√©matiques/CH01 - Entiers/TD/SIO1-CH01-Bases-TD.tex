\documentclass[a4paper,10pt]{book}
\usepackage[margin=2cm]{geometry}
\usepackage[thinfonts]{uglix2}
\nouveaustyle
\newcommand{\carre}[2]
{\draw[thick,fill  = UGLiOrange!50] (#1+.1,#2+.1) rectangle (#1+.9,#2.9);
}

\begin{document}


\chapter*{Exemples d'utilisations du \\ binaire et de l'hexadécimal}

\exo{ - Cryptage d'un message \tiny (extrait de métropole mai 2019)}\\


On dispose d'une clé de cryptage notée $\left(a_1,\:a_2,\:a_3,\:a_4,\:a_5\right) = (63, 62, 65, 68, 34)$.

Cette clé, publiée par la personne destinataire, permet à quiconque de lui envoyer un message crypté. Voici comment on crypte le message.

On associe d'abord à chaque lettre son rang dans l'alphabet, selon la correspondance suivante :

\begin{center}
	\begin{tabularx}{\linewidth}{|l|*{13}{>{\centering \arraybackslash}X|}}\hline
\rowcolor{UGLiOrange!25}		Lettre&A&B&C&D&E&F&G&H&I&J&K&L&M\\ \hline
		Rang &1&2&3&4&5&6&7&8&9&10&11&12&13\\ \hline
\rowcolor{UGLiOrange!25}	Lettre &N&O&P&Q&R&S&T&U&V&W&X&Y&Z\\ \hline
		Rang&14&15&16&17&18&19&20&21&22&23&24&25&26\\ \hline
	\end{tabularx}
\end{center}

Pour crypter une lettre:
\begin{itemize}
	\item on détermine son rang à l'aide du tableau de correspondance précédent;
	\item on écrit ce nombre en base 2 sur 5 bits; on ainsi obtient 5 chiffres $\left(m_1, m_2, m_3,  m_4, m_5\right)$, chaque chiffre étant égal à $0$ ou à $1$ ;
	\item on détermine alors la valeur cryptée, égale à la somme $\sigma =  a_1m_1 + a_2m_2 + a_3m_3 + a_4m_4 + a_5m_5$.
\end{itemize}

On remarque qu'une lettre est ainsi cryptée par un nombre entier.

\emph{Exemple} : on veut crypter la lettre \og I \fg.

\begin{itemize}
	\item Le rang de I est $9$  ;
	\item on écrit ce nombre en base deux sur 5 bits : $9_{10} = 8 + 1= (01001)_2$,
	\item on calcule la somme $\sigma=0 \times 63 + 1 \times62 + 0 \times 65 + 0 \times 68 + 1 \times 34 = 96$.
\end{itemize}

La lettre \og I \fg est donc cryptée par l'entier $96$.\\

\textbf{Question :} crypter la lettre \og W  \fg{} selon cette méthode.\\


\exo{ - Un petit jeu}\\

On veut programmer un petit jeu : une bille commence dans la grille suivante, toujours à la case départ.\\
On doit l'amener à la case arrivée en appuyant seulement sur les touches \touche{$\rightarrow$}\ \ et \touche{$\downarrow$}\ \ du clavier.
Voici un exemple de partie :
\begin{center}
	\begin{tikzpicture}
		\foreach \i in {0,1,2,3}
		{\foreach \j in {0,1,2,3}
			{\carre{\i}{\j}}}
		\draw[thick,fill=UGLiBlue] (-.3,3.9)--(-.3,-.3)--(3.9,-.3)--(3.9,-.1)--(-.1,-.1)--(-.1,3.9)--cycle;
		\draw[thick,fill=UGLiBlue] (.1,4.1)--(4.1,4.1)--(4.1,.1)--(4.3,.1)--(4.3,4.3)--(.1,4.3)--cycle;
		\node[left] at (-.1,4.1) {départ};
		\node[right] at (4.1,-.1) {arrivée};
		\draw[ultra thick,->](0,4)--(0,3)--(2,3)--(2,2)--(3,2)--(3,0)--(4.1,0);
	\end{tikzpicture}
\end{center}

Pour représenter les différents parcours, on associe à chacun d'entre eux un entier de la manière suivante :
\begin{enumerate}[--]
	\item 	on note la séquence de touches pressées dans l'ordre;
	\item 	on remplace chaque \touche{$\downarrow$}\ \ par 0 et chaque \touche{$\rightarrow$}\ \ par 1;
	\item 	on obtient l'écriture binaire de l'entier final.
\end{enumerate}
\begin{enumerate}[\bfseries 1.]
	\item 	Justifier que l'entier obtenu à partir de l'exemple vaut 105.
	\item 	Combien faut-il de bits pour représenter un chemin donné ?
	\item 	Représenter le parcours associé au nombre 85.
	\item 	On considère un entier $N$ représentant un chemin.
	\begin{enumerate}[\bfseries a.]
		\item 	Est-il possible d'avoir $N=31$ ?
		\item 	Rappeler ce que vaut $ (\underbrace{1...1}_{\text{k bits}})_2$.
		\item 	Quel est le plus petit entier $N$ qui représente un chemin ? Dessiner le chemin associé.
		\item 	Quel est le plus grand ? Dessiner le chemin associé.\\
	\end{enumerate}
\end{enumerate}


\exo{ - Additions en base 2}

Les additions \og à la main\fg{} en base 2 s'effectuent de la même manière qu'en base 10, la seule différence c'est que deux 1 donnent $(2)_{10}$ donc $(10)_2$, donc un zéro et une retenue de 1.
Quand il y a deux 1 et une retenue de 1 en plus, cela donne $(3)_{10}$ donc $(11)_2$, donc un 1 et une retenue de 1.

\begin{exemple}[]
	On veut calculer $(1101\ 0101)_2+(1\ 1100)_2$, on pose l'opération en mettant les retenues en rouge.
	\begin{center}
		\begin{tabular}{ccccccccc}
			&   &   &  & {\color{red}\textbf{\footnotesize 1}} & {\color{red}\textbf{\footnotesize 1}} &  &  &  \\
			& 1 & 1 & 0 & 1 & 0 & 1 & 0 & 1 \\
			+ &  &  &  & 1 & 1 & 1 & 0 & 0 \\
			\hline
			=& 1 & 1 &  1& 1 & 0 & 0 & 0 & 1 \\
		\end{tabular} \\[2em]
	\end{center}
Le résultat est donc $(1111\ 0001)_2$.
\end{exemple}
\begin{enumerate}[\bfseries 1.]
	\item 	Donner les écritures décimales de  $(1101\ 0101)_2$, $(1\ 1100)_2$, $(1111\ 0001)_2$ et vérifier qu'il n'y a pas d'erreur d'addition.
	\item 	Vérifier que l'on a bien l'égalité $138+111=249$ lorsque l'on effectue l'addition en binaire.
	\item 	Calculer $(1011\ 1101)_2+(11)_2$ et donner les écritures décimales des 3 nombres intervenant dans cette addition.\\
\end{enumerate}

\exo{ - Entiers non signés sur un octet}\\

Dans certains langages, on utilise un octet (c'est-à-dire 8 bits) pour stocker un entier positif. On stocke dans l'octet l'écriture binaire de l'entier, telle quelle.
\begin{enumerate}[--]
	\item 	en C et C++, ce type de variable s'appelle \tw{unsigned char};
	\item 	en C\#, il s'appelle \tw{byte}.
\end{enumerate}
\begin{enumerate}[\bfseries 1.]
	\item 	Quels sont les entiers représentables de cette manière ?
	\item 	Pour ajouter deux valeurs de ce type, on effectue l'addition en binaire, mais s'il y a une retenue à reporter à la fin (qui irait théoriquement dans un 9ème bit) on ne la prend pas en compte : on ne garde que les 8 premiers bits du résultats en partant de la droite.
	\begin{enumerate}[\bfseries a.]
		\item 	Ajouter avec cette méthode 129 et 130 et donner le résultat sous forme décimale.
		\item	Ci dessus on a écrit un petit programme en C++. Dire ce qu'il fait et interpréter le résultat affiché dans la console.
	\end{enumerate}
\begin{encadrecolore}{Code C++}{UGLiOrange}
\begin{minted}[breaklines]{cpp}
#include <iostream> // bibliothèque d'affichage
int main() // début de la fonction principale
{
    unsigned char c = 0; // on définit la variable c
    for (int i = 0; i < 300; i++) // on fait une boucle pour
    {
        std::cout << "valeur de i : " << i; // on affiche la valeur de i
        std::cout << "  et valeur de c : " << (int)c; // on affiche la valeur de c en base 10
        std::cout << endl; // on revient à la ligne (END Line)
        c++; // on augmente c de 1
    }
    return 0; // la fonction principale renvoie zéro par convention
}
\end{minted}
\end{encadrecolore}
Voilà ce que la console affiche :\\

\texttt{valeur de i : 0  et valeur de c : 0}\\
\texttt{valeur de i : 1  et valeur de c : 1}\\
\textit{et c\ae tera}\\
\texttt{valeur de i : 254  et valeur de c : 254}\\
\texttt{valeur de i : 255  et valeur de c : 255}\\
\texttt{valeur de i : 256  et valeur de c : 0}\\
\texttt{valeur de i : 257  et valeur de c : 1}\\
\textit{et c\ae tera}\\
\texttt{valeur de i : 298  et valeur de c : 42}\\
\texttt{valeur de i : 299  et valeur de c : 43}\\

\end{enumerate}

\exo{ - Entiers signés en complément à 2 sur 8 bits.}\\

Pour représenter un entier signé (positif ou négatif) sur un octet, on peut penser au système suivant : le bit de poids fort vaut 0 si l'entier est positif et 1 s'il est négatif. Les 7 autres bits servent à représenter la partie numérique du nombre.\\
Par exemple, $$\Large(\color{red}1\color{blue}001\ 1010\color{black})_2$$
représente un nombre négatif car son bit de poids fort est à 1. Sa partie numérique est $(1\ 1010)_2=26$, donc il représente -26.
\begin{enumerate}[\bfseries 1.]
	\item 	\begin{enumerate}[\bfseries a.]
		\item 	Que représente $(0000\ 0000)_2$ ? Et $(1000\ 0000)_2$ ?
		\item 	Donner la représentation de +26 et ajouter les représentations de +26 et -26 en binaire. Quel nombre représente le résultat obtenu ? Est-ce cohérent ?
	\end{enumerate}
	\item 	Pour pallier ces problèmes, on représente les entiers \textit{en complément à 2 sur 8 bits} :
			\begin{enumerate}[--]
				\item 	si on veut représenter un entier compris entre 0 et $2^7-1=127$, alors on le représente tel quel sur un octet;
				\item 	si on veut représenter un entier $n$ compris entre $-2^8=-128$ et -1, alors on le représente par $n+256$.
			\end{enumerate}

\begin{center}
	\begin{tikzpicture}[scale=.5]
		\def\RayonListe{6.5}
		\def\EpaisseurListe{1.5}
		\def\LongueurListe{16}
		\draw[fill = UGLiOrange!15] (0,0) circle(\RayonListe+\EpaisseurListe);
		\draw[fill = white] (0,0) circle(\RayonListe);
		\foreach \compt in {0,1,...,\numexpr\LongueurListe-1}
		{\draw (90-\compt/\LongueurListe*360:\RayonListe)--(90-\compt/\LongueurListe*360:\RayonListe+\EpaisseurListe);}
		\foreach \compt in {0,1,2}	{\draw (90-\compt/\LongueurListe*360-180/\LongueurListe:\RayonListe+\EpaisseurListe/2)node{\color{UGLiGreen}\compt};}
		\foreach \compt in {3,4,5}	{\draw (90-\compt/\LongueurListe*360-180/\LongueurListe:\RayonListe+\EpaisseurListe/2)node{\color{UGLiGreen}...};}
		\def \compt{6}	{\draw (90-\compt/\LongueurListe*360-180/\LongueurListe:\RayonListe+\EpaisseurListe/2)node{\color{UGLiGreen} 126};}
		\def \compt{7}	{\draw (90-\compt/\LongueurListe*360-180/\LongueurListe:\RayonListe+\EpaisseurListe/2)node{ \color{UGLiGreen} 127};}
		\def \compt{8}	{\draw (90-\compt/\LongueurListe*360-180/\LongueurListe:\RayonListe+\EpaisseurListe/2)node{\color{UGLiRed} -128};}
		\def \compt{9}	{\draw (90-\compt/\LongueurListe*360-180/\LongueurListe:\RayonListe+\EpaisseurListe/2)node{ \color{UGLiRed} -127};}
		\foreach \compt in {10,11,12}	{\draw (90-\compt/\LongueurListe*360-180/\LongueurListe:\RayonListe+\EpaisseurListe/2)node{\color{UGLiRed}...};}
		\def \compt{13}	{\draw (90-\compt/\LongueurListe*360-180/\LongueurListe:\RayonListe+\EpaisseurListe/2)node{\color{UGLiRed} -3};}
		\def \compt{14}	{\draw (90-\compt/\LongueurListe*360-180/\LongueurListe:\RayonListe+\EpaisseurListe/2)node{\color{UGLiRed} -2};}
		\def \compt{15}	{\draw (90-\compt/\LongueurListe*360-180/\LongueurListe:\RayonListe+\EpaisseurListe/2)node{\color{UGLiRed} -1};}
		\def\RayonListe{5}
		\def\EpaisseurListe{1.5}
		\def\LongueurListe{16}
		\draw[fill=UGLiBlue!15] (0,0) circle(\RayonListe+\EpaisseurListe);
		\draw[fill=white] (0,0) circle(\RayonListe);
		\foreach \compt in {0,1,...,\numexpr\LongueurListe-1}
		{\draw (90-\compt/\LongueurListe*360:\RayonListe)--(90-\compt/\LongueurListe*360:\RayonListe+\EpaisseurListe);}
		\foreach \compt in {0,1,2}	{\draw (90-\compt/\LongueurListe*360-180/\LongueurListe:\RayonListe+\EpaisseurListe/2)node{\color{blue}\compt};}
		\foreach \compt in {3,4,5}	{\draw (90-\compt/\LongueurListe*360-180/\LongueurListe:\RayonListe+\EpaisseurListe/2)node{\color{blue}...};}
		\def \compt{6}	{\draw (90-\compt/\LongueurListe*360-180/\LongueurListe:\RayonListe+\EpaisseurListe/2)node{\color{blue} 126};}
		\def \compt{7}	{\draw (90-\compt/\LongueurListe*360-180/\LongueurListe:\RayonListe+\EpaisseurListe/2)node{\color{blue}  127};}
		\def \compt{8}	{\draw (90-\compt/\LongueurListe*360-180/\LongueurListe:\RayonListe+\EpaisseurListe/2)node{\color{blue}128};}
		\def \compt{9}	{\draw (90-\compt/\LongueurListe*360-180/\LongueurListe:\RayonListe+\EpaisseurListe/2)node{\color{blue} 129};}
		\foreach \compt in {10,11,12}	{\draw (90-\compt/\LongueurListe*360-180/\LongueurListe:\RayonListe+\EpaisseurListe/2)node{\color{blue}...};}
		\def \compt{13}	{\draw (90-\compt/\LongueurListe*360-180/\LongueurListe:\RayonListe+\EpaisseurListe/2)node{\color{blue} 253};}
		\def \compt{14}	{\draw (90-\compt/\LongueurListe*360-180/\LongueurListe:\RayonListe+\EpaisseurListe/2)node{\color{blue} 254};}
		\def \compt{15}	{\draw (90-\compt/\LongueurListe*360-180/\LongueurListe:\RayonListe+\EpaisseurListe/2)node{\color{blue} 255};}
	\end{tikzpicture}
\end{center}
 Ce schéma représente l'encodage des entiers relatifs en complément à 2 sur 8 bits (que l'on retrouve dans le type \tw{char} en C et C++).\\

	En bleu figurent les nombres tels qu'ils sont stockés dans la mémoire, sur un octet.\\
	En orange figurent les nombres représentés par les nombres bleus :
	\begin{enumerate}[--]
		\item 	lorsqu'il est compris entre 0 et 127, un nombre bleu représente ce même nombre (affiché en vert);
		\item 	lorsqu'il est compris entre 128 et 255, un nombre bleu $p$ représente le nombre $p-256$ (affiché en rouge).
	\end{enumerate}
	\begin{enumerate}[\bfseries a.]
		\item 	Donner la représentation en complément à 2 sur 8 bits du nombre +7.
		\item 	Faire de même avec -3.
		\item 	Ajouter en binaire ces deux représentations (une éventuelle retenue après le 8\eme bit est ignorée), quel nombre cela représente-t-il ?
	\end{enumerate}
	\item  Recommencer la question précédente et vérifier que l'égalité $-12+8=-4$ est vérifiée en complément à 2 sur 8 bits.\\
	\item Le programme suivant affiche -116. Expliquer ce résultat.
\begin{encadrecolore}{Code C++}{UGLiOrange}
	\begin{minted}{cpp}
#include <iostream> // nécessaire pour utiliser cout
int main() // début de la fonction main
{
    char c1 = 120; // on définit une première variable
    char c2 = 20; // puis une deuxième
    char c3 = c1+c2; // on les ajoute
    std::cout << (int) c3; // on affiche le résultat en base 10
    return 0; // la fonction main renvoie traditionnellement zéro
}
	\end{minted}

\end{encadrecolore}

\end{enumerate}

\exo{ - Masque jetable \tiny(métropole mai 2017)}

Le but de cet exercice est d'étudier une méthode de cryptage inventée par Gilbert Vernam en 1917, et appelée \og masque jetable \fg.

Dans tout l'exercice, on note respectivement $M$ le mot initial, $K$ la clé de cryptage et $Y$ le mot crypté.

Les trois nombres $M$, $K$, $Y$ sont des entiers naturels.

Les chiffres hexadécimaux sont notés 0, 1, 2, 3, 4, 5, 6, 7, 8, 9, A, B, C, D, E, F.

\medskip

\begin{enumerate}[\bfseries 1.]
	\item Questions préliminaires
	\begin{enumerate}[\bfseries a.]
		\item Donner la représentation en hexadécimal de l'entier binaire $1011101_2$.
		\item Calculer en travaillant dans le système hexadécimal les sommes $(7)_{16} + (4)_{16}$ et $(A)_{16} + (C)_{16}$.
	\end{enumerate}
	\item Soit $M$ et $K$ deux entiers naturels écrits en hexadécimal, tels que la longueur de l'écriture de $K$ est supérieure ou égale à celle de $M$, et tels que l'écriture de $K$ ne comporte aucun chiffre $0$.

	Pour crypter le mot $M$ avec la clé $K$, on procède comme suit : pour chaque  chiffre $m$ du mot
	initial $M$, on considère le chiffre $k$ de la clé $K$ qui a la même position que $m$ dans l'écriture.

	On obtient alors le chiffre $y$ du mot crypté $Y$ qui a la même position que $m$ dans l'écriture du mot initial $M$, de la façon suivante : $y$ est le chiffre hexadécimal des unités de la somme $m + k$.

	Le mot crypté $Y$ est déterminé en hexadécimal par la juxtaposition dans le même ordre des
	chiffres $y$ calculés pour chaque chiffre $m$ du mot $M$.

	\begin{center}
		\begin{tabularx}{\linewidth}{|X|}\hline
			\emph{Exemple} : avec $M =(49)_{16}$ et $K = (19)_{16}$
			\begin{itemize}
				\item Avec le chiffre de rang 1 en partant de la droite : $m = 9$ et $k = 9$ ; donc $m + k = (12)_{16}$ et par suite $y = 2$ ;
				\item avec le chiffre de rang 2 : $m = 4$ et $k = 1$ ; donc $m +k = (5)_{16}$ et par suite $y = 5$.
			\end{itemize}\\\hline
		\end{tabularx}
	\end{center}

	Donc le mot crypté est $Y = (52)_{16}$.

	\smallskip

	\textbf{Question }: avec le mot initial $M = (7 A)_{16}$ et la clé $K = (4C)_{16}$, déterminer le mot crypté $Y$.
	\item Par cette méthode, on admet que le décryptage suit les mêmes étapes en remplaçant la clé $K$ par une autre clé $K'$. Lorsque l'écriture de $K$ comporte au maximum deux chiffres hexadécimaux,
	la clé $K'$ est l'écriture en hexadécimal de la différence (écrite en décimal) $(272)_{10} - (K)_{10}$.

	\smallskip

	Cette question est une question à choix multiple. Une seule réponse est exacte. Recopier sur la
	copie seulement la réponse exacte. On ne demande pas de justification.

	\smallskip

	Avec la clé de cryptage $K = (19)_{16}$, la clé de décryptage $K'$ est égale à :

	\begin{center}
		\begin{tabularx}{0.75\linewidth}{X X}
			\textbf{Réponse A~~} : $(253)_{16}$	&\textbf{Réponse B~~} : $(247)_{16}$\\
			\textbf{Réponse C~~} : $(FD)_{16}$ 	&\textbf{Réponse D~~} : $(F7)_{16}$\\
		\end{tabularx}
	\end{center}
\end{enumerate}

\end{document}