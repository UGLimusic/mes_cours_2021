\documentclass[a4paper,12pt]{book}
\usepackage[margin=2cm]{geometry}		
\usepackage[thinfonts]{uglix2}
\usepackage{array,numprint}

\usepackage{systeme}

\setcounter{chapter}{4}


\begin{document}
\chapter*{Matrices}
\section{Notion de matrice}

\begin{definition}[ : matrice]
Une matrice $A$ peut être vue comme \og un tableau de nombres\fg{}.\\
Supposons qu'elle comporte n lignes et p colonnes (n et p sont des entiers plus grands que 1), on la note ainsi
$$A = \begin{pmatrix}
    a_{11}      & a_{12}&\cdots & a_{1p} \\ 
    a_{21}  & a_{22}& \cdots & a_{2p} \\      
    \vdots 	& \vdots & \ddots & \vdots \\ 
    a_{n1}  & a_{n2}    & \cdots & a_{np}
\end{pmatrix}$$
L'élément qui se situe à la i\eme ligne et à la j\eme colonne est noté $a_{ij}$. On l'appelle également \textit{coefficient}.\\

\textbf{Attention} : les indices des lignes et des colonnes commencent à 1 (et non à zéro comme dans la plupart des langages informatiques).\\

Pour résumer l'écriture précédente on écrit 
$$A=(a_{ij})_{\substack{1\leqslant i\leqslant n\\1\leqslant j\leqslant p}}$$

On dit aussi que $A$ est une matrice $n\times p$. Si $n=p$ on dit que $A$ est une \textit{matrice carrée d'ordre $n$}.
\end{definition}

\begin{exemple}[s]
\begin{enumerate}[--]
	\item 	$B =	\begin{pmatrix}
					    1      & 2 & 4 \\ 
					    -3  & 5 & 0
					\end{pmatrix}$ est une matrice à 2 lignes et 3 colonnes. On a $b_{21}=-3$.
	\item 	$C =	\begin{pmatrix}
						    2,4      & 7 & 4 & -1 \\ 
						    -3  & 5 & 10,1 & 1 \\
						    0,01 & 3 & 12 & 100
						\end{pmatrix}$ est une matrice à 3 lignes et 4 colonnes. On a $c_{33}=12$.
	\item 	$D =	\begin{pmatrix}
							    4      & 2\\ 
							    2  &8
							\end{pmatrix}$ est une matrice carrée d'ordre 2.
							
	
\end{enumerate}
\end{exemple}

\begin{exercice}[]
On considère 	$E =	\begin{pmatrix}
						    -4      & 7,6 & 4 & -1 & 12 \\ 
						    8 & -3  & 5,7 & 101 & 1 \\
						    12 & 0,01 & 3 & 12 & 1
						\end{pmatrix}$.\\
						
Donne les valeurs de $e_{12}$, $e_{21}$, $e_{35}$ et $e_{24}$.
\end{exercice}

Le script \textsc{Python} suivant permet de générer une matrice $n\times p$ avec des coefficients entiers aléatoires compris entre -100 et 100.


\pythonfile{mat_alea.py}

\begin{exercice}[]
\begin{enumerate}[\bfseries 1.]
	\item \'Ecris complètement la matrice suivante : $M=(m_{ij})_{\substack{1\leqslant i\leqslant 3\\1\leqslant j\leqslant 5}}$ où $m_{ij}=i$ si i=j et 0 sinon.
	\item 	\'Ecris complètement la matrice suivante : $M=(m_{ij})_{\substack{1\leqslant i\leqslant 4\\1\leqslant j\leqslant 5}}$ où $m_{ij}=0$ si i<j et 1 sinon.
	\item 	\'Ecris complètement la matrice suivante : $M=(m_{ij})_{\substack{1\leqslant i\leqslant 3\\1\leqslant j\leqslant 3}}$ où $m_{ij}=1$ si $i+j$ est pair et 0 sinon.
	\item \textsc{bonus} : écris des programmes \textsc{Python} qui génèrent ces matrices.
\end{enumerate}
\end{exercice}

\begin{definition}[s : Matrices nulles et identités]
\begin{enumerate}[--]
	\item 	Une matrice dont tous les coefficients sont nuls est dite \textit{nulle} (c'est \og un tableau de zéros\fg{});
	\item 	la matrice \textit{carrée d'ordre $n$} dont tous les éléments sont nuls sauf ceux de la \textit{diagonale} (c'est-à-dire ceux qui s'écrivent $a_{ii})$ qui valent 1 s'appelle \textit{la matrice identité d'ordre $n$} et se note $I_n$.
\end{enumerate}
\end{definition}

\begin{exemple}[]
La matrice identité $I_3$ est
$I_3= \begin{pmatrix}
1&0&0\\
0&1&0\\
0&0&1
\end{pmatrix}$.
\end{exemple}

\section{Opérations sur les matrices}

\begin{definition}[ : addition]
Soient $A$ et $B$ deux matrices $n\times p$, on note $A+B$ la matrice $n\times p$ obtenu en ajoutant les coefficients correspondants de $A$ et de $B$ :

$$\begin{pmatrix}
    a_{11}      & \cdots & a_{1p} \\ 
 
    \vdots 	& \ddots & \vdots \\ 
    a_{n1}      & \cdots & a_{np}
\end{pmatrix} +\begin{pmatrix}
    b_{11}      & \cdots & b_{1p} \\ 
   
    \vdots 	& \ddots & \vdots \\ 
    b_{n1}      & \cdots & b_{np}
\end{pmatrix}=
\begin{pmatrix}
    a_{11} +b_{11}     & \cdots & a_{1p}+b_{1p} \\ 
  
    \vdots 	& \ddots & \vdots \\ 
    a_{n1}+n_{n1}      & \cdots & a_{np}+b_{np}
\end{pmatrix}$$
\end{definition}

\begin{exemple}[]
Prenons $A = \begin{pmatrix}
				3&1\\
				4&7\\
				-2&8
\end{pmatrix}$ et $B = \begin{pmatrix}
				2&5\\
				1&-3\\
				-5&9
\end{pmatrix}$, alors $A+B=\begin{pmatrix}
5&6\\5&4\\-7&17
\end{pmatrix}$.\end{exemple}
\begin{remarque}[]
\textbf{Attention :} on ne peut ajouter deux matrices que si elles ont les mêmes dimensions (c'est-à-dire même nombre de lignes et même nombre de colonnes).
\end{remarque}
\begin{definition}[ : multiplication par un réel]
Soient $A$ une matrice $n\times p$ et $k$ un nombre réel, on note $kA$ la matrice $n\times p$ obtenue en multipliant chaque coefficient de $A$ par $k$ :

$$k\begin{pmatrix}
    a_{11}      & \cdots & a_{1p} \\   
    \vdots 	& \ddots & \vdots \\ 
    a_{n1}      & \cdots & a_{np}
\end{pmatrix}=\begin{pmatrix}
    k\times a_{11}      & \cdots & k\times a_{1p} \\      
    \vdots 	& \ddots & \vdots \\ 
   k\times a_{n1}      & \cdots &k\times a_{np}
\end{pmatrix}
$$
\end{definition}

\begin{exemple}[]
Prenons $A = \begin{pmatrix}
				8&3&1\\
				4&7&2\\
				-2&1&8
\end{pmatrix}$ et $k=5$, alors on obtient que $5A = \begin{pmatrix}
				40&15&5\\
				20&35&10\\
				-10&5&40
\end{pmatrix}$.\end{exemple}


La propriété suivante énonce quelques résultats utiles pour calculer.
\begin{propriete}[ : règles de calcul]
Soient $A$, $B$ et $C$ trois matrices de mêmes dimensions et $k$ et $k'$ 2 réels.
\begin{enumerate}[\textbullet]
	\item 	$A+B = B+A$
	\item 	$(A+B)+C = A+(B+C)$
	\item 	$k(A+B) = kA + kB$
	\item 	$(k+k')A = kA+k'A$
\end{enumerate}
\end{propriete}

\begin{exercice}[]
On pose $A=\begin{pmatrix}
1&2\\3&-4
\end{pmatrix}$, $B=\begin{pmatrix}
11&10\\-9&7
\end{pmatrix}$ et $C=\begin{pmatrix}
-3&-2\\5&-5
\end{pmatrix}$.\\

Montrer que $B-2A+3C$ est une matrice nulle.
\end{exercice}

\begin{definition}[ : multiplication de deux matrices]

Soient $A$ une matrice $n\times p$ et $B$ une matrice $p\times q$ (le nombre de colonnes de la 1\ere est égal au nombre de lignes de la 2\eme) alors il est possible de définir la matrice $C=A\times B$, produit de $A$ par $B$.\\
C'est une matrice $n\times q$ dont les coefficients sont ainsi :
\begin{center}
\begin{tabular}{cc}
&$\begin{pmatrix}
    b_{11}   &\cdots   &\color{orange@color}b_{1j}& \cdots & a_{1q} \\ 
    \vdots 	& \ddots &\color{orange@color}\cdots&\dots & \vdots \\ 
    b_{k1} & \cdots & \color{orange@color}b_{kj}&\cdots & b_{kq}\\
        \vdots 	& \cdots &\color{orange@color}\cdots&\ddots & \vdots \\ 
    b_{p1}    &\cdots  &\color{orange@color}b_{pj} & \cdots &b_{pq}
\end{pmatrix}$\\

$\begin{pmatrix}
    a_{11}   &\cdots   &\cdots& \cdots & a_{1p} \\ 
    \vdots 	& \ddots &\cdots&\dots & \vdots \\ 
\color{vertfonce@color}    a_{i1} & \color{vertfonce@color}\cdots &\color{vertfonce@color} a_{ik}&\color{vertfonce@color}\cdots & \color{vertfonce@color}a_{ip}\\
        \vdots 	& \cdots &\cdots&\ddots & \vdots \\ 
    a_{n1}    &\cdots  & \cdots & \cdots &a_{np}
\end{pmatrix}$&
$\begin{pmatrix}
    c_{11}   &\cdots   &\cdots& \cdots & c_{1q} \\ 
    \vdots 	& \ddots &\cdots&\dots & \vdots \\ 
   \vdots & \cdots & \color{red}c_{ij}&\cdots & \vdots\\
        \vdots 	& \cdots &\cdots&\ddots & \vdots \\ 
    c_{n1}    &\cdots  & \cdots & \cdots &c_{nq}
\end{pmatrix}$
\end{tabular}
\end{center}
{\LARGE$$ {\color{red}c_{ij}} = {\color{vertfonce@color}a_{i1}}\times {\color{orange@color}b_{1j}}+ {\color{vertfonce@color}a_{i2}}\times {\color{orange@color}b_{2j}}+\ldots+ {\color{vertfonce@color}a_{ip}}\times {\color{orange@color}b_{pj}}$$}
\end{definition}

\begin{exemple}[]

Prenons $A=\begin{pmatrix}
1&3 & -2 \\5 &-3&4
\end{pmatrix}$ et $B=\begin{pmatrix}
-5 & 1 & 0 & 2\\2& 1 & -1 & -8\\3 &4 & 0 & 9
\end{pmatrix}$ alors $A$ est une matrice $2\times 3$, $B$ est une matrice $3\times 4$ donc il est possible de définir la matrice $C=A\times B$, ce sera une matrice $2\times 4$.
\begin{center}
\begin{tabular}{cc}
 & $\begin{pmatrix}
 -5 & 1 & \color{orange@color}0 & 2\\2& 1 & \color{orange@color}-1 & -8\\3 &4 & \color{orange@color}0 & 9
 \end{pmatrix}$\\
$\begin{pmatrix}
1&3 & -2 \\\color{vertfonce@color}5 &\color{vertfonce@color}-3&\color{vertfonce@color}4
\end{pmatrix}$ & $\begin{pmatrix}
-5&-4&-3&-40\\-19&18&\color{red}3&70
\end{pmatrix}$
\end{tabular}
\end{center}
Par exemple, pour calculer $c_{33}$, on fait ${\color{vertfonce@color}5}\times {\color{orange@color}0}+ {\color{vertfonce@color}(-3)}\times {\color{orange@color}(-1)}+ {\color{vertfonce@color}4}\times {\color{orange@color}0}={\color{red}3} $.
\end{exemple}

\begin{remarque}[]
\textbf{Attention :} 
\begin{enumerate}[--]
	\item 	on ne peut multiplier $A$ par $B$ que si le nombre de colonnes de $A$ est égal au nombre de lignes de $B$;
	\item 	ce n'est pas parce qu'on peut calculer $A\times B$ qu'on peut calculer $B\times A$ : les matrices de l'exemple précédent ne permettent pas de calculer $B\times A$ car le nombre de colonnes de $B$ n'est pas égal au nombre de lignes de $A$:
	\newpage
	\begin{center}
	\begin{tabular}{cc}
	 & 	$\begin{pmatrix}
	 	1&3 & -2 \\\color{vertfonce@color}5 &\color{vertfonce@color}-3&\color{vertfonce@color}4
	 	\end{pmatrix}$ \\
	$\begin{pmatrix}
		 -5 & 1 & \color{orange@color}0 & 2\\2& 1 & \color{orange@color}-1 & -8\\3 &4 & \color{orange@color}0 & 9
		 \end{pmatrix}$& Impossible
	\end{tabular}
	\end{center}
	\item pour pouvoir calculer $A\times B$ et $B\times A$ il faut que ces deux matrices soient carrées d'ordre $n$ et \textit{en général} on n'a pas $A\times B=B\times A$.
\end{enumerate}
\end{remarque}

\begin{exercice}[]
\begin{enumerate}[--]
	\item 	On pose $A=\begin{pmatrix}
	1&2\\3&0
	\end{pmatrix}$ et $B=\begin{pmatrix}
	5&2\\0&3
	\end{pmatrix}$.\\
	
	Calculer $AB$ et $BA$.
	
	\item 	Recommencer avec $A=\begin{pmatrix}
		-4&6\\-3&5
		\end{pmatrix}$ et $B=\begin{pmatrix}
		8&-10\\5&-7
		\end{pmatrix}$.
\end{enumerate}
\end{exercice}

\begin{propriete}[s de calcul]

$A$, $B$ et $C$ sont des matrices.\\
Lorsque les opérations sont possibles (bonnes dimensions des matrices) on a :
\begin{enumerate}[\textbullet]
	\item 	$A(BC)=(AB)C$;
	\item 	$(A+B)C=AC+BC$;
	\item 	$A(B+C)=AB+AC$.\\
\end{enumerate}
Soit $k$ un nombre réel alors on a également $A\times kB=kAB$.\\

Si $A$ est \textit{carrée d'ordre} $n$ on a
\begin{enumerate}[\textbullet]
	\item 	$AI_n=I_nA=A$ où $I_n$ est la matrice identité d'ordre $n$.
	\item 	$A\times 0=0\times A=0$ en notant 0 la matrice carrée d'ordre $n$ nulle.
\end{enumerate}
\end{propriete}

\section{Exemple concret d'utilisation}

Imaginons une école qui forme des ingénieurs en informatique, avec seulement 3 matières.
Trois élèves de première année ont obtenu les résultats suivants :

\begin{center}
\textbf{Résultats pour le premier trimestre}\\[1em]

\begin{tabular}{|c|c|c|c|c|}
\hline
 & Maths & Physique & Info\\
\hline
Adam & 12 & 8 & 16\\
\hline
Bertrand & 18 & 14 & 12\\
\hline
Charles & 5 & 20 & 15\\
\hline
\end{tabular}\ \\[2em]

\textbf{Résultats pour le deuxième trimestre}\\[1em]

\begin{tabular}{|c|c|c|c|c|}
\hline
 & Maths & Physique & Info\\
\hline
Adam & 10 & 10 & 14\\
\hline
Bertrand & 18 & 12 & 14\\
\hline
Charles & 7 & 14 & 17\\
\hline
\end{tabular}
\end{center}

Ces deux tableaux peuvent s'écrire matriciellement 
$S_1= \begin{pmatrix}
12 & 8 & 16 \\
18 & 14 & 12 \\
5 & 20 & 15 \\
\end{pmatrix}$
et $S_2= \begin{pmatrix}
10 & 10 & 14 \\
18 & 12 & 14 \\
7 & 14 & 17 \\
\end{pmatrix}$\\

Pour calculer les moyennes mensuelles des élèves \og en une fois\fg{} on peut définir $M=0,5(S_1+S_2)$: 

$$M = 0,5 \times \left[\begin{pmatrix}
12 & 8 & 16 \\
18 & 14 & 12 \\
5 & 20 & 15 \\
\end{pmatrix}+\begin{pmatrix}
10 & 10 & 14 \\
18 & 12 & 14 \\
7 & 14 & 17 \\
\end{pmatrix}\right]$$

$$M = 0,5\times \begin{pmatrix}
22 & 18 & 30 \\
36 & 16 & 26 \\
12 & 34 & 32 \\
\end{pmatrix}$$

$$M = \begin{pmatrix}
11 & 9 & 15 \\
18 & 8 & 13 \\
6 & 17 & 16 \\
\end{pmatrix}$$

Le coefficient des mathématiques est 1, celui de la physique est 2 et celui de l'informatique est 5.\\
Pour passer en deuxième année, il faut un total de points supérieur ou égal à 120.

Pour faire \og d'un coup\fg{} le total des points, on peut considérer la matrice de coefficients $C=\begin{pmatrix}
1\\2\\5
\end{pmatrix}$.\\

Les points des élèves sont donnés par la matrice $$P=MC$$
$$P = \begin{pmatrix}
11 & 9 & 15 \\
18 & 8 & 13 \\
6 & 17 & 16 \\
\end{pmatrix}\begin{pmatrix}
1\\2\\5
\end{pmatrix}$$

$$P = \begin{pmatrix}
104\\109\\120\\
\end{pmatrix}$$

Ainsi seul Charles est admis à passer en 2\eme année.

\section{Matrices inversibles et systèmes}

\begin{definition}[ et propriété : Matrice inversible, inverse d'une matrice]
Soit $A$ une matrice \textbf{carrée d'ordre $n$}. S'il existe une matrice $B$ d'ordre $n$ telle que 
$$AB = I_n\qquad\text{ou}\qquad BA=I_n$$

alors automatiquement les deux égalités sont vérifiées, $B$ est nécessairement \textit{unique} et on dit alors que $B$ \textit{est l'inverse de $A$}. De manière symétrique $A$ est également l'inverse de $B$ si bien qu'on dit que $A$ et $B$ sont inverses l'une de l'autre.

On note ceci $A=B^{-1}$ ou, ce qui revient au même, $B=A^{-1}$.
\end{definition}

\begin{exemple}[]
$A=\begin{pmatrix}
-1&2\\-2&3
\end{pmatrix}$ et $B=\begin{pmatrix}
3&-2\\2&-1
\end{pmatrix}$ sont inverses l'une de l'autre :
\begin{center}
\begin{tabular}{cc}
& $\begin{pmatrix}
3&-2\\2&-1
\end{pmatrix}$\\
$\begin{pmatrix}
-1&2\\-2&3
\end{pmatrix}$
 & $\begin{pmatrix}
 1&0\\0&1
 \end{pmatrix}$
\end{tabular}
\end{center}
\end{exemple}
\begin{exercice}[]
Montrer que $A=\begin{pmatrix}
3 & -2&1\\
-2&2&-1\\
1&-1&1
\end{pmatrix}$ et 
$B=\begin{pmatrix}
1& 1&0\\
1&2&1\\
0&1&2
\end{pmatrix}$ sont inverses.

\end{exercice}
\begin{remarque}[]
Il existe des matrices non inversibles, par exemple $\begin{pmatrix}
1 & 2\\ 2 & 4 
\end{pmatrix}$.
\end{remarque}


\newcolumntype{C}{>{{}}c<{{}}} % for columns with binary operators
\renewcommand\vv{\multicolumn{1}{c}{\vdots}}
\begin{methode}[ : résoudre des systèmes avec des matrices]
On considère un \textit{système de $n$ équations à $n$ inconnues} :

$$\left\{
\setlength{\arraycolsep}{0pt}
\begin{array}{c<{x_1} C c<{x_2} C c C c<{x_n} C l}
a_{11} & + & a_{12} & + & \cdots & + & a_{1n} & = & b_1 \\
a_{21} & + & a_{22} & + & \cdots & + & a_{2n} & = & b_2 \\
\vv    &   & \vv    &   &        &   & \vv    &   & \vdots\\
a_{n1} & + & a_{n2} & + & \cdots & + & a_{nn} & = & b_n \\
\end{array}\right.$$

On connaît tous les nombres $a_{ij}$ et tous les $b_i$, et on veut trouver les valeurs des inconnues $x_i$.\\

Ce système peut se réécrire de manière matricielle :

$$\begin{pmatrix}
    a_{11}      & a_{12}&\cdots & a_{1n} \\ 
    a_{21}  & a_{22}& \cdots & a_{2n} \\      
    \vdots 	& \vdots & \ddots & \vdots \\ 
    a_{n1}  & a_{n2}    & \cdots & a_{nn}
\end{pmatrix}
\begin{pmatrix}
    x_{1}       \\ 
    x_2\\      
    \vdots \\ 
    x_n
\end{pmatrix}
=\begin{pmatrix}
    b_{1}       \\ 
    b_2\\      
    \vdots \\ 
    b_n
\end{pmatrix}$$

Ou encore :
$$AX=B$$

Si la matrice $A$ est inversible (en pratique ce sera toujours le cas parce qu'on nous donnera sa matrice inverse ou bien parce qu'on l'aura déterminée à l'aide de la calculatrice) alors, on peut reprendre l'égalité précédente et écrire : $A^{-1}AX=A^{-1}B$, ce qui donne $I_nX=A^{-1}B$. En définitive on a $$X = A^{-1}B$$

Ainsi pour trouver les valeurs des inconnues $x_i$, on effectue simplement le produit matriciel $A^{-1}B$ : chacune de ses lignes nous donne la valeur du $x_i$ correspondant.
\end{methode}

\begin{remarque}[]
Pour savoir comment utiliser la calculatrice, regarder ici :
\begin{enumerate}[--]
	\item 	modèles \textsc{CASIO} : \url{https://youtu.be/yjvQx13Vhlk}
	\item 	modèles \textsc{Texas Instrument} : \url{https://youtu.be/rxDxBnIwaGo}
\end{enumerate}
\end{remarque}

\begin{exemple}[]
On considère le système suivant : \systeme{2x+5y+2z=1,
5x-3y-2z=2,
-x+2y+z=-3}\\

Il peut se réécrire de manière matricielle :

$$\begin{pmatrix}
2 & 5 & 2 \\
5 & -3 & -2\\
-1 & 2 & 1
\end{pmatrix}
\begin{pmatrix}
x \\ y \\ z
\end{pmatrix}=
\begin{pmatrix}
1 \\ 2 \\ -3
\end{pmatrix}$$

Appelons $A$ la matrice carrée du membre de gauche. On détermine que $A$ est inversible avec la calculatrice et que son inverse est 
$$A^{-1}=\begin{pmatrix}
1 & -1 & -4 \\
-3 & 4 & 14\\
7 & -9 & -31
\end{pmatrix}$$
On a donc

$$\begin{pmatrix}
x \\ y \\ z
\end{pmatrix}=\begin{pmatrix}
1 & -1 & -4 \\
-3 & 4 & 14\\
7 & -9 & -31
\end{pmatrix}\begin{pmatrix}
1 \\ 2 \\ -3
\end{pmatrix}$$
C'est à dire, en effectuant le produit dans le membre de droite 

$$\begin{pmatrix}
x \\ y \\ z
\end{pmatrix}=\begin{pmatrix}
11\\-37\\82
\end{pmatrix}$$

On a donc résolu le système : \systeme{x=11,y=-37,z=82}
\end{exemple}
\begin{exercice}[]
\begin{enumerate}[\bfseries 1.]
	\item 	Effectue le produit suivant : $\begin{pmatrix}
	1 & 2 \\ -3 & 3\end{pmatrix}\begin{pmatrix}x\\y\end{pmatrix}$.
	\item 	\'A l'aide de la calculatrice détermine l'inverse de la matrice $\begin{pmatrix}
		1 & 2 \\ -3 & 3\end{pmatrix}$.
	\item Résous le système suivant : \systeme{x+2y = 15,-3x+3y = -6}
\end{enumerate}
\end{exercice}
\end{document}