\documentclass[a4paper,12pt,french]{article}
\usepackage[margin=2cm]{geometry}
\usepackage[thinfonts,latinmath]{uglix2}
\pagestyle{empty}
\nouveaustyle
\begin{document}
\titre{Format IEEE-754 32 bits}{NSI1}{10/2021}

On considère le nombre $x=-129,25$.\\

On veut donner sa représentation en virgule flottante suivant la norme IEEE-754 32 bits.\\

Commence par écrire 129,25 en base 2.\\

$129,25 =\dotfill$\\

Tu vas maintenant donner l'écriture scientifique en base 2 de $x$:\LARGE

	$$(-1)^{\underbrace{......}_s}\times \left(\underbrace{1,...............}_{1,m}\right)_2\times 2^{\underbrace{......}_{e}}$$\normalsize


Ensuite, $1,m$ doit être écrit avec 23 bits après la virgule en rajoutant des zéros inutiles :\\


$1,m=\dotfill$\\


Quand à $e$, on va le représenter par l'écriture binaire de $e+127$, sur 8 bits, que l'on va noter $b$ et que tu vas donner ici :\\

$b = \dotfill$\\

Finalement, en mémoire, $x$ est représenté ainsi :

\LARGE
$$\underbrace{......}_{s\text{ (1 bit)}}\underbrace{........................}_{b \text{ (8 bits)}}\ \underbrace{...................................................}_{m \text{ (23 bits)}}$$

\normalsize
Et en regroupant par paquets de 4 bits, tu peux facilement l'écrire sur 4 octets, en hexadécimal :\\

$x$ est représenté par $\left(........................\right)_{16}$
\end{document}
