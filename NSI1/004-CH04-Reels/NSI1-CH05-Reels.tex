\documentclass[a4paper,12pt,french]{book}
\usepackage[margin=2cm]{geometry}
\usepackage[thinfonts]{uglix2}

\begin{document}
	\setcounter{chapter}{4}

	\chapter{\large Représentation des données \\[-1em]\fontsize{35pt}{42pt}\selectfont Représentation approximative des réels}
	\introduction{Tout cela est-il bien réel ?}


\section{De la calculatrice à l'ordinateur}
Dans une machine on \textit{ne peut pas} représenter tous les nombres réels car la majorité a une écriture décimale illimitée et parmi celle-ci la majorité a une écriture décimale illimitée sans qu'aucun motif se répète, comme c'est le cas pour $\pi\simeq\np{3,141 592 653 589 793}$.\\
Ceci dit on peut donner une valeur approchée d'un nombre réel $r$ en écriture décimale en
utilisant l'\textit{écriture scientifique} vue au collège :

	$$\boxed{r\approx (-1)^s\times d\times 10^{e}}$$

\begin{itemize}
\item 	$s$ vaut 0 ou 1.
\item 	$d$ est un nombre décimal entre 1 inclus et 10 exclu.
\item 	$e$ est un entier relatif.
\end{itemize}

Ainsi, avec la calculatrice, on obtient :
$$\boxed{5,4^{-5}\approx (-1)^0\times 2,177866231\times10^{-4}}$$

Dans ce cas, le nombre $d$ comporte 10 chiffres décimaux, appelés \textbf{chiffres significatifs}.

Dans un ordinateur ne travaille pas avec des écritures décimales, mais avec des écritures \textbf{dyadiques} (l'équivalent des nombres décimaux en
base 2).
\begin{exemple}[ : nombre décimal / nombre dyadique]
\begin{enumerate}[\textbullet]
	\item 	Considérons le nombre $x = 53,14$ (écrit en base 10). C'est un \textbf{nombre décimal} et on peut écrire :
			\begin{tabbing}
				$x$	\=	$=53+1\times 0,1+4\times 0,04$\\
							\>	$=5\times 10^1+3\times 10^0+1\times 10^{-1}+4\times 10^{-2}$\\
			\end{tabbing}
	\item 	Les nombres dyadiques sont l'équivalent des nombres décimaux en base 2 : considérons $y=(101,011)_2	$, on peut écrire:
			\begin{tabbing}
				$y$	\=	$=1\times 2^2+0\times 2^1+1\times 2^0+0\times 2^{-1}+1\times 2^{-2}+1\times 2^{-3}$\\
							\>	$=4+1+0,25+0,125$\\
							\>	$=5,375$
			\end{tabbing}
\end{enumerate}
\end{exemple}

\begin{exemple}[ : écriture scientifique pour un nombre dyadique]
\begin{enumerate}[\textbullet]
	\item 	En reprenant l'exemple précédent on a $53,14=5,314\times 10^1$, c'est une écriture scientifique.
	\item 	Pour $(101,011)_2$, en se rappelant que multiplier par 2 décale la virgule d'un cran vers la gauche, on a $(101,011)_2=(1,01011)_2\times
	2^2$.
\end{enumerate}
\end{exemple}

Il faut donc décider d'un \textbf{format de représentation} des nombres dyadiques dans l'ordinateur :
\begin{enumerate}[\textbullet]
	\item 	Combien de bits significatifs pour le nombre dyadique ?
	\item 	Quelle est la plage de valeurs pour l'exposant ?
\end{enumerate}
\section{Le format IEEE 754}

Ce format est une norme pour représenter les nombres dyadiques (notamment par \textsc{Python}, en format 64 bits).\\
Par simplicité commençons par le format 32 bits (4 octets, donc).
Voici comment cela fonctionne :

\begin{definition}
On considère un mot de 32 bits.
\begin{itemize}
\item Le premier bit $s$, indique le signe.
\item Les 8 bits suivants  $e_7e_6\ldots e_0$ servent à coder l'exposant $e$ de 2, qui vaut $e=(e_7\ldots e_0)_2-127$.
\item Les 23 bits restants $m_1\ldots m_{23}$ servent à coder la \textbf{mantisse}, notons $m=(1,m_1\ldots m_{23})_2$.
\item En définitive, ce mot de 32 bits représente $$\boxed{(-1)^s\times m\times 2^e}$$
		Ce qu'on peut noter
		$$se_7\ldots e_0m_1\ldots m_{23}\mapsto (-1)^s\times(1,m_1\ldots m_{23})_2\times 2^{(e_7\ldots e_0)_2-127}$$

		$se_7\ldots e_0m_1\ldots m_{23}$ s'appelle une écriture \textbf{normalisée}.
\end{itemize}

En format 64 bits, les nombres sont représentés sur 8 octets : 1 bit de signe, 11 bits d'exposant (donc une plage de -1023 à 1024) et 52 bits de
mantisse. Le principe est le même qu'en 32 bits.
\end{definition}

\begin{exemple}[ : des 32 bits au nombre]
Que représente $1\ \underbrace{0100\ 0011}\ \underbrace{1110\ 0000\ 0000\ 0000\ 0000\ 000}$?
\begin{itemize}
\item $s = 1$.
\item $e =(0100\ 0011)_2-127=67-127=-60$
\item $m= (\boxed{1},1110 0000 0000 0000 0000 000)_2=1+2^{-1}+2^{-2}+2^{-3}=1,875$
\item $1\ 0100\ 0011\ 1110\ 0000\ 0000\ 0000\ 0000\ 000\rightsquigarrow (-1)^1\times 1,875\times 2^{-60}$
\end{itemize}
Cela fait environ $-1,6263032587282567\times 10^{-18}$
\end{exemple}



\section{Les limitations du format IEEE 754}


Lorsqu'un format de représentation en virgule flottante est choisi (32 ou 64 bits), on ne peut pas représenter tous les nombres réels : il y a un
plus grand nombre représentable (et son opposé pour les nombres négatifs) et un \og plus petit nombre positif représentable\fg{} (le plus proche de
zéro possible). De plus \textbf{ on a automatiquement des valeurs approchées si le nombre que l'on veut représenter n'est pas de de la forme
$\frac{a}{2^n}$, avec $a\in\Z$ et $n\in\N$.}\\

Par exemple, un nombre aussi simple que 0,1 (c'est-à-dire $\frac{1}{10}$) n'est pas de la forme $\frac{a}{2^n}$, avec $a\in\Z$ et $n\in\N$, donc son
écriture dyadique ne termine pas .\\
On peut montrer que :$$(0,1)_{10} = (0,0001\ 1001\ 1001\ 1001\ \ldots)_2$$
C'est l'équivalent dyadique de $$\dfrac{1}{3}=(0,3333\ldots)_{10}$$

Et puisque \textsc{Python} ne peut pas représenter intégralement le nombre 0,1 il l'approche du mieux qu'il peut : en fait pour \textsc{Python} la
valeur de 0,1 est :\\

\tw{0.1000000000000000055511151231257827021181583404541015625}\\

Mais celui-ci a la gentillesse d'afficher \tw{0.1}.\\

Il faut se résigner à accepter les erreurs d'arrondis :
\begin{pythonshell}
>>> 0.1 + 0.1 + 0.1
0.30000000000000004
\end{pythonshell}


Cet exemple prouve que \textbf{tester l'égalité de 2 \tw{float} n'a pas d'utilité}. On aura plutôt intérêt à \textbf{tester si leur différence est
très petite}.\\

De même, l'addition de plusieurs \tw{float} donne un résultat qui peut dépendre de l'ordre dans lequel on les ajoute.
Elle \textbf{n'est pas non plus associative} : \\

on peut avoir \tw{a + (b + c) $\neq$ (a + b) + c}.\\


Les erreurs d'arrondis se cumulent. Pour les minimiser on aura intérêt à appliquer la règle suivante :

\begin{propriete}[ : Règle de la photo de classe]
Dans une somme de \tw{float}, l'erreur est minimisée quand on commence par ajouter les termes de plus petite valeur absolue.
\end{propriete}


\exostart

\begin{exercice}[]
	Donner l'écriture décimale des nombres suivants\begin{enumerate}[\bfseries a.]
		\item 	$(101,1)_2$
		\item 	$(1,011)_2$
		\item 	$(0,1111\ 111)_2$ en remarquant que c'est \og $(111\ 1111)_2$ divisé par $2^7$.
	\end{enumerate}
\end{exercice}
\begin{exercice}[]
	\'Ecrire en base 2 les nombres suivants :
	\begin{enumerate}[\bfseries a.]
		\item 	3,5
		\item 	7,75
		\item 	27,625
	\end{enumerate}
\end{exercice}

\begin{exercice}[]
On considère le format IEEE 754 64 bits : 1 bit de signe, 11 bits d'exposant (donc une plage de -1023 à 1024) et 52 bits de
mantisse.
\begin{enumerate}[\bfseries 1.]
	\item 	Quel est le plus grand nombre positif représentable ?
	\item 	Quel est le plus petit nombre positif représentable ?
\end{enumerate}
\end{exercice}

\begin{exercice}
\'Ecrire un programme déterminant le plus petit entier $n$ pour lequel \textsc{Python} considère que $1+2^{-n}=1$.\\
Faire le lien avec le format IEEE 754 64 bits.
\end{exercice}

\begin{exercice}
Faire calculer \tw{1+2**(-53)-1} puis \tw{1-1+2**(-53)}.\\
Que remarque-t-on ?\\
Comment, au regard de l'exercice 1, expliquer ce résultat ?
\end{exercice}



\begin{exercice}
\'Ecrire un programme déterminant le plus petit entier $n$ pour lequel \textsc{Python} considère que $2^{-n}=0$.\\
En faisant le lien avec le format IEEE 754 64 bits, quelle valeur devrait-on trouver ?\\
Quelle explication peut-on imaginer (sachant qu'en \textsc{Python}, les \tw{float} sont bien codés sur 64 bits) ?
\end{exercice}

\begin{exercice}
On peut prouver (c'est dur) que $$\sum_{n=1}^{+\infty}\dfrac{1}{n^4}=\dfrac{\pi^4}{90}$$

Appelons $c$ cette constante.

\textbf{1.} Calculer à l'aide d'un script $\displaystyle S=\sum_{n=1}^{10^6}\dfrac{1}{n^4}$ .\\

Ajoute-t-on des termes de plus en plus petits ou de plus en plus grands ?\\
Continuer le script pour afficher $c-S$ (on pourra utiliser \tw{from math import pi}).\\

\textbf{2.} Calculer S \og dans l'autre sens \fg.\\
Afficher $c-S$.\\

\textbf{3.} Qu'illustre cet exercice ?
\end{exercice}


\begin{exercice}

\textbf{1.}	Montrer qu'un triangle (3,4,5) est rectangle, ainsi qu'un triangle ($\sqrt{11}$, $\sqrt{12}$, $\sqrt{23}$).\\

\textbf{2.} \'Ecrire une fonction \tw{est\_rectangle(a,b,c)}: qui renvoie \tw{True} si \tw{c**2 == a**2 + b**2} et, sinon, qui renvoie la différence
entre \tw{c**2} et \tw{a**2+b**2}.\\

\textbf{3.} Tester la fonction \tw{est\_rectangle()} avec les deux triangles précédents.\\
			Que remarque-t-on ? Comment modifier la fonction pour qu'elle soit plus satisfaisante ?
\end{exercice}

\begin{exercice}[**]
	On aimerait trouver l'écriture dyadique (illimitée) de $\dfrac{1}{3}$.
	On note donc $$\dfrac{1}{3}=(0,a_1a_2a_3\ldots)_2$$
	où $a_i$ vaut 1 ou 0.
	\begin{enumerate}[\bfseries 1.]
		\item 	Expliquer pourquoi $a_1$ vaut \textit{nécessairement 0}.
		\item 	On note $x=\frac{1}{3}$. Montrer que $x$ vérifie $4x=1+x$.
		 \item 	Quelle est l'écriture dyadique de $4x$ ?
		 \item 	Quelle est celle de $1+x$ ?
		 \item 	En écrivant que ces 2 écritures représentent le même nombre, en déduire que $$\dfrac{1}{3}=(0,0101\ 0101\ldots )_2$$
	\end{enumerate}
\end{exercice}

\end{document}