\documentclass[a4paper,12pt,french]{article}
\usepackage[margin=2cm]{geometry}
\usepackage[thinfonts,latinmath]{uglix2}
\pagestyle{empty}
\nouveaustyle
\begin{document}
\section*{À propos des devoirs à rendre en NSI}


Bonjour à tous les parents d'élèves,\\

Ce document expose les règles concernant les devoirs à faire à la maison, dont certains donneront lieu à une note certificative (c'est-à-dire directement prise en compte pour le baccalauréat) : la spécialité NSI est une matière nouvelle et son lien étroit avec le monde numérique et Internet la rend particulière à ce sujet.\\

Il ne traite que des devoirs qui demandent de fournir un programme, à remettre dans l'espace \textsc{Pronote} dédié.\\

Je vous remercie pour le temps que vous prendrez à le lire et vous prie d'accuser réception en le signant.


\subsection*{Internet : un outil à double tranchant}

Les élèves ont le droit de se documenter et sur Internet on trouve parfois des programmes « tout prêts » qui semblent répondre au problème à résoudre... Ce n'est pas toujours le cas : parfois ils ne marchent simplement pas, dans d'autres cas ils font appel à des concepts de programmation beaucoup trop avancés pour des élèves de première.
\begin{enumerate}[--]
	\item 	les regarder en détail, les « faire tourner » pour en comprendre le fonctionnement est permis;
	\item 	les recopier tels quels ou en changeant quelques lignes sans avoir vraiment compris relève de la fraude.
\end{enumerate}
 Il n'est pas normal qu'un programme \textit{réellement personnel ou réellement approprié} ne puisse être expliqué quelques jours plus tard.
En cas de doute, j'inviterai l'élève à décrire le fonctionnement de son programme ligne par ligne pour montrer qu'il est bien le fruit de son travail.

\subsection*{Travail personnel ou collectif ?}

Lors des projets, les élèves travaillent en équipe et partagent leur code. Les remarques du paragraphe précédent s'appliquent encore.
Un devoir à la maison à rendre est en revanche personnel et à ce titre

\begin{enumerate}[--]
	\item 	il est permis aux élèves de discuter entre eux, de se donner quelques conseils;
	\item 	montrer son code est interdit;
	\item   en cas de blocage, je suis disponible pour aider les élèves à distance en dehors des heures de cours.
\end{enumerate}
Là encore, on repère facilement le plagiat, même lorsque quelques lignes ont été ajoutées, quelques variables renommées \textit{et c\ae tera}. Il est plus simple, plus rapide et plus sûr de me demander de l'aide : cela n'entraîne aucune pénalité sur la note finale.\\


J'aurai plaisir à vous recevoir pour parler des résultats de votre enfant lors de la réunion parents-professeur de Première.\\


Bien cordialement,

\flushright Frédéric Leleu \\
Professeur de spécialité NSI\\

\includegraphics[width=2cm]{signature}\\
\flushleft
Signature :
\end{document}
