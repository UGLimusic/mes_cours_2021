\documentclass[a4paper,12pt]{article}
\usepackage[margin=2cm,top=1cm,bottom=1cm]{geometry}		
\usepackage[thinfonts,latinmath]{uglix2}
\pagestyle{empty}

\begin{document}
\titreinterro{Interrogation : logique}{NSI1}{2022}

En utilisant les tables de vérités, montrer que, les expressions booléennes suivantes sont équivalentes :
$(non\, A)\, et\, (non\, B)$ d'une part, et d'autre part $non\, (A\, ou\, B)$.\\

\carreauxseyes{16.8}{4.8}\\

En utilisant les tables de vérités, montrer que, les expressions booléennes suivantes sont équivalentes :
$A$ d'une part, et d'autre part $(A\,et\,B)\,ou\,(A\,et\,(non\,B))$.\\


\carreauxseyes{16.8}{4.8}\\

Représenter ci-dessous $(A\,ou\,B)\,et\,(A\,ou\,(non\,B))$ à l'aide d'un circuit logique (pas de symboles européens sur les portes, juste les noms des opérations).\\

\begin{tikzpicture}
\draw(0,0) rectangle(17,6);
\end{tikzpicture}
\end{document}
