\documentclass[a4paper,12pt]{article}
\usepackage[margin=2cm,top=1cm,bottom=1cm]{geometry}		
\usepackage[thinfonts,latinmath]{uglix2}
\pagestyle{empty}

\begin{document}
	\titreinterro{Interrogation\bis : Logique}{NSI1}{2022}
	
	En utilisant les tables de vérités, montrer que, les expressions booléennes suivantes sont équivalentes :
	$(non\, A)\, ou\, (non\, B)$ d'une part, et d'autre part $non\, (A\, et\, B)$.\\
	
	\carreauxseyes{16.8}{4.8}\\
	
	En utilisant les tables de vérités, montrer que, les expressions booléennes suivantes sont équivalentes:
	$A$ d'une part, et d'autre part $(A\,ou\,B)\,et\,(A\,ou\,(non\,B))$.\\
	
	
	\carreauxseyes{16.8}{4.8}\\
	
	Représenter ci-dessous $(A\,et\,B)\,ou\,(A\,et\,(non\,B))$ à l'aide d'un circuit logique (pas de symboles européens sur les portes, juste les noms des opérations).\\
	
	\begin{tikzpicture}
	\draw(0,0) rectangle(17,6);
	\end{tikzpicture}
\end{document}
