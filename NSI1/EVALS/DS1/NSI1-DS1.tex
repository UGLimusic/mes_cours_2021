\documentclass[a4paper,10pt,french]{article}
\usepackage[margin=2cm,top=1cm,bottom=1.5cm]{geometry}
\usepackage[thinfonts]{uglix2}
\pagestyle{plain}
\nouveaustyle
\makeatletter
\def\FV@DefineFindStop{%
	\ifx\FancyVerbStopString\relax
	\ifnum\FancyVerbStopNum<\@ne
	\let\FV@FindStartStop\FV@@PreProcessLine
	\else
	\let\FV@FindStartStop\FV@FindStopNum
	\fi
	\else
	\let\FV@FindStartStop\FV@FindStopString
	\fi}
\makeatother
\begin{document}
\titreinterro{Contrôle}{NSI1}{26/11/2021}

\exo{ - Bases, représentation des nombres}\\


Quelle est l'écriture en base 2 de 45 ? Détailler les calculs.\\

\carreauxseyes{16.8}{4}\\

Que vaut $(1111\,1011)_2$ en base 10 ? Détailler les calculs.\\

\carreauxseyes{16.8}{4}\\

Que vaut $(1111\,1011)_2$ en base 10 ? Détailler les calculs.\\

\carreauxseyes{16.8}{4}\\

Comment représente-t-on -7 en complément à 2 sur 8 bits ? Détailler les calculs.\\

\carreauxseyes{16.8}{4} \\
\newpage
Que vaut $(110,011)_2$ en base 10 ? Détailler les calculs.\\

\carreauxseyes{16.8}{4} \\

\'Ecrire 15,75 en base 2. Détailler les calculs.\\

\carreauxseyes{16.8}{4} \\

Explique \textbf{sommairement} pourquoi en \textsc{Python} \pythoninline{0.1 + 0.1 + 0.1} ne vaut pas \pythoninline{0.3}.\\

\carreauxseyes{16.8}{4} \\


\exo{ - Représentation du texte}\\


Comment s'appelle le premier format historique d'encodage du texte ?\\

\carreauxseyes{16.8}{1.6} \\

Explique ce qu'est le test par bit de parité sur un octet et donne deux exemples.\\

\carreauxseyes{16.8}{4.8}\\

\carreauxseyes{16.8}{4}\\

Quel est l'avantage de l'encodage UTF-8 par rapport aux autres encodages tels que l'ISO ou Windows-1252 vus en cours ?\\

\carreauxseyes{16.8}{4} \\


\exo{ - Méthodes de programmation}\\

Donne un exemple de valeur de chaque type que nous avons rencontré en \textsc{Python}.\\

\carreauxseyes{16.8}{3.2} \\

Quels sont les 2 types de boucles et quelle est leur différence principale ? \\

\carreauxseyes{16.8}{4} \\

Dans le code suivant, une ligne est inutile. Laquelle et pourquoi ?

\begin{encadrecolore}{Code Python}{python@color}
\begin{minted}[linenos]{python}
for i in range(100):
    if i % 3 == 0 :
        print(i)
    i = i + 1
\end{minted}
\end{encadrecolore}

\carreauxseyes{16.8}{4} \\

Explique en une phrase ce qu'affiche le programme précédent.\\

\carreauxseyes{16.8}{4} \\


\newpage 

\exo{ - \'Ecriture de programmes}

Écrire un programme qui 
\begin{enumerate}[--]
	\item 	demande d'abord à l'utilisateur de choisir un nombre entier compris entre 1 et 10;
	\item 	choisit un nombre au hasard entre 1 et 10 et l'affiche;
	\item 	recommence tant que ce n'est pas le nombre choisi par l'utilisateur;
	\item 	affiche le nombre d'essais nécessaires pour trouver le nombre de l'utilisateur
\end{enumerate}

\begin{exemple}[ ]
\begin{verbatim}
Choisis un nombre entre 1 et 10 : 7
2
4
2
9
7
Il m'a fallu 5 tirages aléatoires pour retrouver ton 7.
\end{verbatim}	
\end{exemple}

\carreauxseyes{16.8}{16.8} \newpage
\'Ecrire un programme qui
\begin{enumerate}[--]
	\item 	demande à l'utilisateur combien d'ami\cdot  e \cdot s il a;
	\item 	demande d'entrer le prénom de chacun\cdot e;
	\item 	affiche une phase du type : « Vos amis sont ... ».
\end{enumerate}

\begin{exemple}[ ]
	\begin{verbatim}
Combien d'ami(e)s as-tu ? : 4
Entre un prénom : Elouan
Entre un prénom : Mehdi
Entre un prénom : Paul
Entre un prénom : Kilian
Tes amis sont Elouan, Mehdi, Paul, Kilian.
\end{verbatim}	
\end{exemple}

\carreauxseyes{16.8}{16.8}
\end{document}
