\documentclass[a4paper,12pt]{article}
\usepackage[margin=2cm,top=1cm]{geometry}
\usepackage[thinfonts]{uglix2}
\pagestyle{empty}
\setlength{\columnseprule}{0.5pt}
\setminted{fontsize=\small}
\begin{document}
\titreinterro{Interrogation 03}{NSI1}{09/2021}

\begin{multicols}{2}
\begin{pythoncode}
n = 2
while n < 13:
    n += 3
print(n)
        \end{pythoncode}
%
    \begin{enumerate}[\case\ \ a.]
        \item 2
        \item 11
        \item 14
    \end{enumerate}


\begin{pythoncode}
n = 7
while n > 7:
    n = n + 4
print(n)
\end{pythoncode}
Ce script affiche
    \begin{enumerate}[\case\ \ a.]
        \item 7
        \item 8
        \item 11
    \end{enumerate}

\begin{pythoncode}
n = 0
for i in range(10):
    n += 2
print(n)
        \end{pythoncode}
    \columnbreak

    \begin{enumerate}[\case\ \ a.]
        \item 10
        \item 20
        \item 18
    \end{enumerate}

\begin{pythoncode}
s = "bonjour"
c = 0
for e in s:
    c += 1
print(c)
        \end{pythoncode}
Ce script affiche
    \begin{enumerate}[\case\ \ a.]
        \item Il génère une erreur
        \item 6
        \item 3
    \end{enumerate}

\begin{pythoncode}
n_e = 0
c = "la matiere revee"
for l in c:
    if l == "e":
            n_e += 1
print(ne)
          \end{pythoncode}

    \begin{enumerate}[\case\ \ a.]
        \item 3
        \item 5
        \item Il génère une erreur
    \end{enumerate}
\newpage
\begin{pythoncode}
c = "la vache mache"
r = ""
for l in c:
    if l == "a":
        r = r + "o"
    else:
        r = r + l
print(r)
          \end{pythoncode}
    Qu'affiche ce script ?\\

    \carreauxseyes{7.2}{2.4}



\begin{pythoncode}
c = "C'est dur la vie."
r = ""
i = 0
while i < len(c) and c[i] != "u":
    r = r + c[i]
    i += 1
print(r)
        \end{pythoncode}
    Qu'affiche ce script ?\\

\carreauxseyes{7.2}{2.4}
\end{multicols}

Quelles doivent être les valeurs de \pythoninline{a}, \pythoninline{b} et \pythoninline{c} pour que la plage de valeurs \pythoninline{range(a, b, c)} soit\\

\pythoninline{0, 1, 2, 3, 4, 5} ? \\

\carreauxseyes{16.8}{1.6}\\

\pythoninline{2, 5, 8, 11, 14,  ... , 998} ? \\

\carreauxseyes{16.8}{1.6}\\

\pythoninline{100, 98, 96, ... , 2, 0} ? \\

\carreauxseyes{16.8}{1.6}\\

\end{document}

