\documentclass[a4paper,10pt]{article}
\usepackage[margin=2cm,top=1cm,bottom = 1cm]{geometry}		
\usepackage{uglix}

%\nocolumnrule
\begin{document}
\titreinterro{QCM 01\tiny (rattrapage)}{NSI1}{09/2020}

Qu'affiche chacun des scripts suivants ?
\begin{multicols}{2}
\begin{enumerate}[\bfseries 1.]
\item 

\begin{python}
a = 10 
b = a - 4 
c = a - b
b = c + a
print(b)
\end{python}
\begin{enumerate}[\case\ \ a.]
\item \tw{3}
\item \tw{14}
\item \tw{4}
\end{enumerate}	

\item 	
\begin{python}
resultat = 22 // 7 + 1
print(resultat)
\end{python}
\begin{enumerate}[\case\ \ a.]
\item \tw{4}
\item \tw{2}
\item Il génère une erreur
\end{enumerate}	

\item 
\begin{python}
variable = ['a', 1]
print(type(variable))
\end{python}
\begin{enumerate}[\case\ \ a.]
\item 	\tw{<class 'list'>}
\item 	\tw{<class 'str'>}
\item 	\tw {<class 'int'>}

\end{enumerate}	

\item 
\begin{python}
a = 'ga'
b = 'bu'
c = a + b  + a * 3
print(c)
\end{python}
\begin{enumerate}[\case\ \ a.]
\item 	\tw{gagabububu}
\item 	\tw{gabugagaga}
\item 	Il génère une erreur
\end{enumerate}	

\item 
\begin{python}
variable = "(7-3)*2"
print(variable)
\end{python}
\begin{enumerate}[\case\ \ a.]
\item 	\tw{8}
\item 	\tw{1}
\item 	\tw{((7-3)*2}
\end{enumerate}	

\item 
\begin{python}
a = '20'
b = '3'
print(a+b)
\end{python}
\begin{enumerate}[\case\ \ a.]
\item 	\tw{203}
\item 	Il génère une erreur
\item 	\tw{23}
\end{enumerate}	

\item 
\begin{python}
a = 2
b = 3
print(a + B)
\end{python}
\begin{enumerate}[\case\ \ a.]
\item 	\tw{23}
\item 	Il génère une erreur
\item 	\tw{5}
\end{enumerate}	

\item 
\begin{python}
prenom = 'Coucou'
print(prenom[2:5])
\end{python}
\begin{enumerate}[\case\ \ a.]
\item 	\tw{ucou}
\item 	\tw{oucou}
\item 	\tw{uco}
\end{enumerate}	

\item 
\begin{python}
note = [12,13,16]
print(note[13])
\end{python}
\begin{enumerate}[\case\ \ a.]
\item 	\tw{1}
\item 	\tw{2}
\item 	Il génère une erreur
\end{enumerate}

\item 
\begin{python}
capitale = {'France':'Paris', 'Espagne':'Madrid', 'Italie':'Rome'}
print(capitale['Italie'])
\end{python}
\begin{enumerate}[\case\ \ a.]
\item 	\tw{Rome}
\item 	\tw{Paris}
\item 	Il génère une erreur
\end{enumerate}

\end{enumerate}	
\end{multicols}
\end{document}

