\documentclass[a4paper,10pt]{article}
\usepackage[margin=2cm]{geometry}
\usepackage[thinfonts]{uglix2}
\pagestyle{empty}
\setlength{\columnseprule}{.5pt}
\begin{document}
\titreinterro{QCM 01\bis}{NSI1}{09/2021}

Qu'affiche chacun des scripts suivants ? Une seule réponse est valide.
\begin{multicols}{2}

\begin{pythoncode}
a = 2
b = a - 7
c = a + b
b = c - a
print(b)
\end{pythoncode}
\begin{enumerate}[\case\ \ a.]
\item \tw{-5}
\item \tw{2}
\item \tw{3}

\end{enumerate}

\begin{pythoncode}
x = 2
y = 3 * x - 4
z = (x + y ) // 2
t = (x + y + 3 * z) // 5
print(x + y + z - 3 * t)
\end{pythoncode}

\begin{enumerate}[\case\ \ a.]
\item 	\texttt{-10}
\item 	\texttt{2}
\item 	\texttt{0}
\end{enumerate}

\begin{pythoncode}
age = input('Entre ton âge : ')
# l'utilisateur entre 16 au clavier
print(f'Dans 10 ans tu auras {age +10} ans.')
\end{pythoncode}

\begin{enumerate}[\case\ \ a.]
\item 	\tw{Dans 10 ans tu auras 26 ans.}
\item 	Il génère une erreur.
\item 	\tw{Dans 10 ans tu auras {age +10} ans.}
\end{enumerate}


\begin{pythoncode}
resultat = 1 - 5 % 4
print(resultat)
\end{pythoncode}

\begin{enumerate}[\case\ \ a.]
\item \tw{0}
\item Il génère une erreur
\item \tw{-1}
\end{enumerate}


\begin{pythoncode}
prenom = 'Jean Raoul'
print(prenom[3:7])
\end{pythoncode}
\begin{enumerate}[\case\ \ a.]
\item 	\tw{en R}
\item 	\tw{n Ra}
\item 	\tw{en Ra}
\end{enumerate}

\begin{pythoncode}
a = '2'
b = '3'
print(a+b)
\end{pythoncode}
\begin{enumerate}[\case\ \ a.]
\item 	\tw{5}
\item 	\tw{23}
\item 	Il génère une erreur
\end{enumerate}
\newpage
\begin{pythoncode}
a = 'mer'
b = 'ci'
c = b + a * 2 + b
print(c)
\end{pythoncode}
\begin{enumerate}[\case\ \ a.]
\item 	\tw{mercimerci}
\item 	Il génère une erreur
\item 	\tw{cimermerci}
\end{enumerate}


\begin{pythoncode}
variable = "(40-3)*20"
print(variable)
\end{pythoncode}
\begin{enumerate}[\case\ \ a.]
\item 	\tw{-20}
\item 	\tw{(40-3)*20}
\item 	\tw{740}
\end{enumerate}


\begin{pythoncode}
a = 4
b = a + 3
a = 5
print(b)
\end{pythoncode}
\begin{enumerate}[\case\ \ a.]
\item \tw{5}
\item 	\tw{7}
\item \tw{8}
\end{enumerate}


\begin{pythoncode}
variable ='12'
print(type(variable))
\end{pythoncode}
\begin{enumerate}[\case\ \ a.]
\item 	\tw{<class 'int'>}
\item 	\tw{<class 'str'>}
\item 	\tw {<class 'float'>}
\end{enumerate}

\end{multicols}
\end{document}

