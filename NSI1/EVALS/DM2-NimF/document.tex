\documentclass[a4paper,12pt,french]{article}
\usepackage[margin=2cm]{geometry}
\usepackage[thinfonts,latinmath]{uglix2}
\nouveaustyle
\begin{document}
\titre{Devoir facultatif }{NSI1}{11/2021}


\textit{Ce travail \textbf{personnel} est à faire pendant les vacances et doit être rendu sur \textsc{pronote} pour \textbf{le jeudi 8 décembre au plus tard}. }

\textit{Il est facultatif : seule la meilleure des deux notes comptera.}

\textit{Aucun délai ne sera accepté et entraînera la nullité de la note.}

\textit{L'unique programme à coder devra être rendu dans un fichier \texttt{.py} ou \texttt{.txt}, les formats \textsc{Word}, \textsc{OpenOffice} ou autres ne seront pas acceptés, les captures d'écran sont interdites.}



\subsection*{Variante du Jeu de Nim}

Deux joueurs A et B jouent à tour de rôle. A commence. Au début de la partie il y a 25 jetons.\\
Le gagnant sera celui qui prendra le dernier jeton (c'est donc différent du jeu de Nim).
\begin{enumerate}[--]
	\item 	Au premier tour, A ne peut pas prendre tous les jetons, sinon il gagnerait immédiatement.\\
			Il doit en prendre un au minimum et 24 au maximum.
	\item 	Ensuite, alternativement, chaque joueur doit prendre au moins un jeton, et au plus le double du nombre de
			jetons que l'autre joueur a pris au tour d'avant.
\end{enumerate}

\subsection*{Exemples de partie}

Partie rapide :
\begin{enumerate}[--]
	\item 	A prend 10 jetons, il en reste 15.
	\item 	B peut prendre entre 1 et 20 jetons (double de 10) : il en prend 15, il a gagné ! \\
\end{enumerate}

Partie moins rapide :
\begin{enumerate}[--]
	\item 	A prend 5 jetons, il en reste 20.
	\item 	B peut en prendre entre 1 et 10, il en prend 6, il en reste 14.
	\item 	A peut en prendre entre 1 et 12, il en prend 4, il en reste 10.
	\item 	B peut en prendre entre 1 et 8, il en prend 3, il en reste 7.
	\item 	A peut en prendre entre 1 et 6, il en prend 2, il en reste 5.
	\item 	B peut en prendre entre 1 et 4, il en prend 1, il en reste 3.
	\item 	A peut en prendre entre 1 et 2, il en prend 1, il en reste 2.
	\item 	B peut en prendre entre 1 et 2, il en prend 2, il a gagné !
\end{enumerate}

\subsection*{Le programme à rendre}

Tu devras écrire un programme dans lequel :
\begin{enumerate}[--]
	\item 	A est l'ordinateur et commence;
	\item 	on fait jouer l'humain grâce à des \pythoninline{input};
	\item 	à chaque fois qu'un joueur termine son tour, on affiche combien de jetons il a pris, combien il en reste 
	et combien le prochain peut en prendre;
	\item 	on affiche le gagnant à la fin.\\
\end{enumerate}
La stratégie de l'ordinateur est de jouer au hasard (sauf pour ceux qui ont des idées de stratégies gagnantes).


Pour rappel, voici le corrigé du devoir précédent, vous pouvez vous en inspirer :\\

\inputminted[fontsize=\scriptsize]{python}{prog1.py}

\end{document}
