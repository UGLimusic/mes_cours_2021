\documentclass[a4paper,12pt]{article}
\usepackage[margin=2cm,top=1cm,bottom=1cm]{geometry}			
\usepackage{uglix}
\nouveaustyle
\begin{document}
\titre{Le jeu de Nim (bonus)}{BTS SIO}{12/2020}	

\begin{enumerate}[\bfseries 1.]
	\item 	Donner les valeurs de N pour lesquelles A est sûr de gagner en un coup (c'est-à-dire laisser un seul jeton à B).
	\item 	Donner les valeurs de N pour lesquelles A est sûr de gagner en deux coups (en se ramenant à la question précédente)
	\item 	En extrapolant, quel est l'ensemble des valeurs de N pour lequel A est sûr de gagner ? (on ne demande pas de preuve)
	\item 	Quand A est sûr de gagner, quelle est sa stratégie ?
	\item 	En déduire une fonction (en langage naturel ou en \textsc{Python}) \tw{A\_joue\_bien} qui implémente la stratégie précédente (et si A n'est pas sûr de gagner, faire au mieux).
\end{enumerate}

\end{document}