\documentclass[a4paper,12pt]{article}
\usepackage[margin=2cm,top=1cm]{geometry}
\usepackage[thinfonts]{uglix2}
\pagestyle{empty}
\setlength{\columnseprule}{0.5pt}
\setminted{fontsize=\small}
\begin{document}
    \titreinterro{Interrogation 05}{NSI1}{2022}

    \'Ecris en compréhension les listes suivantes.\\

    Liste des carrés des entiers compris entre 1 et 1000 (inclus).\\

    \carreauxseyes{16.8}{1.6}\\
    \pythoninline{lst} est une liste d 'entiers donnée.\\
    Liste des éléments de \pythoninline{lst} supérieurs à 10. \\

    \carreauxseyes{16.8}{1.6}\\

    Liste des \textit{indices} des éléments de \pythoninline{lst} supérieurs à 10.\\

    \carreauxseyes{16.8}{1.6}\\

    Liste des nombres impairs de \pythoninline{lst}, avec \pythoninline{0} à la place des nombres pairs.
    Par exemple si \pythoninline{lst} vaut \pythoninline{[1, 2, 5, 0, 3, 7, 10]} on veut que l'écriture en compréhension produise \pythoninline{[1, 0, 5, 0, 3, 7, 0]}.\\

    \carreauxseyes{16.8}{1.6}\\

    Liste suivante : \pythoninline{[[0, 0, 0], [0, 0, 0], [0, 0, 0], [0, 0, 0], [0, 0, 0]]}\\

    \carreauxseyes{16.8}{2.4}\\
    \vfill\ \hfill\Large ... / ...
    \newpage\normalsize
    Liste suivante :\\ \pythoninline{[[0, 0, 0, 0], [1, 0, 0, 0], [1, 1, 0, 0], [1, 1, 1, 0], [1, 1, 1, 1]]}\\

    \carreauxseyes{16.8}{2.4}\\
\end{document}

