\documentclass[a4paper,12pt,french]{book}
\usepackage[margin=2cm]{geometry}
\usepackage[thinfonts]{uglix2}

\setminted{fontsize=\small}\nouveaustyle
\begin{document}
\titre{Exercices de programmation}{NSI1}{10/2021} 

\begin{exercice}
On considère une liste \texttt{L}.\\
\'Ecrire un programme qui enlève tous les 2 de la liste.\\

Par exemple, si la valeur de \texttt{L} avant le programme est \begin{center}
\texttt{[1, 1, 2, 2, 2, 4, 3, 2, 2, 1]}
\end{center} alors après l'exécution du programme \texttt{L} devra valoir \begin{center}
\texttt{[1, 1, 4, 3, 1]}
\end{center}
\end{exercice}

\begin{exercice}
\'Ecrire une fonction \texttt{enleve\_2} qui \begin{enumerate}[--]
	\item en entrée prend une liste;
    \item en sortie renvoie une liste, composée des éléments de la liste d'entrée, dans le même ordre mais sans les 2.
\end{enumerate}
Par exemple si \texttt{L} vaut \texttt{[1, 2, 2, 3, 2]} alors \texttt{enleve\_2(L)} vaut \texttt{[1, 3]} et le fait d'avoir appelé la fonction n'a pas modifié \texttt{L}.
\end{exercice}

\begin{exercice}
\'Ecrire une fonction \texttt{renverse} qui \begin{enumerate}[--]
	\item en entrée prend une liste;
    \item en sortie renvoie une liste, composée des éléments de la liste d'entrée mais dans l'ordre inverse.
\end{enumerate}
Par exemple si \texttt{L} vaut \texttt{[1, 2, 3, 4]} alors \texttt{renverse(L)} vaut \texttt{[4, 3, 2, 1]} et le fait d'avoir appelé la fonction n'a pas modifié \texttt{L}.\\

\textbf{Attention :} on n'a pas le droit d'utiliser \pythoninline{L.reverse()}.
\end{exercice}

\begin{exercice}
\'Ecrire une fonction \texttt{renverse\_en\_place} qui
\begin{enumerate}[--]
	\item en entrée prend une liste;
    \item ne renvoie rien mais renverse les éléments de la liste.
\end{enumerate}
Par exemple si \texttt{L} vaut \texttt{[3, 4, 5]}, alors après l'appel de \texttt{renverse\_en\_place(L)}, \texttt{L} vaut \texttt{[5,4,3]}.\\

\textbf{Attention :} on n'a pas le droit de créer une deuxième liste dans la fonction.
\end{exercice}

\begin{exercice}
Compléter
\begin{minted}[breaklines]{python}
def doublon(L : list) -> bool :
    """ Renvoie True si une valeur de la liste apparaît au moins deux fois et False si chaque valeur est unique """
\end{minted}
\end{exercice}

\end{document}