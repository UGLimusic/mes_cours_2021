\documentclass[a4paper,11pt,french]{book}
\usepackage[margin=2cm]{geometry}
\usepackage{uglix}
\usepackage{colortbl}
\pagestyle{empty}
\nouveaustyle
\begin{document}
	\titre{Identifiants}{SIO}{12/2020}\\
	
\subsection{Principe}

Une entreprise crée un identifiant pour chacun des employés de ses différentes succursales de Paris, Lyon, Marseille, Toulouse et Bordeaux. L’identifiant comporte 8 caractères :\\ 

\double
{
	\begin{enumerate}[--]
		\item 	le premier caractère est l’initiale du nom de l’employé;
		\item 	le second caractère est '1' pour un homme et '2' pour une femme;
		\item 	les troisième et quatrième caractères sont pour son année de naissance;
		\item 	le cinquième caractère est le numéro de commune donné par le tableau ci-contre;
		\item 	les trois derniers caractères sont pour le numéro d’enregistrement de l’employé.\\
		
	\end{enumerate}
}
{
\begin{tabular}{|c|c|}
	\hline
	Paris & 1\\
	\hline
	Lyon & 2\\
	\hline
	Marseille & 3 \\
	\hline
	Toulouse & 4 \\
	\hline
	Bordeaux & 5 \\
	\hline
\end{tabular}
}
{3cm}
Par exemple une femme appelée Mme AZERTY née en 1979 qui travaille à Toulouse et qui est la 37ème à s’enregistrer aura
l’identifiant A2794036 car en décomposant on a 
$$\underbrace{A}_{initiale}\underbrace{2}_{sexe}\underbrace{79}_{année}\underbrace{4}_{ville}\underbrace{36}_{enregistrement}$$
Le numéro d'enregistrement est 36 et non 37 car le premier enregistré se voit attribuer le numéro 000 et ainsi de suite.

\subsection{Travail sur table}

Le tableau est représenté par une liste de listes : \\
\pythoninline{T = [['Paris',1],['Lyon',2],['Marseille',3],['Toulouse',4],['Bordeaux',5]]}

Pour les conversions on utilisera les fonctions suivantes :
\begin{enumerate}[--]
	\item 	\tw{entier} : prend une chaine de caractères et la convertit (si c'est possible) en entier;
	\item 	\tw{chaine} : prend un entier et le convertit en chaine de caractères.
\end{enumerate}

\begin{enumerate}[\bfseries 1.]
	\item 	On considère la fonction \tw{mystere} suivante qui prend en entrée une commune du tableau ci-dessus de type chaîne de caractères :
\begin{pseudocode}\begin{lstlisting}[language=naturel]
fonction mystere(x : chaine de caractères):
variables
	i entier
début
	i (<* $\leftarrow$ *>) 0
	tant que T[i][0](<* $\neq$ *>) x faire
		i (<* $\leftarrow$ *>) i + 1
	fin tant que
	renvoyer i + 1
fin
\end{lstlisting}\end{pseudocode}
\begin{enumerate}[\bfseries a.]
	\item 	Décrire le rôle de cet algorithme dans le contexte de l'énoncé et donner un nom plus parlant à la fonction.
			Quel est le type de la variable renvoyée par la fonction ?
	\item 	Modifier cette fonction pour que le type de la variable renvoyée soit une chaîne de caractère.	
\end{enumerate}
	\item \'Ecrire une fonction \tw{conversion} qui prend en entrée un entier compris entre 1 et 999 et qui renvoie la chaîne de 3 caractères formée par les chiffres de cet entier éventuellement complétés par des 0 à gauche.\\
	Par exemple \tw{conversion(37)} renvoie \pythoninline{'37'}.
	\item    L’entreprise compte 530 salariés et dispose d’une liste appelé \tw{listing} qui contient sur chaque ligne les données de ses employés de type chaîne de caractères au format suivant :\\
	\pythoninline{['NOM', 'SEXE', 'NAISSANCE', 'COMMUNE']}.\\
	La liste est donnée dans l'ordre d'inscription des employés.\\
	Par exemple la ligne correspondant à Mme AZERTY est la ligne d'indice 36 et est :\\
	\pythoninline{['AZERTY', 'femme', '1979', 'Toulouse']}.\\	
	Écrire une fonction \tw{identifiants} qui prend en entrée une liste telle que \tw{listing} et qui y rajoute une colonne avec l’identifiant de chaque employé.
	\item 	Pour détecter une erreur lors de la saisie d'un identifiant, on rajoute un neuvième caractère appelé clé, déterminé comme ceci :
	\begin{enumerate}[--]
		\item 	On calcule S, la somme de tous les chiffres de l'identifiant;
		\item 	On calcule le reste de la division euclidienne de 66-S par 11. Si c'est 10 alors la clé vaut \pythoninline{'X'}, sinon on convertit ce chiffre en chaine de caractères.
	\end{enumerate}
		Par exemple avec l'identifiant \pythoninline{'A2794036'} on $S= 2 + 7 + 9 + 4 + 0 + 3 + 6=31$.\\
	Ensuite $66-31=35$ et $35=3\times 11 + 2$ donc la clé est \pythoninline{'2'} et finalement l'identifiant avec clé est \pythoninline{'A27940362'}.\\
	
	\'Ecrire une fonction \tw{ajout\_cle} qui prend en entier un identifiant à 8 caractères et renvoie l'identifiant avec la clé.
\end{enumerate}

\subsection{Implémentation sur machine}
	Implémenter les fonctions précédentes en \textsc{Python}.\\
	
	\textbf{\textsc{Bonus} :} \'Ecrire une fonction \tw{mise\_a\_jour} qui prend en entrée un listing de clients avec identifiants, regarde si leurs identifiants ont une clé, si c'est le cas vérifie qu'il n'y a pas d'erreur et s'il n'y a pas de clé, la crée.
\end{document}