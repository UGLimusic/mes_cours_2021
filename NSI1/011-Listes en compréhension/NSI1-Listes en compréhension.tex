\documentclass[10pt]{beamer}
\usepackage{uglixbeamer}
\usepackage[utf8]{inputenc}
\usepackage[T1]{fontenc}
\usepackage{blindtext}
\title{Listes en compréhension}
\author{NSI1}

\begin{document}

    \maketitle


    \section{Principe}
    \begin{frame}{Jusqu'à présent}
        \pause
        Pour construire des listes on fait souvent ceci : \pause
        \begin{enumerate}[\textbullet]
            \item    on crée une liste \pythoninline{lst} vide ;\pause
            \item    on construit une boucle \pythoninline{for} ou \pythoninline{while} ;\pause
            \item    on peuple la liste avec \pythoninline{lst.append} .
        \end{enumerate}
    \end{frame}

    \begin{frame}[fragile]{Exemple}
        \pause
        \begin{minted}{python}
lst = []
for i in range(10):
    lst.append(i*i)
print(lst)
        \end{minted}
        \pause
        Ce programme affiche : \\\pause

        \pythoninline{[0, 1, 4, 9, 16, 25, 36, 49, 64, 81]}\\\pause

        C'est la liste des carrés des 10 premiers entiers naturels.
    \end{frame}
    \begin{frame}{L'écriture en compréhension}
        \pause

        En mathématiques, l'ensemble des carrés des 10 premiers entiers naturels se note \pause

        \[\left\lbrace i^2\ |\ i\in\N,\,i<10\right\rbrace\]\pause
        C'est une écriture en \textit{compréhension}.\\\pause
        On peut faire la même chose en \textsc{Python} : \\\pause

        \pythoninline{lst = [i*i for i in range(10)]}\\\pause

        \'Evidemment, c'est plus rapide que la méthode précédente...\pause \\ Et on peut faire bien plus !
    \end{frame}

    \begin{frame}[fragile]{Des applications}
        \pause
        On peut utiliser une liste pour en construire une autre :\pause
        \begin{minted}{python}
lst1 = [2, -1, 3, 4, 7]
lst2 = [x + 1 for x in lst1]
print(lst2)
        \end{minted}
        \pause
        Ce programme affiche : \\\pause

        \pythoninline{[3, 0, 4, 5, 8]}
    \end{frame}

    \begin{frame}[fragile]{Des applications}
        \pause
        Dans le même esprit :\pause
        \begin{minted}{python}
lst1 = ['2', '0', '13']
lst2 = [int(x) for x in lst1]
print(lst2)
        \end{minted}
        \pause
        Ce programme affiche : \\\pause

        \pythoninline{[2, 0, 13]}
    \end{frame}

    \begin{frame}[fragile]{Des applications}
        \pause
        Ou encore :\pause
        \begin{minted}{python}
lst1 = ['Fred', 'Titouan', 'Tinaïg']
lst2 = [prenom[0] for prenom in lst1]
print(lst2)
        \end{minted}
        \pause
        Ce programme affiche : \\\pause

        \pythoninline{['F', 'T', 'T']}
    \end{frame}


    \section{\'Ecriture en compréhension avec conditions}

    \begin{frame}[fragile]{Encore mieux}
        \pause
        Il est possible d'utiliser \pythoninline{if} en compréhension :\pause
        \begin{minted}{python}
lst1 = [8, 0, 11, 10, 3, 15]
lst2 = [2 * x for x in lst1 if x > 10]
print(lst2)
        \end{minted}
        \pause

        Ce programme affiche :\\\pause
        \pythoninline{[22, 30]}\\\pause
        On met dans \pythoninline{lst2} le double de chaque élément de \pythoninline{lst1} qui est supérieur à 10 (dans l'ordre de parcours).
    \end{frame}

    \begin{frame}[fragile]{Encore plus fort}
        \pause

        Il est possible d'utiliser \pythoninline{if ... else ...} en compréhension, mais à ce moment là il faut écrire les conditions au début :\pause


        \begin{minted}{python}
lst1 = [8, -10, 11, -4, -3, 15]
lst2 = [(x if x > 0 else 0) for x in lst1]
# parenthèses facultatives
print(lst2)
        \end{minted}
        \pause

        Ce programme affiche :\\\pause
        \pythoninline{[8, 0, 11, 0, 0, 15]}\\\pause
        On crée une nouvelle liste en remplaçant tous les nombres négatifs de \pythoninline{lst1} par zéro.
    \end{frame}

    \begin{frame}[fragile]{Un dernier pour la route}
        \pause

        Que fait le programme suivant ?\pause

        \begin{minted}{python}
lst1 = [8, -10, 11, -4, -3, 15]
lst2 = [i for i in range(len(lst1)) if lst1[i] > 0]
print(lst2)
        \end{minted}
        \pause

        Ce programme affiche :\\\pause
        \pythoninline{[0, 2, 5]}\\\pause
        On crée une liste contenant les indices des éléments de \pythoninline{lst1} qui sont strictement positifs.
    \end{frame}


    \section{Les écritures en compréhension imbriquées}


    \begin{frame}{C'est un peu dur au départ}
        \pause
        $$\begin{matrix}
              0 & 0 & 0 & 0 \\0 & 0 & 0 & 0 \\0 & 0 & 0 & 0
        \end{matrix}$$\pause
        Si on veut représenter ce « tableau de nombres » par une liste, on peut écrire\\ \pause

        \pythoninline{lst = [[0, 0, 0, 0], [0, 0, 0, 0], [0, 0, 0, 0]]}\\pause

        (3 lignes de chacune 4 colonnes)\\

        Mais c'est plus pratique d'écrire\\\pause

        \pythoninline{lst=[[0 for j in range(4)] for i in range(3)]}
    \end{frame}


    \section{Pour conclure}

    \begin{frame}{Super cocktail}
        \pause

        On peut combiner toutes les techniques que nous venons de voir...\\\pause

        Par exemple on peut créer une liste de listes de listes avec des conditions, \textit{et c\ae tera}.\\\pause

        La seule limite, c'est l'imagination et la capacité à écrire en \textsc{Python} !
    \end{frame}
\end{document}