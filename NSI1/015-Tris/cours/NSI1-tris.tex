\documentclass[10pt]{beamer}
\usepackage{uglixbeamer}
\usepackage{animate}
\title{Tris}
\subtitle{Algorithmique}
\author{NSI1}
\metroset{block=fill}

\begin{document}
    \maketitle
    

\begin{frame}{Introduction}
Dans ce document nous allons
\begin{enumerate}[--]
	\item étudier quelques tris simples ;
    \item déterminer leur complexité.
\end{enumerate}
\end{frame}
\section{Tri par sélection}
\begin{frame}{Illustration}
        \begin{center}
       \animategraphics[step,width=6cm]{0.2}{img/sel/sel}{0}{18}
    \end{center}
\end{frame}
\begin{frame}{Principe}
On dispose d'une liste \pythoninline{lst}.\\
On trouve le plus petit élément de \pythoninline{lst}.\\
On l'échange avec le premier élément de \pythoninline{lst}.\\
Le premier élément de \pythoninline{lst} est en place et ne bougera plus, on recommence avec les éléments restants (du deuxième au dernier).\\
Quand on a bien placé l'avant dernier élément, on s'arrête car tous les éléments sont rangés.
\end{frame}
\begin{frame}[fragile]{Algorithme}
On peut l'écrire ainsi :
\begin{verbatim}
fonction tri_selection(lst)
    n = longueur(lst)
    pour i allant de 0 à n - 2
        p = indice_minimum(lst,i)
        échanger lst[p] et lst[i]
\end{verbatim}
Où \pythoninline{indice_minimum} est une fonction qui 
\begin{enumerate}[--]
	\item en entrée prend une liste et un entier qui est un indice de cette liste ;
    \item renvoie l'indice du plus petit élément de la liste situé à partir de l'indice i.
\end{enumerate}
\end{frame}
\begin{frame}[fragile]{Fonction \texttt{indice\_minimum}}
\begin{verbatim}
fonction indice_minimum(lst,i)
    n = longueur(lst)
    mini = lst[i]
    i_mini = i
    pour j allant de i à n - 1
        si lst[j] < mini
            mini = lst[j]
            i_mini = j
    renvoyer i_mini
\end{verbatim}
\end{frame}
\begin{frame}[fragile]{Complexité}
On décide que les \textsc{opel} ne sont que les accès en lecture / écriture aux éléments de la liste.\\
On considère la complexité dans le pire des cas.
\begin{verbatim}
fonction indice_minimum(lst,i)
    n = longueur(lst)               
    mini = lst[i]                   # 1 opel
    i_mini = i                      
    pour j allant de i à n - 1      
        si lst[j] < mini            # 1 opel
            mini = lst[j]           # 1 opel
            i_mini = j              
    renvoyer i_mini
\end{verbatim}
Puisque la boucle est appelée n - 1 - i fois, la complexité de cette fonction est $2(n-1-i)+1$
\end{frame}
\begin{frame}[fragile]{Complexité}
\small
\begin{verbatim}
fonction tri_selection(lst)
    n = longueur(lst)               
    pour i allant de 0 à n - 2      
        p = indice_minimum(lst,i)   # 2(n-1-i)+1 opels
        échanger lst[p] et lst[i]   # 2 opels
\end{verbatim}
\normalsize
Ce qui nous donne une complexité de $$\sum_{0}^{n-2} 2(n-1-i) +3$$
C'est-à-dire $$ \sum_{0}^{n-2} 2(n-i) + 1$$
\end{frame}

\begin{frame}{Calcul à ne pas noter}
\begin{tabbing}
$\sum_{0}^{n-2} 2(n-i) + 1$ \= $=2n(n-1)-2\left(\sum_{0}^{n-2}i\right) + (n-1)$\\
\\
\> $= 2n(n-1)-(n-2)(n-1)+(n-1)$\\
\\
\> $= (2n-n+2+1)(n-1)$\\
\\
\> $= (n+3)(n-1)$\\
\\
\> $= n^2+2n-3$\\
\end{tabbing}
Donc quand $n$ est grand, seul $n^2$ prévaut et on dira que la complexité du tri par sélection est en $n^2$.
\end{frame}
\section{Tri par insertion}


\begin{frame}{Principe}
On considère une liste (d'entiers ou d'éléments de tout type triable)  \pythoninline{lst}.

\begin{enumerate}[--]
    \item 	la liste composée de \pythoninline{lst[0]} est (évidemment) triée ;
    \item 	on regarde si \pythoninline{lst[1]} est plus petit que \pythoninline{lst[0]}, si c'est le cas échange ces deux éléments ;
    \item 	on peut donc dire que la liste composée des élements \pythoninline{lst[0]} et \pythoninline{lst[1]} est triée.
    \item 	on continue ainsi : à chaque étape, la liste composée des i premiers éléments de \pythoninline{lst} est déjà triée, on va alors \og insérer \fg{} \pythoninline{lst[i]} dans cette sous-liste, et ainsi on aboutira à une liste des i+1 éléments triée ;
    \item 	concrètement pour insérer \pythoninline{lst[i]} au bon endroit, on le compare avec l'élément précédent : tant qu'il est plus petit on l'échange avec celui-ci.
\end{enumerate}
\end{frame}

\begin{frame}{Algorithme}
On peut déjà produire la fonction \pythoninline{place} qui
\begin{enumerate}[--]
    \item 	prend en entrée une liste \pythoninline{lst} et un entier \pythoninline{i} avec les \textit{préconditions} suivantes
            \begin{enumerate}[\textbullet]
                \item i est inférieur à la longueur de la liste ;
                \item  les i premiers éléments de \pythoninline{lst} sont déjà triés dans l'ordre croissant ;
            \end{enumerate}
    \item	ne renvoie rien mais place le i\eme élément à la bonne place pour que les i+1 premiers éléments de \pythoninline{lst} soient triés.
\end{enumerate}
\end{frame}
\begin{frame}[fragile]{Algorithme de la fonction \texttt{place}}
\begin{verbatim}
fonction place(lst : liste, i : entier):
    tant que i > 0 et que lst[i-1] > lst[i]:
        echange lst[i] et lst[i-1]
        i = i - 1
\end{verbatim}
\end{frame}
\begin{frame}[fragile]{Algorithme de tri par insertion}
\begin{verbatim}
fonction tri_insertion( lst : liste): 
    n = longueur(lst)
    pour i allant de 1 à n-1:
        place(lst,i)
\end{verbatim}
\end{frame}

\begin{frame}{Complexité}
Tout comme le tri par sélection, on peut montrer que la complexité temporelle pire cas du tri par insertion est $2n^2-2n$, et donc de l'ordre de $n^2$.
\end{frame}

\section{Tri à bulles}
\begin{frame}{Principe}
On dispose toujours d'une liste \pythoninline{lst} d'éléments que l'on peut comparer.\\
\begin{enumerate}[--]
	\item on parcourt la liste du premier élément à l'avant-dernier : si cet élément est plus grand que le suivant, on les échange ;
    \item tant qu'on a fait au moins un échange, on recommence.
\end{enumerate}
\end{frame}
\begin{frame}{Algorithme}
Il est à trouver par toi-même.
\end{frame}
\begin{frame}{Complexité}
Elle est encore de l'ordre de $n^2$, (presque proportionnelle à $n^2$ quand $n$ est grand) mais on peut baisser le \og facteur de proportionnalité\fg{} en optimisant l'algorithme (voir activités).
\end{frame}
\end{document}


    


