\documentclass[a4paper,12pt,french]{book}
\usepackage[margin=2cm]{geometry}
\usepackage[thinfonts]{uglix2}
\begin{document}
\setcounter{chapter}{2}

\chapter{\large Représentation des données\\[-1em]\fontsize{35pt}{42pt}\selectfont\titlefont Représentation des entiers}
\introduction{Pour tout comprendre, lire ce chapitre en entier.}

Nous avons vu au chapitre précédent comment écrire les entiers naturels en binaire ou en hexadécimal. Maintenant nous allons
étudier comment les entiers \textit{relatifs}, c'est-à-dire positifs ou négatifs (on dit aussi \textit{signés} en Informatique) sont
représentés en machine.\\
On va d'abord se limiter aux entiers naturels et on va voir qu'il n'y a pas de difficulté majeure à comprendre leur représentation en machine.


\begin{remarque}[ importante]
Il existe des \textit{centaines} de langages de programmation. Les principaux sont : \textsc{C},  \textsc{C++}, \textsc{C\#}, \textsc{Java},
\textsc{Python}, \textsc{PHP} et \textsc{Javascript} (cette liste est non exhaustive). Chaque langage utilise ses propres \textit{types de variable}
mais en général il y a beaucoup de ressemblances. On travaillera donc sur des exemples de types de variables utilisés dans tel ou tel langage\ldots
sachant que ce type n'existe pas nécessairement en \textsc{Python}.
\end{remarque}

\section{Représentation des entiers naturels :\\ l'exemple du unsigned char}

En Informatique on dit souvent qu'un entier naturel est \textit{non signé} (ou \textit{unsigned} en Anglais). Le type \textit{unsigned char} se
rencontre en \textsc{C} et en \textsc{C++} (entre autres).\\

Un \textit{unsigned char} est stocké sur un octet, c'est à dire 8 bits :
\begin{enumerate}[\textbullet]
	\item 	l'octet 0000 0000 représente l'entier 0;
	\item 	0000 0001 représente 1;
	\item 	et ainsi de suite jusqu'au plus grand entier représentable sur un octet : \\$(1111\ 1111)_2=255$.
\end{enumerate}

On peut donc représenter les 256 premiers entiers avec un \textit{unsigned char} et c'est logique : un octet, c'est 8 bits, chaque bit peut prendre 2
valeurs et $2^8=256$.

Si on a besoin de représenter des entiers plus grands, on pourra utiliser l'\textit{unsigned short} : c'est la même chose mais ce type est représenté
sur 2 octets. Donc on peut représenter les $2^{16}$ premiers entiers naturels, c'est à dire les nombres compris entre 0 et 65 535 inclus.\\

Cela continue avec l'\textit{unsigned int} (sur 4 octets) et l'\textit{unsigned long} (8 octets).


\begin{remarque}[]
En  \textsc{Python} c'est différent : le type \texttt{int} (abréviation de \textit{integer}, qui veut dire \og entier \fg{} en Anglais) permet de
représenter des entiers arbitrairement grands, les seules limitations
étant la mémoire de la machine. Il n'y a qu'à évaluer $2^{100000}$  dans un \textit{shell} \textsc{Python} pour s'en convaincre.
\end{remarque}



\section{Représentation des entiers relatifs :\\l'exemple du type char}

Le type \textit{char} (qui n'existe pas en \textsc{Python}) utilise un octet et l'on veut représenter des
entiers relatifs (donc plus seulement positifs).

\subsection{Une première idée... qui n'est pas si bonne}

On pourrait décider que le bit de poids fort est un bit de signe : 0 pour les positifs et 1 pour les négatifs, par
exemple. Les 7 autres bits serviraient à représenter la valeur absolue du nombre. Puisqu'avec 7 bits on peut aller
jusqu'à $(111\ 1111)_2=127$, ce format permettrait de représenter tous les nombres entiers de -127 à 127.\\

Par exemple 1000 0011 représenterait -3 et 0001 1011 représenterait 27.\\

Il est un peu dommage que zéro ait 2 représentations : 0000 0000 et 1000 0000, mais ce qui est encore plus dommage
c'est que lorsqu'on ajoute les représentations de -3 et de 27 voici ce qui se passe:
\begin{center}

\begin{tabular}{cccccccccc}
 &  & 1 & 0 & 0 & 0 & 0 & 0 & 1 & 1 \\

+ &  & 0 & 0 & 0 & 1 & 1 & 0 & 1 & 1 \\
\hline
= &  & 1 & 0 & 0 & 1 & 1 & 1 & 1 & 0 \\
\end{tabular}

\end{center}

On obtient 10001 1110, qui représente -30... On aurait bien sûr préféré que cela nous donne 24.

\subsection{La bonne idée : le complément à deux}

Ce format, qui est utilisé avec le type \textit{char}, permet de représenter les entiers de -128 à 127 :

\begin{propriete} \textbf{Représentation en complément à 2 sur un octet :}
\begin{enumerate}[\textbullet]
	\item Soit $x$ un entier positif plus petit ou égal à 127, alors on représente $x$ par son écriture binaire (qui
	comprend 7 bits) et donc le bit de poids fort de l'octet (le 8\eme) est égal à zéro.
	\item Sinon si $x$ est un entier strictement négatif plus grand que -128, on le représente par l'écriture binaire
	de $256+x$, qui est toujours représenté par un octet avec un bit de poids fort égal à 1.
\end{enumerate}
\end{propriete}

\begin{center}
	\begin{tikzpicture}[scale=.5]
		\def\RayonListe{6.5}
		\def\EpaisseurListe{1.5}
		\def\LongueurListe{16}
		\draw[fill = UGLiOrange!15] (0,0) circle(\RayonListe+\EpaisseurListe);
		\draw[fill = white] (0,0) circle(\RayonListe);

		\foreach \compt in {0,1,...,\numexpr\LongueurListe-1}
		{\draw (90-\compt/\LongueurListe*360:\RayonListe)--(90-\compt/\LongueurListe*360:\RayonListe+\EpaisseurListe);}

		\foreach \compt in {0,1,2}	{\draw (90-\compt/\LongueurListe*360-180/\LongueurListe:\RayonListe+\EpaisseurListe/2)node{\color{UGLiGreen}\compt};}
		\foreach \compt in {3,4,5}	{\draw (90-\compt/\LongueurListe*360-180/\LongueurListe:\RayonListe+\EpaisseurListe/2)node{\color{UGLiGreen}...};}
		\def \compt{6}	{\draw (90-\compt/\LongueurListe*360-180/\LongueurListe:\RayonListe+\EpaisseurListe/2)node{\color{UGLiGreen} 126};}
		\def \compt{7}	{\draw (90-\compt/\LongueurListe*360-180/\LongueurListe:\RayonListe+\EpaisseurListe/2)node{ \color{UGLiGreen} 127};}
		\def \compt{8}	{\draw (90-\compt/\LongueurListe*360-180/\LongueurListe:\RayonListe+\EpaisseurListe/2)node{\color{UGLiRed} -128};}
		\def \compt{9}	{\draw (90-\compt/\LongueurListe*360-180/\LongueurListe:\RayonListe+\EpaisseurListe/2)node{ \color{UGLiRed} -127};}

		\foreach \compt in {10,11,12}	{\draw (90-\compt/\LongueurListe*360-180/\LongueurListe:\RayonListe+\EpaisseurListe/2)node{\color{UGLiRed}...};}
		\def \compt{13}	{\draw (90-\compt/\LongueurListe*360-180/\LongueurListe:\RayonListe+\EpaisseurListe/2)node{\color{UGLiRed} -3};}
		\def \compt{14}	{\draw (90-\compt/\LongueurListe*360-180/\LongueurListe:\RayonListe+\EpaisseurListe/2)node{\color{UGLiRed} -2};}
		\def \compt{15}	{\draw (90-\compt/\LongueurListe*360-180/\LongueurListe:\RayonListe+\EpaisseurListe/2)node{\color{UGLiRed} -1};}



		\def\RayonListe{5}
		\def\EpaisseurListe{1.5}
		\def\LongueurListe{16}
		\draw[fill=UGLiBlue!15] (0,0) circle(\RayonListe+\EpaisseurListe);
		\draw[fill=white] (0,0) circle(\RayonListe);
		\foreach \compt in {0,1,...,\numexpr\LongueurListe-1}
		{\draw (90-\compt/\LongueurListe*360:\RayonListe)--(90-\compt/\LongueurListe*360:\RayonListe+\EpaisseurListe);}

		\foreach \compt in {0,1,2}	{\draw (90-\compt/\LongueurListe*360-180/\LongueurListe:\RayonListe+\EpaisseurListe/2)node{\color{blue}\compt};}
		\foreach \compt in {3,4,5}	{\draw (90-\compt/\LongueurListe*360-180/\LongueurListe:\RayonListe+\EpaisseurListe/2)node{\color{blue}...};}
		\def \compt{6}	{\draw (90-\compt/\LongueurListe*360-180/\LongueurListe:\RayonListe+\EpaisseurListe/2)node{\color{blue} 126};}
		\def \compt{7}	{\draw (90-\compt/\LongueurListe*360-180/\LongueurListe:\RayonListe+\EpaisseurListe/2)node{\color{blue}  127};}
		\def \compt{8}	{\draw (90-\compt/\LongueurListe*360-180/\LongueurListe:\RayonListe+\EpaisseurListe/2)node{\color{blue}128};}
		\def \compt{9}	{\draw (90-\compt/\LongueurListe*360-180/\LongueurListe:\RayonListe+\EpaisseurListe/2)node{\color{blue} 129};}

		\foreach \compt in {10,11,12}	{\draw (90-\compt/\LongueurListe*360-180/\LongueurListe:\RayonListe+\EpaisseurListe/2)node{\color{blue}...};}
		\def \compt{13}	{\draw (90-\compt/\LongueurListe*360-180/\LongueurListe:\RayonListe+\EpaisseurListe/2)node{\color{blue} 253};}
		\def \compt{14}	{\draw (90-\compt/\LongueurListe*360-180/\LongueurListe:\RayonListe+\EpaisseurListe/2)node{\color{blue} 254};}
		\def \compt{15}	{\draw (90-\compt/\LongueurListe*360-180/\LongueurListe:\RayonListe+\EpaisseurListe/2)node{\color{blue} 255};}
	\end{tikzpicture}
\end{center}

\begin{exemple}[s]
\textbf{Comment représenter 97 ?} Ce nombre est positif, on le représente par son écriture binaire sur 8 bits : 0110
0001\\

\textbf{Comment représenter -100 ?} Ce nombre est négatif, il est donc représenté en machine par 256-100=156,
c'est-à-dire 1001 1100\\

\textbf{Que représente 0000 1101 ?} Le bit de poids fort est nul donc cela représente $(0000\ 1101)_2$, c'est à dire
13.\\

\textbf{Que représente 1000 1110 ?} Le bit de poids fort est non nul. $(1000\ 1110)_2=142$ représente $x$ avec donc
$256+x=142$, c'est-à-dire $x=-114$.
\end{exemple}

\begin{methode}
	Pour passer d'un nombre à son opposé en complément à 2, en binaire on procède de \textit{la droite vers la gauche}
	\begin{enumerate}[\textbullet]
		\item 	On garde tous les zéros et le premier 1.
		\item 	On \og inverse\fg{} tous les autres bits.
	\end{enumerate}
\end{methode}

\begin{exemple}
Si on veut l'écriture en complément à 2 de -44 on commence par écrire 44 en base 2:
$$44=(0010\ 1100)_2$$
Puis on applique la méthode précédente :
$$\underbrace{00101}_{\textrm{\tiny on change}}\underbrace{100}_{\textrm{\tiny on garde}}$$
Ce qui nous donne $$\underbrace{11010}_{\textrm{\tiny on a changé}}\underbrace{100}_{\textrm{\tiny on a gardé}}$$
Ainsi la représentation de -44 en complément à 2 sur 8 bits est 1101 0100.

\end{exemple}
Ce qui est agréable, c'est que \textbf{l'addition naturelle est compatible avec cette représentation} dans la mesure où
:
\begin{enumerate}[\textbullet]
\item 	On ne dépasse pas la capacité : on n'ajoutera pas 100 et 120 car cela dépasse 127.
\item 	Si on ajoute deux octets et que l'on a une retenue à la fin de l'addition (ce serait un 9\eme bit), alors
celle-ci n'est pas prise en compte.
\end{enumerate}

\begin{exemple}
\textbf{Ajoutons les représentations de 97 et -100 :}
\begin{center}

\begin{tabular}{cccccccccc}
  &  & 0 & 1 & 1 & 0 & 0 & 0 & 0 & 1 \\

+ &  & 1 & 0 & 0 & 1 & 1 & 1 & 0 & 0 \\
\hline
= &  & 1 & 1 & 1 & 1 & 1 & 1 & 0 & 1 \\
\end{tabular}
\end{center}

On a $(1111\ 1101)_2=253$, il représente donc $x$ sachant que $256+x=253$, c'est à dire x=-3.\\

On retrouve bien 97+(-100)=-3.
\end{exemple}

Attention à ne pas dépasser la capacité du format : en ajoutant 120 et 20 on obtient 140 qui, étant plus grand que 128, représente 140-156 = -116. C'est ce qu'affiche le programme suivant.
\begin{minted}{cpp}
#include <iostream> // nécessaire pour utiliser cout

int main() // début de la fonction main
{
    char c1 = 120; // on définit une première variable
    char c2 = 20; // puis une deuxième
    char c3 = c1+c2; // on les ajoute
    std::cout << (int) c3; // on affiche le résultat en base 10
    return 0; // la fonction main renvoie traditionnellement zéro
}
\end{minted}

\section{Les principaux formats}

En général, dans la majorité des langages (C, C++, C\#, Java par exemple)  les types suivants
sont utilisés pour représenter les entiers relatifs (les noms peuvent varier d'un langage à l'autre):
		\begin{enumerate}[\textbullet]
		\item	\tw{char} :  Pour représenter les entiers compris entre -128 et +127. Nécessite un octet.
		\item 	\tw{short} :  Pour des entiers compris entre \np{-32768} et \np{37267} ($-2^{15}$ et $2^{15}-1$). Codage sur 2
		octets.
		\item 	\tw{int} :  (4 octets) entiers compris entre \np{-2147483648} et \np{+2147483647} ($-2^{31}$ et $2^{31}-1$).
		\item 	\tw{long} : (8 octets) entiers compris entre \np{-9223372036854775808} et\\ \np{+9223372036854775807} ($-2^{63}$ et
		$2^{63}-1$).
		\end{enumerate}

\section{Quelques ordres de grandeur}
\begin{enumerate}[\textbullet]
	\item 	1 kilooctet = 1 ko =  \np{1000} octets (fichiers textes)
			\item 	1 mégaoctet = 1 Mo = \np{1000} (fichiers .mp3)
			\item 	1 gigaoctet = 1 Go = \np{1000} (RAM des PC actuels, jeux)
			\item 	1 téraoctet = 1 To = \np{1000} (Disques durs actuels)
			\item 	1 pétaoctet = 1 Po = \np{1000} To
			\item  	1 exaoctet = 1 Eo = \np{1000} Po = $10^{18}$ octets (trafic internet mondial mensuel prévu en 2022 : 376Eo)
\end{enumerate}

\exostart


\begin{exercice}
	On considère le code \textsc{C++} suivant :
\begin{minted}{cpp}
#include <iostream> // bibliothèque d'affichage
int main() // début de la fonction principale
{
    unsigned char c = 0; // on définit la variable c
    for (int i = 0; i < 300; i++) // on fait une boucle pour
    {
        std::cout << "valeur de i : " << i;
        // on affiche la valeur de i
        std::cout << "  et valeur de c : " << (int)c;
         // on affiche la valeur de c en base 10
        std::cout << endl; // on revient à la ligne (END Line)
        c++; // on augmente c
    }
    return 0; // la fonction principale renvoie zéro
}
\end{minted}

Voilà ce que la console affiche :\\

\texttt{valeur de i : 0  et valeur de c : 0}\\
\texttt{valeur de i : 1  et valeur de c : 1}\\
\textit{et c\ae tera}\\
\texttt{valeur de i : 254  et valeur de c : 254}\\
\texttt{valeur de i : 255  et valeur de c : 255}\\
\texttt{valeur de i : 256  et valeur de c : 0}\\
\texttt{valeur de i : 257  et valeur de c : 1}\\
\textit{et c\ae tera}\\
\texttt{valeur de i : 298  et valeur de c : 42}\\
\texttt{valeur de i : 299  et valeur de c : 43}\\

Comment expliquer ceci ?

\end{exercice}

\begin{exercice}
	On considère le code \textsc{C++} suivant :
\begin{minted}{cpp}
#include <iostream> // bibliothèque d'affichage
int main() // début de la fonction principale
{
    char c = 0; // on définit la variable c
    for (int i = 0; i < 257; i++) // on fait une boucle pour
    {
        std::cout << "valeur de i : " << i;
        // on affiche la valeur de i
        std::cout << "  et valeur de c : " << (int)c;
        // on affiche la valeur de c en base 10
        std::cout << endl; // on revient à la ligne (END Line)
        c++; // on augmente c
    }
    return 0; // la fonction principale renvoie zéro
}
\end{minted}

	Voilà ce que la console affiche :\\

	\texttt{valeur de i : 0  et valeur de c : 0}\\
	\texttt{valeur de i : 1  et valeur de c : 1}\\
	\textit{et c\ae tera}\\
	\texttt{valeur de i : 126  et valeur de c : 126}\\
	\texttt{valeur de i : 127  et valeur de c : 127}\\
	\texttt{valeur de i : 128  et valeur de c : -128}\\
	\texttt{valeur de i : 129  et valeur de c : -127}\\
	\textit{et c\ae tera}\\
	\texttt{valeur de i : 254  et valeur de c : -2}\\
	\texttt{valeur de i : 255  et valeur de c : -1}\\
	\texttt{valeur de i : 256  et valeur de c : 0}\\


	Comment expliquer ceci ?

\end{exercice}



\begin{exercice}[]
	Donner les représentations en complément à deux sur un octet de :\\ 88, 89, 90, -125, -2 et -3.
\end{exercice}

\begin{exercice}
 	Donner les représentations en complément à 2 (sur un octet) de -1 et de 92.\\
 	Ajouter ces deux représentations (en ignorant la dernière retenue). Quel nombre représente cette somme ?\\
\end{exercice}
\end{document}






