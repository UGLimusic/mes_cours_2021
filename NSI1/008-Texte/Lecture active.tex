\documentclass[a4paper,12pt,french]{article}
\usepackage[margin=2cm]{geometry}
\usepackage[thinfonts,latinmath]{uglix2}
\pagestyle{empty}
\nouveaustyle
\begin{document}
\titre{CH08- Représentation du texte}{NSI1}{2021}

\subsection*{Lecture active}

Lire le cours sur \link{https://impytoyable.fr}{impytoyable.fr} et répondre aux questions au fur et à mesure.

\begin{enumerate}[\bfseries 1.]
	\item 	Combien y a -t-il de caractères dans ce catalogue ?\\
            Combien de bits sont nécessaires pour pouvoir représenter tous leurs numéros de code ?
	\item 	Voici un message reçu à l'issue d'une transmission : 53 E1 6C F5 70\\
            Ces 6 octets sont censés représenter 6 caractères ASCII codés sur 7 bits, le 8ème étant réservé au contrôle d'erreur par parité.\\
            Décoder ces 6 octets en disant s'il y a des erreurs ou non.\\
            Quel était le message initial ?
    \item   Combien de nouveaux symboles a-t-on pu coder en autorisant le huitième bit dans le codage ?
    \item   Ces deux encodages sont-ils totalement compatibles ? Pourquoi ?
    \item   Lequel de ces encodages semble le plus performant ?
    \item   Si un ordinateur lit cet encodage UTF-8 du symbole € selon l'encodage ISO8859-15, qu'affichera-t-il ?
\end{enumerate}

\subsection*{Activités}

Faire les 3 activités de fin de chapitre.
\end{document}
