\documentclass[a4paper,10pt,french]{article}
\usepackage[margin=2cm]{geometry}
\usepackage[thinfonts,latinmath]{uglix2}
\pagestyle{empty}
\nouveaustyle
\begin{document}

\titreinterro{Rendu de monnaie glouton}{NSI1}{2022}	

On veut rendre la monnaie en euros\textbf{avec le moins de pièces et de billets possible}.\\
Il suffit de commencer par les plus gros billets ou les plus grosses pièces.\\

Comment rends-tu 59 € ?\\

\carreauxseyes{16.8}{2.4}\\

Comment rends-tu 28,81 € ?\\

\carreauxseyes{16.8}{2.4}\\

On va essayer d'écrire l'algorithme associé à cette méthode.\\

Pour que ce soit facile :
\begin{enumerate}[--]
	\item 	on va écrire toutes les valeurs des pièces et des billets en centimes ;
	\item 	on va trier dans l'ordre décroissant, ce qui nous aidera à « commencer par les plus gros ».
\end{enumerate}
On crée donc la variable suivante\\ 

\mintinline[fontsize=\scriptsize]{python}{liste_pieces = [20_000, 10_000, 5_000,2_000,1_000, 500, 200, 100, 50, 20, 10, 5, 2, 1]} \\

Tu vas expliquer en détail comment tu procèdes pour rendre 59 € : 
\begin{enumerate}[--]
	\item 	On considère la variable \pythoninline{montant} valant \pythoninline{5_900} (on l'a convertie en centimes) ;
	\item 	Que faire avec avec \pythoninline{liste_pieces} ?
    \item 	Y a-t-il une ou plusieurs boucles ?
    \item  	A-t-on besoin d'autres variables ?\\
\end{enumerate}
 
\carreauxseyes{16.8}{3.2}\\

\carreauxseyes{16.8}{3.2}\\

\'Ecris maintenant cet algorithme (en langage naturel ou en Python).\\

\carreauxseyes{16.8}{8}\\

\textbf{Pour les expert.e.s}\\

Pour prouver cet algorithme on a besoin 
\begin{enumerate}[--]
	\item 	d'un variant de boucle pour la terminaison ;
	\item 	d'un invariant de boucle pour la correction.
\end{enumerate}
Donne ces deux éléments.\\

\carreauxseyes{16.8}{8}
\end{document}